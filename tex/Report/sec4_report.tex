\label{sec: linearization}
Let us return to the control equations for \textsc{SPr4} given by (\ref{eq:dynamical system}). The response of the control system is characterized by the two matrix valued functions $F_c, F_{\theta}: \SO(3) \times \R^4 \to M_{3 \times 4}(\R)$ which can by Proposition \ref{prop: rotational invariance} can be factorized as:
\begin{equation}
\label{eq: reminder control system}
	F_{c}(R, \zeta) = R F_{c}(\zeta) \text{ and } F_{\theta}(R, \xi) = R F_{\theta}(\zeta),
\end{equation}
with $F_{c}(\zeta) := F_c(I, \zeta)$ and $F_{\theta}(\zeta) := F_{\theta}(I, \zeta)$. Hereinafter, we suppose that $\zeta := \xi_0 + \xi$ with $\xi_0 \in \M$ having all its components equal. Furthermore, we set $F_{c, \xi_0}(\xi) := F_{c}(\xi_0 + \xi)$ and analogously $F_{\theta, \xi_0}(\xi) := F_{\theta}(\xi_0 + \xi)$. It has been shown in \cite{Alouges2013} that $F$ and thus both $F_{ c, \xi_0}$ and $F_{\theta, \xi_0}$ are analytic functions. Therefore, we can consider their first order expansions in $\xi$:
\begin{align}
\label{eq: spatial control expansion}
	F_{c, \xi_0}(\xi)\eta &= F_{c, 0} \eta  + \h_{c,0}(\xi \otimes \eta) + \mathcal{O}(|\xi|)\eta\\
\label{eq: angular control expansion}
	F_{\theta, \xi_0}(\xi) \eta &= F_{\theta, 0} \eta + \h_{\theta, 0}(\xi \otimes \eta) + \mathcal{O}(|\xi|)\eta,
\end{align}
where $F_{c,0} := F_{c}(\xi_0) \in M_{3 \times 4}(\R)$, $\h_{c,0}\in \mathcal{L}(\R^4 \otimes \R^4, \R^3)$ represents the first order derivative of $F_{c, \xi_0}$ at $\xi = 0$ and for $F_{\theta, \xi}$ the analogous definitions are made.
The purpose of this section is to reveal the structure of the different terms in the expansions (\ref{eq: spatial control expansion}) and (\ref{eq: angular control expansion}) in light of the symmetry properties fulfilled by $F_{c}$ and $F_{\theta}$ due to Propositions \ref{prop: spatial permutation invariance} and \ref{prop: angular permutation invariance} , i.e.
\begin{align}
\label{eq: spatial and angular first order expansion}
	F_{c}(P_{ij} \xi) = S_{ij} F_{c}(\xi) P_{ij}& & \text{ and } & & 			F_{\theta}(P_{ij} \xi) = - S_{ij} F_{\theta}(\xi) P_{ij} & & \forall \xi \in \M.
\end{align}

The following slightly generalized statement of Lemma 9 in \cite{Alouges2017} proves useful in our case as well. However, the proof, being exactly the same, is omitted.

\begin{lemma}
\label{lem: general symmetries of expansion terms}
Let $G: \R^n \to M_{m \times n}(\R)$ be an analytic function and $S \in M_{m \times m}(\R)$ and $P \in M_{n \times n}(\R)$ matrices such that $G(P \xi) = S G(\xi) P$ for every $\xi \in \R^n$. For $\xi_0 \in \R^n$ with all components equal, set $G_{\xi_0}(\xi) := G(\xi_0 + \xi)$ and write the first order expansion
\begin{equation}
	G_{\xi_0}(\xi)\eta = G_0 \eta + \h_0(\xi \otimes \eta) + \mathcal{O}(|\xi|)\eta.
\end{equation}
Then we have
\begin{equation}
	G_0 = S G_0 P,
\end{equation}
and
\begin{eqnarray}
	\h_0((P \xi) \otimes \eta) = S \h_0(\xi \otimes (P \eta)) &  & \forall \xi, \eta \in \R^n.
\end{eqnarray}
\end{lemma}

\subsection{The zeroth order terms $F_{c,0}$ and $F_{\theta, 0}$} 
One could directly use Lemma \ref{lem: general symmetries of expansion terms} to obtain the terms of order zero up to a scalar by solving a system of two matrix equations as it was done in \cite{Alouges2017}. Yet, there is a direct geometric argument, which incidentally also works for \textsc{SPr3} in \cite{Alouges2017},  that reveals their structure.
Let us denote by $A^{(i)}$ the $i$-th column of a matrix $A \in M_{m \times n}(\R)$. Then we note that for the spatial term $F_{c,0}$ and any $i \in \N_4$ we obtain from Lemma \ref{lem: general symmetries of expansion terms} and Proposition \ref{prop: spatial permutation invariance}
\begin{equation}
\label{eq:zeroth spatial term condition 1}
	F_{c,0}^{(i)} = F_{c,0} e_i = S_{ij} F_{c,0} P_{ij} e_i = S_{ij} F_{c,0} e_j = S_{ij} F_{c,0}^{(j)}.
\end{equation}
Similarly, for any $k \in \N_4$ one has
\begin{equation}
\label{eq:zeroth spatial term condition 2}
F_{c,0}^{(k)} = S_{ij} F_{c,0} P_{ij} e_k = S_{ij} F_{c,0} e_k = S_{ij} F_{c,0}^{(k)}.
\end{equation}
Recall that $S_{ij}$ is the reflection that maps the arm $||i$ onto arm $||j$ and vice-versa in the reference orientation, i.e. where $||i$ is collinear to $z_i$, the $i$-th arm of the reference tetrahedron defined in section \ref{sec:modeling}. In particular, the plane of reflection is defined by the origin and the remaining arms of the reference tetrahedron, i.e. by the vectors $z_k$ and $z_l$. Thus, equation (\ref{eq:zeroth spatial term condition 2}) implies that $F_{c,0}^{(k)} \in \Span\{z_k, z_l\}$ as $F_{c,0}^{(k)}$ apparently is an eigenvector associated to the eigenvalue 1 of $S_{ij}$. Yet, the same argument shows that $F_{c,0}^{(k)} \in \Span\{z_k, z_i\}$ and $F_{c,0} \in \Span\{z_k, z_j\}$ and the vectors $z_i, z_j$ and $z_l$ being linearly independent implies that $F_{c,0}^{(k)} = a_k z_k$ for some $a_k \in \R$. Due to equation (\ref{eq:zeroth spatial term condition 1}) and the fact that $S_{ij}$ is orthogonal, we have $|a_i| = |a_j| = |a_k| = |a_l|$. Finally, we note that the quantity $a_k z_k \cdot z_k$ stays constant under change of indices again in consequence of the symmetry conditions from Lemma \ref{lem: general symmetries of expansion terms}. Hence, we have $a_1 = a_2 = a_3 =a_4 := \mathfrak{a}$ and therefore
\begin{equation}
	F_{c,0} = \mathfrak{a} (z_1 |z_2| z_3|z_4).
\end{equation}

In the following sections, we will exploit the orthonormal basis of $\R^4$ consisting of the vectors $\tau_1 : = \tfrac{1}{\sqrt{6}}(-2,1,1,0)^T$, $\tau_2 := \tfrac{1}{\sqrt{2}}(0,1,-1,0)^T$, $\tau_{3}:= \tfrac{1}{2 \sqrt{3}} (1,1,1,-3)^T$ and $\tau_4:= \tfrac{1}{2}(1,1,1,1)^T$, in terms of which $F_{c,0}$ can be written as $F_{c, 0} = -3 \sqrt{3} \mathfrak{a} [\tau_1|\tau_2| \tau_3 ]^T$.

For the angular term $F_{\theta,0}$ we find by Lemma \ref{lem: general symmetries of expansion terms}, Proposition \ref{prop: angular permutation invariance} and a similar argument for any $k \in \N_4$ that
\begin{equation}
F_{\theta, 0}^{(k)} = -S_{ij} F_{\theta, 0}^{(k)}.
\end{equation}
This means that in this case $F_{\theta,0}^{(k)}$ is an eigenvector to the eigenvalue $-1$ and therefore must be orthogonal to the plane of reflection, i.e. $\Span\{z_k, z_l\}$. The same is true for the reflections $S_{il}$ and $S_{jl}$ and hence $F_{c,0}^{(k)}$ is in particular orthogonal to $z_i, z_j, z_l$. This eventually implies that $ F_{\theta, 0} = 0$ since the latter three vectors form a basis of $\R^3$.

\begin{remark}
First, we observe that the upper left corner of $F_{c,0}$ corresponding to the arms $||1, ||2$ and $||3$ up to multiplication by a scalar is the same as for \textsc{SPr3} in \cite{Alouges2017}. Furthermore, one notes that physically it is  clear that $F_{\theta,0}$ must vanish since by hypothesis $\xi_0$ has all its components equal and thus the swimmer is in a symmetric shape at $\xi = 0$. Therefore, the balls moving along their axes cannot create any torque.
\end{remark}


%However, setting $\mathfrak{a} := -3/\sqrt{2} a$ yields the slightly more convenient form
%\begin{equation}
%	F_{c, 0}  = \left ( \begin{array}{cccc}
%	- 2 \mathfrak{a}  & \mathfrak{a}  & \mathfrak{a}  & 0 \\ 
%	0 & \sqrt{3}\mathfrak{a}  & -\sqrt{3}\mathfrak{a}  & 0 \\ 
%	 \tfrac{1}{\sqrt{2}}\mathfrak{a} & \tfrac{1}{\sqrt{2}}\mathfrak{a} & \tfrac{1}{\sqrt{2}} \mathfrak{a}  & \tfrac{-3}{\sqrt{2}}\mathfrak{a} 
%	\end{array} \right ) \text{ with } \mathfrak{a} \in \R.
%\end{equation}
%One recognizes in the top left corner of $F_{c,0}$ the zeroth order term of the control system associated to the swimmer \textsc{SPr3} in \cite{Alouges2017}. Further

%By applying Lemma \ref{lem: general symmetries of expansion terms} to $F_{c, \xi_0}$ and $F_{\theta, \xi_0}$, we obtain two linear systems of matrix equations in the unknowns $F_{c, 0}$ and $F_{\theta,0}$. To solve these two systems, we eventually have to determine at least some of the matrices $S_{ij}$. Let $S_{kl}(\phi)$ denote the reflection at the plane orthogonal to the $\hat{e}_k-\hat{e}_l$-plane making an angle of $\phi$ with the $\hat{e}_k$-axis. By geometrical inspection of the reference tetrahedron $(S_1, S_2, S_3, S_4)$, we find that
%\begin{eqnarray}
%	S_{12} = S_{12}\left (\tfrac{2\pi}{3}\right ), & S_{23} = S_{12}(0),  & S_{13} = S_{13}\left (\tfrac{\pi - \alpha_{\mathrm{tet}}}{2}\right ),
%\end{eqnarray}
%where $\alpha_{\mathrm{tet}} = \arccos (-1/3)$ denotes the angle between two legs of a regular tetrahedron. Indeed, it happens that these three matrices are enough to determine both terms of order and one finds that $F_{\theta, 0} = 0$ and


%\begin{remark}
%First, we observe that the upper left corner of $F_{c,0}$ corresponding to the arms $||1, ||2$ and $||3$ is exactly the same as for \textsc{SPr3} in \cite{Alouges2017}. Furthermore, one notes that physically it is  clear that $F_{\theta,0}$ must vanish since by hypothesis $\xi_0$ has all its components equal and thus the swimmer is in a symmetric shape at $\xi = 0$. Therefore, the balls moving along their axes cannot create any torque. Lastly, one should note that apparently $F_{c,0} \sim (z_1 |z_2|z_3|z_4)$ which can be shown directly using the symmetry properties.
%\end{remark}

\subsection{The first order terms $\h_{c,0}$ and $\h_{\theta,0}$}
Following the approach in \cite{Alouges2017}, we evaluate the tensors $\h_{c,0}$ and $\h_{\theta,0}$ on the basis $(e_{i} \otimes e_{j})_{i,j \in \N_4}$. Setting $A_k := (\h_{c,0}(e_i \otimes e_j)\cdot \hat{e}_k)_{i,j \in \N_4}$ and $B_k:= (\h_{\theta,0}(e_i \otimes e_j)\cdot L_k)_{i,j \in \N_4}$ for $k \in \N_3$, we can write the vectors $\h_{c,0}(\xi \otimes \eta), \h_{\theta,0}(\xi \otimes \eta) \in \R^3$ for any $\xi, \eta \in \R^3$ as
\begin{equation}
\label{eq: matrix representation of spatial first order term}
	\h_{c,0}(\xi \otimes \eta) = \sum_{k \in \N_3}(A_k \eta \cdot \xi)\hat{e}_k, 
\end{equation}
and similarly
\begin{equation}
\label{eq: matrix representation of angular first order term}
	\h_{\theta, 0}(\xi \otimes \eta) = \sum_{k \in \N_3}(B_k \eta \cdot \xi) L_k.
\end{equation}

We could pursue the approach of \cite{Alouges2017} and directly calculate the matrices $A_k$ and $B_k$. However, as we shall see later, the dynamics of \textsc{SPr4}, up to higher order terms in the norm of the control curve $\xi$, will only be governed by their skew symmetric parts. Thus, we will evade this strenuous task and we determine the skew symmetric matrices $M_k := \tfrac{1}{2}(A_k - A_k^T)$ and $M_{k + 3}:= \tfrac{1}{2}(B_k - B_k^T)$ for $k \in \N_3$, up to two scalar parameters, using a geometric argument similar to the one used in the previous section. Nevertheless, it is possible to calculate the symmetric parts of the matrices $A_k$ and $B_k$ using similar geometric arguments, c.f. appendix, to obtain a complete description of the dynamics.


To that end, we notice that Lemma \ref{lem: general symmetries of expansion terms} together with the fact that $(P_{ij})^2 = I$ yields for all $i,j \in \N_4$ and for all $\xi, \eta \in \N_4$
\begin{equation}
	S_{ij} \h_{c,0}(P_{ij} \xi \otimes P_{ij} \eta) = \h_{c,0}(\xi \otimes \eta),
\end{equation}
as well as
\begin{equation}
	-S_{ij} \h_{\theta,0}(P_{ij} \xi \otimes P_{ij} \eta) = \h_{\theta, 0}(\xi \otimes \eta).
\end{equation}
Next, we define $\mathcal{K}_{c,0}(\xi \otimes \eta) := \tfrac{1}{2}[\h_{c,0}(\xi \otimes \eta) - \h_{c,0}(\eta \otimes \xi)]$ and similarly $\mathcal{K}_{\theta,0}$ such that
\begin{equation}
	M_k = (\mathcal{K}_{c,0}(e_i \otimes e_j) \cdot \hat{e}_k)_{i,j \in \N_4} \text{ and } M_{k + 3} = (\mathcal{K}_{\theta,0}(e_i \otimes e_j) \cdot L_k)_{i,j \in \N_4},\;\; k \in \N_3.
\end{equation} 
In particular, it is clear that $\mathcal{K}_{c,0}$ and $\mathcal{K}_{\theta,0}$ satisfy the same symmetry relations as $\h_{c,0}$ and $\h_{\theta, 0}$, respectively, and additionally, we have $\mathcal{K}_{c,0}(e_i \otimes e_j) = - \mathcal{K}_{c,0}(e_j \otimes e_i)$ and $\mathcal{K}_{\theta,0}(e_i \otimes e_j) = - \mathcal{K}_{\theta,0}(e_j \otimes e_i)$.

For the spatial part, we deduce from the symmetry properties above that for all $i, j \in \N_4$	
\begin{equation}
	\K_{c,0}(e_i \otimes e_j) = S_{ij} \K_{c,0}(e_j \otimes e_i) = -S_{ij} \K_{c,0}(e_i \otimes e_j)
\end{equation}
and therefore $\K_{c,0}(e_i \otimes e_j)$ is an eigenvector associated to the eigenvalue $-1$ of the reflection $S_{ij}$. The reflection $S_{ij}$ taking place at the plane passing through the two remaining arms of the reference tetrahedron $z_k$ and $z_l$, implies that $\K_{c,0}(e_i \otimes e_j) = \alpha_{ij}( z_k \times z_l)$ for some scalar $\alpha_{ij} \in \R$. Additionally, we have $\K_{c,0}(e_i \otimes e_j) = S_{jk} \K_{c,0}(e_i \otimes e_k) = \alpha_{ik}(z_j \times z_l)$ and since $S_{jk}$ is orthogonal, we have $|\alpha_{ij}| = |\alpha_{ik}|$ as the vectors $z_i$ are normalized. Eventually, one quickly verifies that the quantity
\begin{equation}
	\K_{c,0}(e_i \otimes e_j) \cdot \sgn(ijkl) (z_k \times z_l),
\end{equation}
where $\sgn(ijkl)$ denotes the parity of the permutation $(ijkl)$ of $\N_4$, stays constant under any permutation of the indices as well as any symmetry condition. Hence, we may conclude that
\begin{equation}
	\K_{c,0}(e_i \otimes e_j) = \alpha \sgn(ijkl) (z_k \times z_l),
\end{equation}
for all $i \neq j \in \N_4$ and some scalar $\alpha \in \R$. Clearly, the symmetry conditions imply that $\K_{c,0}(e_i \otimes e_i) = 0$ for all $i \in  \N_4$. Thus, we have determined the matrices $M_1, M_2, M_3$ up to one scalar parameter. By explicitly calculating the cross products $z_i \times z_j$, we find
\renewcommand{\arraystretch}{1.1}
\begin{align}
\label{eq: M1 and M2}
M_1 = \alpha \left ( \begin{array}{cccc}
0 & 3 & 3 & 2 \\ 
-3 & 0 & 0 & -1 \\ 
-3 & 0 & 0 & -1 \\ 
-2 & 1 & 1 & 0
\end{array} \right ) &, &
M_2 = \sqrt{3} \alpha \left (
\begin{array}{cccc}
0 & 1 & -1 & 0 \\ 
-1 & 0 & -2 & -1 \\ 
1 & 2 & 0 & 1 \\ 
0 & 1 & -1 & 0
\end{array} \right),
\end{align}
and
\begin{align}
\label{eq: M3}
M_3 = 2 \sqrt{2} \alpha \left (
\begin{array}{cccc}
0 & 0 & 0 & -1 \\ 
0 & 0 & 0 & -1 \\ 
0 & 0 & 0 & -1 \\ 
1 & 1 & 1 & 0
\end{array} 
\right ).
\end{align}
Similarly, for the angular part, we find that
\begin{equation}
	\K_{\theta, 0}(e_i \otimes e_j) = - S_{kl} \K_{\theta,0}(e_i \otimes e_j)
\end{equation}
and therefore $\K_{\theta,0}(e_i \otimes e_j) = \delta_{ij} e_i \times e_j$. By noticing that this time the quantity $\K_{\theta,0}(e_i \otimes e_j) \cdot (z_i \times z_j)$ stays constant, a similar argument to the one above shows that
\begin{equation}
\K_{\theta,0}(e_i \otimes e_j) = \delta (z_i \times z_j),
\end{equation}
for all $i \neq j \in \N_4$. Again, we have $\K_{\theta}(e_i \otimes e_i) = 0$ for all $i \in \N_4$. A calculation similar to the one above now yields
\begin{align}
\label{eq: M4 and M5}
	M_4 = \delta \left (
	\begin{array}{cccc}
	0 & 1 & 1 & 0 \\ 
	-1 & 0 & -2 & 3 \\ 
	1 & 2 & 0 & -3 \\ 
	0 & -3 & 3 & 0
	\end{array} 
	\right ) &, & M_5 = \sqrt{3} \delta \left ( \begin{array}{cccc}
	0 & -1 & -1 & 2 \\ 
	1 & 0 & 0 & -1 \\ 
	1 & 0 & 0 & -1 \\ 
	-2 & 1 & 1 & 0
	\end{array} \right ) ,
\end{align}
and
\begin{align}
\label{eq: M6}
M_6 = 2 \sqrt{2} \delta \left (\begin{array}{cccc}
0 & 1 & -1 & 0 \\ 
-1 & 0 & 1 & 0 \\ 
1 & -1 & 0 & 0 \\ 
0 & 0 & 0 & 0
\end{array}  \right ).
\end{align}

\begin{remark}
At this point, let us point out the apparent similarity between the matrices $M_1, M_2$ and $M_6$ with their corresponding matrices in \cite{Alouges2017}. In fact, in the upper left corner, i.e. the entries that relate the first three arms to each other, we retrieve the same matrices as in \cite{Alouges2017} up to rescaling $\alpha$ and $\delta$, which very well reflects the similarity between \textsc{SPr3} and \textsc{SPr4}. However, the fact that it is the first three arms that corresponds to the three arms in \textsc{Spr3} merely stems from our choice of the reference orientation.
\end{remark}

\subsection{The linearized control equations}
Herein, we denote by $J$ the closed interval  $[0,2\pi] \subset \R$ and we define the so-called \emph{strokes space} as $H_\sharp^1(J, \R^4)$, i.e. the Sobolev space of $2\pi$-periodic vector valued functions of $L_{\sharp}^2(J, \R^4)$ having first order weak derivative in $L_{\sharp}^2(\R^4)$. For every $f \in L^{2}_{\sharp}(J, \R^4)$ we denote by $\langle f \rangle := (2\pi)^{-1} \int_{J} f(s) \dd s$ the average of $f$ on $J$.

In the previous section, we have seen that the control system governing the evolution of \textsc{SPr4} under the action of the control parameters $\zeta \in \M$ is given by
\begin{align}
\begin{cases}
	\dot{c} &= R F_c(\zeta) \dot{\zeta}\\
	\dot{R} &= R F_{\theta}(\zeta) \dot{\zeta},
\end{cases}
\end{align}
where $(c,R) \in \mathcal{P} = \R^3 \times \SO(3)$, the systems $F_{c}, F_{\theta}: \M \to M_{3 \times
 4}(\R)$ are given by (\ref{eq: reminder control system}) and $\dot{\zeta}  \in T_{\zeta}\M$. Furthermore, we have seen previously, c.f. (\ref{eq: spatial and angular first order expansion}), (\ref{eq: matrix representation of spatial first order term}) and (\ref{eq: matrix representation of angular first order term}), that if we set $\zeta = \xi_0 + \xi$, the response of the system around $\xi = 0$, up to higher order terms, simplifies to
 \begin{align}
 \label{eq: dynamics first approx}
 \begin{cases}
 	\dot{c} &= R F_{c,0} \dot{\xi} + R \sum_{k \in \N_3}(A_k \dot{\xi} \cdot \xi)\hat{e_k}\\
 	\dot{R} &= R \sum_{k \in \N_3} (B_k \dot{\xi} \cdot \xi) L_k.
 \end{cases}
 \end{align}
In particular, if we fix $\xi \in H^1_{\sharp}(J, \R^4)$ and define $\Gamma := \sum_{k \in \N_3} (B_k \dot{\xi} \cdot \xi): J \to \so(3)$, then the dynamics of $R$ can be written as an ordinary differential equation on the Lie group $\SO(3)$:
\begin{align}
\label{eq: orientation ode}
\begin{cases}
	\dot{R}(t) = R(t) \Gamma(t)\\
	R(0) := R_0.
\end{cases}
\end{align}

To simplify equations  (\ref{eq: dynamics first approx}) further, we are interested in the solution of (\ref{eq: orientation ode}) in the regime of a small stroke $\xi \in H_{\sharp}^1(J, \R^4)$ or equivalently in the regime of a small vector field $\Gamma$. Intuitively, the solution $R$ of (\ref{eq: orientation ode}) should not deviate too much from the initial value $R_0$ if the vector field $\Gamma$ driving the differential equation is small. The solution of (\ref{eq: orientation ode}) and its relation to a small vector field $\Gamma$ as well as the notion of smallness for the vector field $\Gamma$ in the first place, are mathematically formalized by the concept of \emph{chronological calculus}. For details on the topic, we refer to \cite{Agrachev2004} but essentially it works as follows: First, one identifies any smooth manifold $M$ with the space $C^{\infty}(M)$, on which one defines a certain metric topology, the Whitney topology. Then, the solution of a differential equation $\dot{q}(t) = q(t) V(t)$ for $V$ a non-autonomous vector field on $M$ is given by
\begin{equation}
	q(t) = q(0)\,\chroexp \int_{0}^{t} V(s) \dd s,
\end{equation}
where $\chroexp$ is a special operator called the \emph{right chronological exponential}. It is defined as a limit of an iterated integral, see \cite{Agrachev2004}. For the series expansion
\begin{equation}
\label{eq: chron exp series expansion}
S_m(t) := I + \sum_{n = 1}^{m-1} \int_{\Delta_n(t)} \dotsm \int V(s_n) \circ \dotsm \circ V(s_1) \dd s_n \dotsm \dd s_1,
\end{equation}
with $\Delta_n(t) = \{(s_1, \dotsc, s_n) \in \R^n | 0 \leq s_n \leq \dotsm \leq s_1 \leq t\}$, we have
\begin{align}
q(t) = S_m(t) + \mathcal{O}(t^m), & & t \downarrow 0.
\end{align}
In particular, for an equation of the form $\dot{q}(t) = q(t) \varepsilon V(t)$, it is shown in \cite{Agrachev2004}, that
\begin{align}
\label{eq: estimation of chronological exponential}
	q(t) = S_m^{\varepsilon}(t) + \mathcal{O}(\varepsilon^m), & & \varepsilon \downarrow 0,
\end{align}
where $S_{m}^{\varepsilon}$ denotes the series expansion (\ref{eq: chron exp series expansion}) for the vector field $\varepsilon V$.

With the estimate (\ref{eq: estimation of chronological exponential}) at hand, let $\hat{\xi} \in H^1_{\sharp}(J, \R^4)$ be a normalized stroke, i.e. $||\hat{\xi}||_{H_{\sharp}^1} = 1$, and $\varepsilon > 0$. Set $\xi := \varepsilon \hat{\xi}$ as well as $\Gamma_\varepsilon := \sum_{k \in \N_3}(B_k \dot{\xi} \cdot \xi) L_k$ such that $\Gamma_1 = \sum_{k \in \N_3}(B_k \dot{\hat{\xi}} \cdot \hat{\xi}) L_k$ and  $\Gamma_\varepsilon = \varepsilon^2 \Gamma_1$. Writing $S_{m}^{\epsilon}$ for the expansion (\ref{eq: chron exp series expansion}) of the vector field $\Gamma_\varepsilon$, we find by (\ref{eq: estimation of chronological exponential})
\begin{align}
	R(t) = R_0 \left ( I + \int_{0}^{t} \Gamma_\varepsilon(\tau) \dd\tau \right ) + \mathcal{O}(\epsilon^4), & & \varepsilon \downarrow 0.
\end{align}
Hence, choosing $R_0 = I$, we have in particular the following approximations for any $t \in J $
\begin{align}
\label{eq: linearized ode}
\begin{cases}
\dot{c} &= \left ( I + \int_{0}^{t} \Gamma_\varepsilon(\tau) \dd\tau \right ) \left ( F_{c,0} \dot{\xi} + \sum_{k \in \N_3} (A_k \dot{\xi} \cdot \xi) \hat{e}_k \right )  + \mathcal{O}(\varepsilon^4)\\
\dot{R} &= \left ( I + \int_{0}^{t} \Gamma_\varepsilon(\tau) \dd\tau \right ) \sum_{k \in \N_3}(B_k \dot{\xi} \cdot \xi) L_k + \mathcal{O}(\varepsilon^4),
\end{cases}
\end{align}
for $\varepsilon\downarrow 0$. By integrating the previous two relations over $J$, we find an estimate of the net displacement undergone by the center $c$ of \textsc{SPr4} as well as its orientation $R$ after a small stroke. Moreover, with equations (\ref{eq: linearized ode}) we can express the net displacements $\delta c$ and $\delta R$ as maps $H_{\sharp}^1(J,\R^4) \to \R^3$ and $H_{\sharp}^1(J,\R^4) \to \so(3)$, respectively, given by $\xi \mapsto 2 \pi \langle \dot{c}(\xi) \rangle $ and $\xi \mapsto 2 \pi \langle \dot{R}(\xi) \rangle$, respectively. Consequently, let us prove that

\begin{proposition}
\label{prop:net displacement}
For any $\xi \in H_\sharp^1(J, \R^4)$, in a neighborhood of $0 \in H_{\sharp}^{1}(J, \R^4)$, the following estimates hold
\begin{equation}
\begin{aligned}
\delta c(\xi) &= 2 \pi \sum_{k \in \N_3} \langle A_k \dot{\xi} \cdot \xi \rangle \hat{e}_k + \mathcal{O}(||\xi||_{H^1_{\sharp}}^3),\\
\delta R(\xi) &= 2 \pi \sum_{k \in \N_3} \langle B_k \dot{\xi} \cdot \xi \rangle L_k + \mathcal{O}(||\xi||^4_{H_\sharp^1}).
\end{aligned}
\end{equation}
\end{proposition}

\begin{proof}
First, let us note that the term $\langle F_{c,0} \dot{\xi} \rangle$ vanishes due to the periodicity of the stroke $\xi$. Next, we observe that it suffices to prove that the scalar quantities of the form 
\begin{align}
\left \langle \left ( \int_{0}^{t} B_k \dot{\hat{\xi}}(\tau)  \cdot \hat{\xi}(\tau) \dd \tau \right ) A_l \dot{\hat{\xi}} \cdot \hat{\xi} \right  \rangle, & & k,l \in \N_3,
\end{align}
as well as $ \langle  ( \int_{0}^{t} B_k \dot{\hat{\xi}}(\tau)  \cdot \hat{\xi}(\tau) \dd \tau  )\dot{\hat{\xi}}_i  \rangle$, $i \in \N_4$ are bounded, where we again set $\xi = \varepsilon \hat{\xi}$. We focus on the terms of the latter form, since the others can be treated in the same manner. We have
\begin{equation}
\begin{aligned}
\left |\int_J \left ( \int_{0}^{t} B_k \dot{\hat{\xi}}(\tau) \cdot \hat{\xi}(\tau) \dd \tau\right ) \dot{\hat{\xi}}_i(t) \dd t \right | = \left | \int_{0}^{2\pi} B_k \dot{\hat{\xi}}(t) \cdot \hat{\xi}(t) \int_{t}^{2 \pi} \dot{\hat{\xi}}_i(s) \dd s \dd t \right |\\
\leq ||B_k||_{op} \int_{J} |\dot{\hat{\xi}}(t)| \cdot |\hat{\xi}(t) - \hat{\xi}(0)|^2 \dd t
\end{aligned}
\end{equation}
The Sobolev-Morrey embedding $H_\sharp^{1}(J, \R^4) \subseteq L^\infty_{\sharp}(J, \R^4)$ guarantees the existence of a $c_S > 0$ such that $||\xi||_{\infty} \leq c_S ||\xi||_{H_\sharp^1}$ for every $\xi \in H_\sharp^1(J, \R^4)$. Hence, we have
\begin{equation}
\begin{aligned}
\left | \left \langle \left  ( \int_{0}^{t} B_k \dot{\hat{\xi}}(\tau)  \cdot \hat{\xi}(\tau) \dd \tau  \right )\dot{\hat{\xi}}_i \right  \rangle\right |
& \leq ||B_k||_{op} ||\hat{\xi}||_{\infty}^2 ||||_{H_\sharp^1}\\
&\leq c_S ||B_k||_{op} ||\hat{\xi}||_{H_\sharp^1}^3 = c_S ||B_k||_{op},
\end{aligned}
\end{equation}
which is clearly bounded. This finishes the proof.
\end{proof}

To end this section, we note that on the one hand, we have $\langle A_k \dot{\xi} \cdot \xi \rangle = \langle M_k \dot{\xi} \cdot \xi \rangle$ and $\langle B_k \dot{\xi} \cdot \xi \rangle = \langle M_{k + 3} \dot{\xi} \cdot \xi \rangle$ for all $k \in \N_3$. Indeed, if $A$ is a symmetric matrix, we have by integration by parts that $\langle A \xi \cdot \dot{\xi} \rangle = \langle A \dot{\xi} \cdot \xi \rangle = - \langle A \xi \cdot \dot{\xi} \rangle $ and thus only the skew-symmetric parts of the matrices $A_k$ and $B_k$ contribute to the net displacement. Furthermore, similarly to \cite{Alouges2017}, we can represent the terms $ M_k \dot{\xi} \cdot \xi$ in terms of certain operations of the orthonormal basis $\{\tau_i\}_{i \in \N_4}$ of $\R^4$. In fact, we find by straightforward calculation using  that
\begin{align}
  M_k \dot{\xi} \cdot \xi &= - 2 \sqrt{6} \,\alpha \det( \xi | \dot{\xi} | \tau_{k+1} | \tau_{k+2}), & & k \in \N_3 \\
  M_{3 + k} \dot{\xi} \cdot \xi &= - 2 \sqrt{6} \,\delta \det ( \xi | \dot{\xi} | \tau_{k} | \tau_{4}), & & k \in \N_3,
\end{align}
where $\det(\xi|\dot{\xi}|\tau_j|\tau_k)$ denotes the determinant of the matrix $(\xi|\dot{\xi}|\tau_j |\tau_k)$ and the index $k$ is reduced mod 3 to simplify the notation. Ultimately, using that $\R^3 \times \so(3) \simeq \R^6$, we can write the net displacement in position and orientation simultaneously as
\begin{equation}
\label{eq: net displacement}
\frac{\delta p}{2 \pi}= - 2  \sqrt{6} \alpha \sum_{k \in \N_3}\langle \det( \xi | \dot{\xi} | \tau_{k+1} | \tau_{k+2}) \rangle f_k  - 2  \sqrt{6} \delta \sum_{k \in \N_3}\langle \det ( \xi | \dot{\xi} | \tau_{k} | \tau_{4})\rangle f_{k + 3},
\end{equation}
where $\{f_i\}_{i \in \N_6}$ denotes the canonical Basis of $\R^6$ and the index $k$ is once more reduced mod 3. This representation will prove particularly useful in the following section.