\label{sec: optimization}
In the spirit of \cite{Alouges2013} and \cite{Alouges2017}, we follow the notion of swimming efficiency suggested by Lighthill \cite{Lighthill1952} and we adopt the following notion of optimality: energy minimizing strokes are the ones that minimize the kinematic energy dissipated while trying to reach a given net displacement $\delta p \in \R^3 \times \so(3) \simeq \R^6$. Mathematically speaking, the total energy dissipation due to a stroke $\xi \in H^1_{\sharp}(J, \R^4)$ can be evaluated through an adequate quadratic energy functional, c.f. \cite{Alouges2013},
\begin{equation}
 \mathcal{G}_S(\xi) := \int_{J} \mathfrak{g}(\xi(t))\dot{\xi}(t) \cdot \dot{\xi}(t) \dd t,
\end{equation}
where the energy density $\mathfrak{g} \in C^1(\R^4)$ is a function with values in the space of symmetric and positive definite matrices $M_{4 \times 4}(\R)$. In other words, $\mathfrak{g}$ defines a continuous Riemannian metric on $\M$. In the small stroke regime, we can approximate the energy density by $\mathfrak{g}(\xi) = \mathfrak{g}(0) + o(1)$, where $\mathfrak{g}(0) \in M_{4 \times 4}(\R)$ is symmetric and positive definite. More precisely,
\begin{equation}
\label{eq: linearized energy functional}
	\mathcal{G}(\xi) := \int_{J} Q_{\mathfrak{g}}(\dot{\xi}(t))\dd t,
\end{equation}
with $Q_{\mathfrak{g}}(\eta) := \mathfrak{g}(0)\eta \cdot \eta$. For the same symmetry reasons as discussed in section \ref{sec: symmetries}, we necessarily have for all $\eta \in \R^4$
\begin{align}
	Q_{\mathfrak{g}}(P_{ij} \eta) = Q_{\mathfrak{g}}(\eta),\; i,j \in \N_4,
\end{align}
where $P_{ij}$ denotes the permutation matrix swapping the $i$-th and $j$-th entries. By direct computation, on finds that the symmetric positive matrix $G$ representing the quadratic form $Q_{\mathfrak{g}}$ is of the form
\begin{equation}
G = \left ( \begin{array}{cccc}
\kappa & h & h & h \\ 
h & \kappa & h & h \\ 
h & h & \kappa & h \\ 
h & h & h & \kappa
\end{array} \right ),
\end{equation}
for two parameters $h$ and $\kappa > \max(h, -3h)$. In particular, we observe that $G \tau_k = (\kappa - h ) \tau_k$ for $k \in  \N_3$ and $G \tau_4 = (\kappa + 3h) \tau_4$. In the following, we denote by $\mathfrak{g}_1 := \mathfrak{g}_2 := \mathfrak{g}_3 := \kappa - h$ and $\mathfrak{g}_4 := \kappa + 3h$ the eigenvalues of $G$. Furthermore, the eigenvalues $(\mathfrak{g}_i)_{i \in \N_4}$ allow us to diagonalize  $G$ as
\begin{eqnarray}
G = U \Lambda_{\mathfrak{g}} U^T, & U := [\tau_1 | \tau_2 |\tau_3 |\tau_4], & \Lambda_{\mathfrak{g}} := \diag(\mathfrak{g}_i).
\end{eqnarray}

The goal of this section is the minimization of $\mathcal{G}$ in $H_{\sharp}^1(J, \R^4)$ subject to a prescribed net displacement $\delta p \in \R^6$, i.e. subject to the constraint (c.f. (\ref{eq: net displacement}))
\begin{equation}
\label{eq: constraint}
\begin{aligned}
	 \delta p = \mathfrak{h}_{c} \sum_{k \in \N_3}\left ( \int_{J} \det( \xi(t) | \dot{\xi}(t) | \tau_{k+1} | \tau_{k+2}) \dd t \right ) f_k\\
	+ \mathfrak{h}_{\theta}  \sum_{k \in \N_3}\left ( \int_{J} \det ( \xi(t) | \dot{\xi}(t) | \tau_{k} | \tau_{4}) \dd t\right ) f_{k + 3},
\end{aligned}
\end{equation}
with $\mathfrak{h}_c = - 2 \sqrt{6} \alpha$ and $\mathfrak{h}_{\theta} = - 2 \sqrt{6} \delta$.


\subsection{Recall of the exterior algebra}
Let us recall in this small section the basic definitions around the notion of the \emph{exterior algebra}, where we refer to \cite{Lounesto2006} for details. For our purposes, the abstract definition of $k$-vectors as alternating $k$-tensors is not very useful. We merely illustrate them in $\R^3$ since the notion then generalizes easily to higher dimensions. In $\R^3$, a bivector is an oriented plane segment; that is, a small piece of surface having a magnitude given by the area of the surface element as well as a direction given by the attitude of the plane the surface element lies in as well as a sense of rotation. Together, they form the vector space $\bigwedge^2 \R^3$. We can represent a bivector $\omega \in \bigwedge^2 \R^3$ as a small parallelogram which suggests that we can think of it as some product of the two vectors along its sides. This is realized by the \emph{exterior product}, also called \emph{wedge product}, $u \wedge v$ of two vectors $u$ and $v$. The product $u \wedge v$ then represents the bivector obtained by sweeping $v$ along $u$. This operation yields a direct link between $\R^3$ and the vector space $\bigwedge^2 \R^3$ of bivectors, a basis of which is given by
\begin{equation}
 \{\hat{e}_1 \wedge \hat{e}_2, \hat{e}_1 \wedge \hat{e}_3, \hat{e}_2 \wedge \hat{e}_3\},
\end{equation}
if $\{\hat{e}_1, \hat{e}_2, \hat{e}_3\}$ is a basis of $\R^3$. In fact, the standard scalar product on $\R^3$ extends to a scalar product on $\bigwedge^2 \R^3$ by
\begin{equation}
\langle u_1 \wedge u_2 , v_1 \wedge v_2 \rangle = \det \left (\begin{array}{cc}
u_1 \cdot v_1 & u_1 \cdot v_2 \\ 
u_2 \cdot v_1 & u_2 \cdot v_2
\end{array}  \right).
\end{equation}
In particular, $\langle u \wedge v, u \wedge v \rangle = |u|^2 |v|^2 \sin^2\psi$, where $\psi$ is the angle between the vectors $u$ and $v$. Eventually, the norm of a bivector $\omega = \omega_{12} \hat{e}_1 \wedge \hat{e}_2 + \omega_{13} \hat{e}_1 \wedge \hat{e}_3 + \omega_{23} \hat{e}_2 \wedge \hat{e}_3$ is given by
\begin{equation}
|\omega| = \sqrt{\omega_{12}^2 + \omega_{13}^2 + \omega_{23}^2}.
\end{equation}
These definitions then extend naturally to all higher dimensions. In particular, we note that if $\{e_1, e_2, e_3, e_4\}$ again denotes the canonical basis of $\R^4$, a basis of the space $\bigwedge^2 \R^4$ is given by
\begin{equation}
\{e_{12}, e_{13}, e_{14}, e_{23}, e_{24}, e_{34}\},
\end{equation}
where we write $e_{ij} := e_i \wedge e_j$ to simplify the notation. More generally, the vectors, bivectors etc. of $\R^4$ all make part of a graded algebra, the \emph{exterior algebra}, defined by
\begin{equation}
\bigwedge \R^4 := \R \oplus \R^4 \oplus \bigwedge^2\R^4 \oplus \bigwedge^3 \R^4 \oplus \bigwedge^4 \R^4,
\end{equation}
whose product is given by the wedge product. As we shall see later, it is due to the fundamental differences between $\bigwedge^2 \R^3$ and $\bigwedge^2 \R^4$ that the structure of optimal control curves for \textsc{SPr3} and \spr are substantially different.


\subsection{G-Orthogonalization}
We begin by rewriting the energy functional (\ref{eq: linearized energy functional}) and the constraint (\ref{eq: constraint}) in terms of the orthonormal basis of eigenvectors $(\tau_i)_{i \in \N_4}$ of the matrix $G$. The change of variables $\eta(t) := U^T \xi(t) \in H_{\sharp}^{1}(J, \R^4)$, allows us to write
\begin{equation}
\label{eq: G-orth energy functional}
\mathcal{G}_{U}(\eta) = \int_{J} \Lambda_{\mathfrak{g}} \dot{\eta}(t) \cdot \dot{\eta}(t) \dd t,
\end{equation}
with $\mathcal{G}_{U}(\eta) := \mathcal{G}(\xi) = \mathcal{G}(U \eta)$. For the constraint, we note that
\begin{equation}
\det(\xi |\dot{\xi} | \tau_i | \tau_j) =  \det U \det (\eta | \dot{\eta} | e_i | e_j) = \det(\dot{\eta} | \eta | e_i |e_j),
\end{equation}
since $\det U = -1$. Eventually, we can express the determinants more elegantly in terms of exterior products. In fact, by direct calculation one obtains $\det(\dot{\eta} | \eta | e_k |e_4) = (\dot{\eta} \wedge \eta, e_{k + 1} \wedge e_{k+2})$ and $\det(\dot{\eta} |\eta | e_{k + 1} |e_{k + 2}) = (\dot{\eta} \wedge \eta, e_k \wedge e_4)$, for $k \in \N_3$ taken mod 3. Then, the isomorphism sending the basis $\{f_i\}_{i \in \N_6}$ of $\R^6$ onto the ordered basis 
\begin{equation}
\label{eq: basis of bivectors}
(e_{14}, e_{24}, e_{34}, e_{23}, e_{31}, e_{12})
\end{equation}
of $\bigwedge^2 \R^4$, where we write $e_{ij} := e_i \wedge e_j$, allows us to rewrite (\ref{eq: constraint}) as
\begin{equation}
\label{eq: G-orth constraint}
\Lambda_{\mathfrak{h}}^{-1} \delta p = \int_{J} \dot{\eta}(t) \wedge\eta(t) \dd t,
\end{equation}
with $\Lambda_{\mathfrak{h}} := \diag(\mathfrak{h}_{c}, \mathfrak{h}_{c}, \mathfrak{h}_{c}, \mathfrak{h}_{\theta}, \mathfrak{h}_{\theta}, \mathfrak{h}_{\theta})$.

\subsection{Fourier transformation of the minimization problem}
We denote by $\ell^2(\R^4)$ the space of sequences $\mathbf{u} := (u_n)_{n \in \N}$ in $\R^4$ such that the norm
\begin{equation}
	||\mathbf{u}||_{\ell^2(\R^4)} := \sqrt{ \sum_{n \in \N} |u_n|^2 }
\end{equation}
is finite. Consequently, we denote by $\dot{\ell}^2(\R^4)$ the Hilbert space of sequences $\mathbf{u} = (u_n)_{n \in \N} \in \ell^2(\R^4)$ such that $(n u_n)_{n \in \N} \in \ell^2(\R^4)$. As the elements in $H_{\sharp}^{1}(J, \R^4)$ are $2\pi$-periodic, we can express $\eta$ in terms of its Fourier series as
\begin{equation}
\eta(t) := \frac{1}{2} a_0 + \sum_{n \in \N} \cos(nt) a_n + \sin(n t) b_n,
\end{equation}
with $(a_n, b_n) \in \dot{\ell}^2(\R^4) \times \dot{\ell}^2(\R^4)$. Substitution of the Fourier series of $\dot{\eta}$ into the energy functional (\ref{eq: G-orth energy functional}) yields due to Parseval's equality
\begin{align}
\mathcal{G}_{U} (\eta) := \int_{J} \Lambda_{\mathfrak{g}} \dot{\eta}(t) \cdot \dot{\eta} dt &= \pi \sum_{n  \in \N} n^2(\Lambda_{\mathfrak{g}} a_n \cdot a_n + \Lambda_{\mathfrak{g}} b_n \cdot b_n)  \\  &=
\frac{1}{2} ||\mathbf{u}||_{\ell^2(\R^4)}^2 + \frac{1}{2} ||\mathbf{v}||_{\ell^2(\R^4)}^2,
\end{align}
where we have set
\begin{align}
\label{eq:relation Fourier coeffs of eta}
	\mathbf{u} := (u_n)_{n \in \N} := \sqrt{2 \pi \Lambda_{\mathfrak{g}}}(n a_n)_{n \in \N} \text{ and } \mathbf{v} := (v_n)_{n \in \N} := \sqrt{2 \pi \Lambda_{\mathfrak{g}}} (n b_n)_{n \in \N}.
\end{align}
Clearly, we have $(\mathbf{u}, \mathbf{v}) \in \ell^2(\R^4) \times \ell^2(\R^4)$. As a result of the $L_{\sharp}^2(J, \R^4)$-orthogonality of the Fourier trigonometric system, we can express the constraint (\ref{eq: G-orth constraint}) in terms of Fourier coefficients as $2 \pi \sum_{n \in \N} \tfrac{1}{n} (nb_n) \wedge (n a_n) = \Lambda_{\mathfrak{h}}^{-1} \delta p$. Returning to our old notation for a moment, we note that for any $n \in \N$, we have
\begin{equation}
2 \pi n^2 \det(b_n | a_n | e_i | e_j) = \frac{\sqrt{\mathfrak{g}_i \mathfrak{g}_j}}{\sqrt{\det \Lambda_{\mathfrak{g}}}} \det(v_n | u_n |e_i | e_j),
\end{equation}
and hence by setting $\tilde{\Lambda}_{\mathfrak{g}} := \diag(\mathfrak{g}_c, \mathfrak{g}_c, \mathfrak{g}_c, \sqrt{\mathfrak{g}_c \mathfrak{g}_{\theta}}, \sqrt{\mathfrak{g}_c \mathfrak{g}_{\theta}}, \sqrt{\mathfrak{g}_c  \mathfrak{g}_{\theta}})$, with $\mathfrak{g}_1 :=\mathfrak{g}_2 := \mathfrak{g}_3 := \mathfrak{g}_c$ and $\mathfrak{g}_4 := \mathfrak{g}_\theta$, we eventually find
\begin{equation}
\sqrt{\det \Lambda_{\mathfrak{g}}} (\Lambda_{\mathfrak{h}} \tilde{\Lambda}_{\mathfrak{g}})^{-1} \delta p = \sum_{n \in \N} \frac{v_n  \wedge u_n}{n}
\end{equation}
We thus have proved the following

\begin{proposition}
\label{prop: l2-minimization}
The $H_{\sharp}^1(J, \R^4)$ minimization of the functional $\mathcal{G}_U$ given by (\ref{eq: G-orth energy functional}) under the constraint (\ref{eq: G-orth constraint}) is equivalent to the minimization of the functional
\begin{equation}
\label{eq:l2-energy}
	\mathcal{F}(\mathbf{u}, \mathbf{v}) := \frac{1}{2} ||\mathbf{u} ||^2_{\ell^2(\R^4)} + \frac{1}{2} ||\mathbf{v}||^2_{\ell^2(\R^4)},
\end{equation}
defined in the product Hilbert space $\ell^2(\R^4) \times \ell^2(\R^4)$ and subject to the constraint
\begin{equation}
\label{eq:l2-constraint}
\sum_{n \in \N} \frac{1}{n} v_n \wedge u_n = \omega \text{ with } \omega := \sqrt{\det \Lambda_{\mathfrak{g}}}(\Lambda_{\mathfrak{h}} \tilde{\Lambda}_{\mathfrak{g}})^{-1} \delta p,
\end{equation}
where $\delta p \in \R^6$ is a prescribed net displacement of position and orientation.
\end{proposition}

\subsection{Bivectors in the fourth dimension}
In this section we will point out some properties of the space $\bigwedge^2 \R^4$ and how it differs from $\bigwedge^2 \R^3$, which then illustrates why the optimal control curves for $\textsc{SPr3}$ and \textsc{SPr4} in general do not have the same structure. Nevertheless, we will be able treat a simplified situation for \textsc{SPr4} in the same manner as \textsc{SPr3} in \cite{Alouges2017}.

As a matter of fact, one peculiarity of the bivectors in $\R^3$ is that they are isomorphic to the space $\R^3$ itself. This is realized by the so-called \emph{Hodge dual operator} $\star$, c.f. \cite{Lounesto2006} p. 38, which is defined in a way such that for any two vectors $u,v \in \R^3$ one has
\begin{equation}
\label{eq: hodge star}
u \wedge v = \star(u \times v),
\end{equation}
where $\times$ denotes the usual cross product. This entails two things: First, every bivector in $\R^3$ is \emph{simple}; that is, it can be written as the wegde product of two vectors. Second, every bivector in $\R^3$ defines one unique plane in $\R^3$. It is due to this underlying geometrical fact that one can always reduce the Fourier coefficients of an optimal control curve to one pair of vectors, by which one attained the same net displacement at the same energy consumption in \cite{Alouges2017}. Furthermore, this implies that the optimal control curves are situated in a certain plane defined by the vector of the net displacement.

Yet, in $\R^4$, the geometry of its bivectors proves to be more involved. As $\dim \bigwedge^2 \R^4 = 6$, it is clear that the bivectors in $\R^4$ are not isomorphic to $\R^4$ itself. In particular, in $\R^4$ not all bivectors are simple. Indeed, the bivector $e_1 \wedge e_2 + e_3 \wedge e_4 \in \bigwedge^2 \R^4$ cannot be written as an exterior product of just two vectors in $\R^4$. Nevertheless, any bivector in $\R^4$ can be written as the sum of two orthogonal simple bivectors \cite{Lounesto2006}. Moreover, we have the following criterion to determine whether a bivector is simple:

\begin{lemma}
\label{lem:simple bivector}
A bivector $\omega \in \bigwedge^2 \R^4$ is simple if and only if $\omega \wedge \omega = 0$.
\end{lemma}

\begin{proof}
If $\omega \in \bigwedge^2 \R^4$ is a simple bivector, i.e. if there are vectors $u,v \in \R^4$ such that $\omega = u \wedge v$, then it is clear from the anticommutativity and associativity of the wedge product that
\begin{equation}
\omega \wedge \omega = (u \wedge v) \wedge (u \wedge v) = - u \wedge u \wedge v \wedge v = 0.
\end{equation}
The inverse requires a rather lengthy proof by induction, see the lecture notes on projective geometry by Nigel Hitchin, chapter 3, p.48 \cite{Hitchin2003}.
\end{proof}


We note that the condition in Lemma \ref{lem:simple bivector} is in particular satisfied if all coefficients corresponding to a certain index are zero, e.g. $\omega_{i4} = 0$ for $i \in \N_4$. This yields four subspaces $D^*_{ijk}$ of $\bigwedge^2 \R^4$ consisting only of simple bivectors. Since net displacement $\delta p$ is merely scaled and multiplied by diagonal matrices to get the constraint vector in (\ref{eq:l2-constraint}), its structure is not changed and thus by inspection of the basis of $\bigwedge^2 \R^4$, we have the following correspondences:
\begin{align*}
D^{*}_{123} &\longleftrightarrow\text{ rotations around all three axes } \hat{e}_1, \hat{e}_2, \hat{e}_3\\
D^{*}_{124} &\longleftrightarrow \text{ translation in the $\hat{e}_1\hat{e}_2$-plane, rotation around the $\hat{e}_3$-axis }\\
D^{*}_{134} &\longleftrightarrow  \text{ translation in the $\hat{e}_1\hat{e}_3$-plane, rotation around the $\hat{e}_2$-axis }\\
D^{*}_{234} &\longleftrightarrow  \text{ translation in the $\hat{e}_2\hat{e}_3$-plane, rotation around the $\hat{e}_1$-axis }
\end{align*}
In particular, the subspaces $D^{*}_{ijk}$ are of dimension three and correspond to the subspaces $D_{ijk} := \Span\{e_i, e_j, e_k\} \subset \R^4$. In fact, we have $D^{*}_{ijk} = \bigwedge^2 D_{ijk}$ and furthermore the isomorphism given by the Hodge star operator $\star$, i.e.
\begin{equation}
\omega = \omega_{ij} e_{ij} + \omega_{ik} e_{ik} + \omega_{jk} e_{jk} \mapsto \omega_{ij} e_k + \omega_{ik} e_j + \omega_{jk} e_i.
\end{equation}
This especially allows us to relate cross products in the subspace $D_{ijk}$ to a unique bivector in $D_{ijk}^*$. More precisely, we have for $u, v \in D_{ijk}$ that $u \wedge v = \star(u \times v)$ just like in (\ref{eq: hodge star}), where the usual cross product is understood in the subspace $D_{ijk}$.

\subsection[The simple case]{The simple case $\omega \in D_{ijk}^*$}
With the remarks from the preceding section, we are able to solve the constrained minimization problem of Proposition \ref{prop: l2-minimization} in a similar manner to \cite{Alouges2017}. In fact, we retrieve essentially the same result, i.e. that the optimal control curves are ellipses in a certain plane defined by the net displacement. Let us prove

\begin{proposition}
\label{prop:simple reduction}
If $\omega = D_{ijk}$, then for any $(\mathbf{u}, \mathbf{v}) \in \ell^2(\R^4) \times \ell^2(\R^4)$ such that the constraint (\ref{eq:l2-constraint}) holds, there exist two vectors $u, v \in \R^4$ such that for the sequences $\mathbf{u_\star} := \mathbf{e}_1 u$ and $\mathbf{v_\star} := \mathbf{e}_1 v \in \ell^2(\R^4)$ one has
\begin{equation}
\mathcal{F}(\mathbf{u_{\star}}, \mathbf{v_{\star}}) = \mathcal{F}(\mathbf{u}, \mathbf{v}) \text{ and } v \wedge u = \omega.
\end{equation}
\end{proposition}

\begin{proof}
If $\omega = 0$, then the proof is trivial. Thus, let us denote by $\hat{\omega}$ the unit bivector associated to $\omega$. For a couple $(\mathbf{u}, \mathbf{v}) \in \ell^2(\R^4) \times \ell^2(\R^4)$, we then choose $u,v \in \R^4$ such that the following relations hold:
\begin{eqnarray}
\label{eq:reduction int1}
	|u| = ||\mathbf{u} ||_{\ell^2(\R^4)}, &  |v| = ||\mathbf{v} ||_{\ell^2(\R^4)} , & \frac{u \wedge v}{|u \wedge v|} = \hat{\omega}.
\end{eqnarray}
The latter is possible since $\hat{\omega}$ is a simple bivector by hypothesis. Furthermore, we have $u \wedge v = ||\mathbf{u}||_{\ell^2(\R^4)} ||\mathbf{v}||_{\ell^2(\R^4)} (\sin\psi)\hat{\omega}$, where $\psi$ is the angle between $u$ and $v$. Therefore, the equality $u \wedge v = \omega$ can be satisfied by choosing the angle $\psi \in (0, \pi)$ such that
\begin{equation}
 \sin \psi = \frac{|\omega|}{||\mathbf{u}||_{\ell^2(\R^4)} ||\mathbf{v} ||_{\ell^2(\R^4)}}
 \end{equation}
This is possible under the condition that the right hand side  of the previous equation is not greater than one. In fact, we have using the Cauchy-Schwarz inequality
\begin{equation}
|\omega| \leq \sum_{n \in \N} \frac{1}{n} |v_n \wedge u_n| \leq \sum_{n \in \N} |v_n| |u_n| \leq ||\mathbf{u} ||_{\ell^2(\R^4)} ||\mathbf{v} ||_{\ell^2(\R^4)}.
\end{equation}
Finally, from (\ref{eq:reduction int1}) we obtain
\begin{equation}
\mathcal{F}(\mathbf{u_\star}, \mathbf{v_{\star}}) = \frac{1}{2} |u|^2 + \frac{1}{2} |v|^2 = \frac{1}{2} ||\mathbf{u}||_{\ell^2(\R^4)}^2 + \frac{1}{2} ||\textbf{v}||_{\ell^2(\R^4)}^2,
\end{equation}
which concludes the proof.
\end{proof}

We immediately have

\begin{corollary}
\label{cor:simple reduction}
The minimization problem for $\mathcal{F}$ in $\ell^2(\R^4) \times \ell^2(\R^4)$, under the constraint (\ref{eq:l2-constraint}), is equivalent to the minimization in $\R^4 \times \R^4$ of the function
\begin{equation}
\label{eq:finite dim energy}
 	f(u,v) := \frac{1}{2}|u|^2 + \frac{1}{2} |v|^2
 \end{equation} 
 under the constraint
 \begin{equation}
 \label{eq:finite dim constraint}
 v \wedge u = \omega.
 \end{equation}
\end{corollary}

\begin{proof}
It suffices to observe that if $\mathcal{V}_{\omega}$ denotes the subset of $\ell^2(\R^4) \times \ell^2(\R^4)$ satisfying the constraint (\ref{eq:l2-constraint}) and by $V_{\omega}$ the subset of $(u, v)  \in \R^4 \times \R^4$ such that $u \times v = \star\omega$, then Proposition \ref{prop:simple reduction} yields
\begin{equation}
\min_{(\mathbf{u}, \mathbf{v}) \in \mathcal{V}_{\omega}}\mathcal{F}(\mathbf{u}, \mathbf{v}) = \min_{(u, v) \in V_\omega} \mathcal{F}(\mathbf{e}_1 u, \mathbf{e}_1 v) = \min_{(u,v) \in V_{\omega}} f(u,v).
\end{equation}
\end{proof}

Let us now prove the following


\begin{proposition}
\label{prop:finite dim minimization}
Any couple of vectors $(u_{\star}, v_{\star}) \in \R^4 \times \R^4$ minimizing the function $f$ given in (\ref{eq:finite dim energy}) and subject to the constraint (\ref{eq:finite dim constraint}) with $\omega \in D_{ijk}^{*}$, is characterized by the following conditions:
\begin{eqnarray}
\label{eq:finite dim minimization conditions}
u_{\star}, v_{\star} \in D_{ijk}, & 
|u_{\star}|^2 = |v_{\star}|^2, & 
u_{\star} \cdot v_{\star} = 0.
\end{eqnarray}
Therefore, for any $\sigma \in D_{ijk}$ such that $\sigma \cdot \star\hat{\omega} = 0$ and $|\sigma|^2 = |\omega|$, the couple
\begin{equation}
(\sigma, \sigma \times \star \hat{\omega}) \in D_{ijk} \times D_{ijk} \subset \R^4 \times \R^4
\end{equation}
is a (global) constrained minimizer for $f$.
\end{proposition}

\begin{remark}
The construction of a couple of vectors satisfying relations (\ref{eq:finite dim minimization conditions}) is straightforward and very similar to the construction of the corresponding pair of vectors for \textsc{SPr3}, c.f. Remark 18 in \cite{Alouges2017}. It suffices to choose any vector $\mu \in D_{ijk}$ linearly independent from $\star \hat{\omega}$ and to set $\sigma := \mu \times \star\hat{\omega}$. In fact, setting $\hat{\mu} := \mu / |\mu|$ yields $|\mu|^2 = |\omega|/(1 - (\hat{\mu} \cdot \star\hat{\omega})^2)$ and therefore by choosing $|\mu|^2 = |\omega|/(1 - (\hat{\mu} \cdot \star \hat{\omega})^2)$ we get $|\sigma|^2 = |\omega|$.
\end{remark}

\begin{proof}
First, we note that it is clear that $u_{\star}, v_{\star} \in D_{ijk}$ by definition of $D_{ijk}^{*}$.
Next, to find the minimizers of the problem (\ref{eq:finite dim energy}) - (\ref{eq:finite dim constraint}), we note that the constraint $u \wedge v = \omega$ implies the existence of a $\psi \in (0, \pi)$ such that $|u||v| \sin \psi= |\omega|$. Hence, the constrained minimization for $f$ is equivalent to the unconstrained minimization of the function $\hat{f}: \R^4 \times (0, \pi) \to \R$ defined by
\begin{equation}
(u, \psi) \mapsto \frac{1}{2} |u|^2 + \frac{1}{2} \frac{|\omega|^2}{|u|^2 \sin^2\psi},
\end{equation}
whose stationary satisfy $\psi_{\star} = \frac{\pi}{2}$ and $|u_{\star}|^2 = |\omega|$. This shows the necessity of the conditions stated in (\ref{eq:finite dim minimization conditions}). To show sufficiency of the condition, one observes that for any such points one has $\hat{f}(u_{\star}, \psi_{\star}) = |\omega|$. Indeed, for any $(u, \psi) \in D_{ijk, \psi} \in \R^4 \times (0, \pi)$ we have
\begin{equation}
\hat{f}(u, \psi) \geq \frac{1}{2} \frac{|u|^4 + |\omega|^2}{|u|^2} = |\omega| + \frac{1}{2}\frac{(|\omega| - |u|^2)^2}{|u|^2} \geq |\omega| = \hat{f}(u_{\star}, \psi_{\star}).
\end{equation}
This proves the first statement. For the second part, we set $\hat{\omega} := \omega/|\omega|$ and we consider a vector $\sigma \in D_{ijk}$ such that
\begin{equation}
	\sigma \cdot \star\hat{\omega} = 0 \text{ and } |\sigma|^2 = |\omega|. 
\end{equation}
By a trivial computation, one finds that the vectors $v := \sigma \times \star \hat{\omega}$ and $u := \sigma$ satisfy the relations $u \cdot v = 0$, $|u|^2 = |v|^2 = |\omega|$ and $v \wedge u = \omega$. 
\end{proof}

Pasting everything worked out above together leads to the final result of this section. We have


\begin{theorem}
\label{thm:optimal control curves in the simple case}
Let $\delta p \in \R^3 \times \so(3) \simeq \bigwedge^2 \R^4$ be a prescribed net displacement. Moreover, assume that $\delta p \in D_{ijk}^*$ for some combination of indices $i,j,k \in \N_4$. Then, any minimizer $\xi \in H_{\sharp}^1(J, \R^4)$ of the energy functional (\ref{eq: linearized energy functional}) subject to the constraint (\ref{eq: constraint}) is of the form
\begin{equation}
\xi(t) := (\cos t) a + (\sin t) b,
\end{equation}
i.e. an ellipse of $D_{ijk} \subset \R^4$ centered at the origin and contained in the plane spanned by the vectors $a$ and $b$. The vectors $a,b \in D_{ijk}$ are obtained as follows:
\begin{enumerate}
\item We compute the vector $\omega$ via the relation
\begin{equation}
\omega := \diag \left (\frac{\sqrt{\mathfrak{g}_c \mathfrak{g}_{\theta}}}{\mathfrak{h}_c}, \frac{\sqrt{\mathfrak{g}_c \mathfrak{g}_{\theta}}}{\mathfrak{h}_c}, \frac{\sqrt{\mathfrak{g}_c \mathfrak{g}_{\theta}}}{\mathfrak{h}_c}, \frac{\mathfrak{g}_c}{\mathfrak{g}_\theta}, \frac{\mathfrak{g}_c}{\mathfrak{g}_\theta}, \frac{\mathfrak{g}_c}{\mathfrak{g}_\theta} \right ) \delta p \in D_{ijk}.
\end{equation}
Then we consider a vector $u \in D_{ijk}$ in the plane orthogonal to $\star \omega$ in $D_{ijk}$ and such that
\begin{equation}
\label{eq:global minimizer condition}
|u|^2 = |\omega|,
\end{equation}
e.g. $u := \sqrt{|\omega|} \frac{\mu \times \star \omega}{|\mu \times \star \omega|}$, with $\mu \in D_{ijk}$ linearly independent from $\star \omega$. Furthermore, we set $v := u \times \star \omega$.

\item We set $\hat{\omega} := \omega/|\omega|$ and we calculate the vectors $a$ and $b$ via the relations
\begin{equation}
\label{eq:global minimizer form}
\begin{aligned}
a := \frac{U \Lambda_{\mathfrak{g}}^{-1/2}}{\sqrt{2 \pi}} u,&& b := \frac{U \Lambda_{\mathfrak{g}}^{-1/2}}{\sqrt{2 \pi}} v.
\end{aligned}
\end{equation}
\end{enumerate}
We then have $ v \wedge u = \omega$ and the minimum value of $\mathcal{G}$ is equal to $|\omega|$.

In addition, the vectors $a$ and $b$ are $\mathfrak{g}$-orthogonal, i.e. with respect to the inner product defined for every $x,y \in \R^4$ by $(x, y)_{\mathfrak{g}} := 2 \pi \Lambda_{\mathfrak{g}} x \cdot y$, and have the same $\mathfrak{g}$-norm $|a|_{\mathfrak{g}}^2 = |b|_{\mathfrak{g}}^2 = |\omega|$. 
\end{theorem}

\begin{remark}
Suppose that $\delta p \mid \mid e_{14}$, i.e. a net displacement along the $\hat{e}_1$-axis. Then we have both $\hat{\omega} \in D^{*}_{124}$ as well as $\hat{\omega} \in D_{134}^{*}$. In the first case, we have $\star \hat{\omega} = e_2$ and thus we can choose $u := \sqrt{|\omega|} e_4$ to satisfy (\ref{eq:global minimizer condition}). Then we have $v = -\sqrt{|\omega|} e_1$ and hence we get from (\ref{eq:global minimizer form}) that
\begin{eqnarray}
	a := \sqrt{\frac{|\omega|}{2 \pi \mathfrak{g}_4}} \tau_{4}&, & b := -\sqrt{\frac{|\omega|}{2 \pi \mathfrak{g}_{1}}} \tau_1.
\end{eqnarray}
In the case $\hat{\omega} \in D^{*}_{134}$ we have $\star\hat{\omega} = e_3$ and therefore we find the same vectors $a$ and $b$ from the relations (\ref{eq:global minimizer condition}) and (\ref{eq:global minimizer form}). Indeed, the resulting vectors are independent of the choice of one of the two possible spaces $D^*_{ijk}$ for any direction $\hat{\omega} = e_{ij}$. In other words, an energy minimizing net displacement along the $\hat{e}_1$-axis with respect to the standard euclidean space $(\R^4, (\cdot, \cdot)_2)$ is realized by an elliptic stroke contained in the plane spanned by the vectors $\tau_1$ and $\tau_4$. On the other hand, with respect to the inner-product space $(\R^4, (\cdot, \cdot)_{\mathfrak{g}})$, the energy optimizing strokes are circles of radius $\sqrt{|\omega|}$. More generally, if $\hat{\omega} = e_{ij}$, then the energy optimizing strokes describe ellipses with respect to the standard inner-product or $(\cdot, \cdot)_2$ circles of radius $\sqrt{|\omega|}$ with respect to the inner-product $(\cdot, \cdot)_{\mathfrak{g}}$ in the plane spanned by the vectors $\tau_i$ and $\tau_j$.
\end{remark}

\begin{proof}
From Proposition \ref{prop:finite dim minimization}, Corollary \ref{cor:simple reduction} and then Proposition \ref{prop: l2-minimization}, we get that any $\sigma \in D_{ijk}$ satisfying the relations
\begin{eqnarray}
 u \cdot \star \hat{\omega} = 0, & |u|^2 = |\omega|, & \omega := \sqrt{\det \Lambda_{\mathfrak{g}}} (\Lambda_{\mathfrak{h}} \tilde{\Lambda}_{\mathfrak{g}})^{-1} \delta p,
\end{eqnarray}
is associated to a (global) constrained minimizer for $\mathcal{G}_{U}$, via the curve $\eta(t) := (\cos t) \tilde{a} + (\sin t) \tilde{b}$, where the Fourier coefficients $\tilde{a}, \tilde{b} \in \R^4$ are related to $\omega$ (c.f. \ref{eq:relation Fourier coeffs of eta}) by $(\sqrt{2 \pi \Lambda_{\mathfrak{g}}}) \tilde{a} = u$ and $(\sqrt{2 \pi \Lambda_{\mathfrak{g}}}) \tilde{b} = (u \times \star\hat{\omega})$. The minimum value of the energy is then $\mathcal{G}_{U}(\eta) = |\omega|$.

Finally, in the $\mathfrak{g}$-orthogonal reference frame, the inner product is defined by $(x, y)_{\mathfrak{g}} := 2 \pi \Lambda_{\mathfrak{g}}x \cdot y$ for $x,y \in \R^4$. Let us denote by $|\cdot|_{\mathfrak{g}}$ the associated norm. Then we have the following relations:
\begin{align}
|\tilde{a}|_{\mathfrak{g}}^2 = |\tilde{b}|_{\mathfrak{g}}^{2} = |\omega| \;  \text{ and } \; (\tilde{a}, \tilde{b})_{\mathfrak{g}} = 0.
\end{align}
Applying the orthogonal map $U$ to $\tilde{a}$ and $\tilde{b}$ finishes the proof.
\end{proof}



