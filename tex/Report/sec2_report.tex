\label{sec:modelling}
We restrict ourselves to considering the swimmer \spr proposed in \cite{Alouges2013}. Let $(S_1, S_2, S_3, S_4)$ be a regular reference tetrahedron centered at $c \in \R^3$ such that $\dist(c, S_i) = 1$ for all $i \in \N_4$. Then the swimmer consists of four balls $(B_i)_{i \in \N_4}$ of $\R^3$ centered at $b_i \in \R^3$, all of radius $a > 0$, such that the ball $B_i$ can move along the ray starting at $c$ and passing through $S_i$. This reflects the situation where the balls are linked together by think jacks that are able to elongate. However, the viscous resistance of these jacks is neglected and therefore the fluid is assumed to permeate the entire open set $\R^3 \setminus \bigcup_{i = 1}^{4} \overline{B}_i$. The balls do not rotate around their arms which implies that the shape of the swimmer is completely determined by the four lengths $\xi_1, \xi_2, \xi_3, \xi_4$ of its arms, measured from the $c$ to the center of each ball $b_i$. However, there are no restrictions for the rotation of the swimmer around the center $c$, i.e. for fixed arm lengths, the swimmer is considered to be a rigid body in a Stokesian fluid.
Hence, the geometrical configuration of the swimmer can be described by two sets of variables:
\begin{enumerate}
	\item The vector of \emph{shape variables} $\xi := (\xi_1, \xi_2, \xi_3, \xi_4) \in \M := (\sqrt{\tfrac{3}{2}}a, +\infty)^4 \subseteq \R_+^4$, where the lower bound in the open intervals is chosen such that the balls cannot overlap.
	\item The vector of \emph{position variables} $p = (c, R) \in \mathcal{P} :=  \R^3 \times \SO(3)$.
\end{enumerate}
To be more precise, we consider the reference tetrahedron convexly spanned by the four unit vectors $z_1 := (2 \sqrt{2}/3,0,-1/3)$, $z_2 := (-\sqrt{2}/3,-\sqrt{2/3},-1/3)$, $z_3 := (-\sqrt{2}/3,\sqrt{2/3},-1/3)$ and $z_4 := (0,0,1)$. Position and orientation in $\R^3$ are then described by the coordinates of the center $c \in \R^3$ and the rotation $R \in SO(3)$ of the swimmer with respect to the reference orientation induced by the reference tetrahedron. Thus, we set $b_i := c + \xi_i R z_i$ for the center of the ball $B_i$.

The swimmer is completely described by the parameters $(\xi, p) \in \M \times \mathcal{P}$. Indeed, if we denote by $B_a$ the ball in $\R^3$ of radius $a$ centered at the origin, then for any $r \in \partial B_a$, the position of the current point on the $i$-th sphere of the swimmer in the state $(\xi, p)$ is given, for any $(\xi, p, r) \in \M \times \mathcal{P} \times \partial B_a$, by the function
\begin{equation}
	r_i(\xi, p, r) :=  c + R(\xi_i z_i + r).
\end{equation}
Note that the functions $(r_i)_{i \in \N_4}$ are analytic in $\M  \times \mathcal{P}$ and thus we can use them to calculate the instantaneous velocity on the $i$-th sphere $B_i$, which for any $(\xi, p, r) \in \M \times \mathcal{P} \times \partial B_a$ and every $i \in \N_4$ is given by
\begin{equation}
	u_i(\xi, p, r) = \dot{c} + \omega \times (\xi z_i + r) + R z_i \dot{\xi}_i,
\end{equation}
where $\omega$ is the axial vector associated with the skew matrix $\dot{R} R$.

In \cite{Alouges2013} it is shown that the system \spr, i.e. both the shape $\xi$ and the position $p$, is controllable only using the rate of change $\dot{\xi}$ of the shape. To do so, we have to understand how $p$ changes when we vary $\dot{\xi}$. To that and, the assumptions of \emph{self-propulsion} and negligible inertia of the swimmer are made. They imply that the total viscous force and torque exerted by the surrounding fluid on the swimmer must vanish. More precisely, the system can be written as
\begin{equation}
\label{eq: control system}
	\dot{p} = F(R, \xi) \dot{\xi} := \left ( \begin{array}{c}
	F_c(R, \xi) \\ 
	\hline
	F_\theta(R, \xi)
	\end{array}  \right ) \dot{\xi},
\end{equation}
where $\dot{c} = F_c(R, \xi) \dot{\xi}$ and $\dot{R} = F_\theta (R, \xi) \dot{\xi} $. 

In preparation for what follows, let us note that we have $F(R, \xi) \in \mathcal{L}(\R^4, T_{p}\mathcal{P})$ for any $R \in \SO(3)$ and $\xi \in \R^4$, where $\mathcal{L}(V, W)$ denotes the linear maps between two vector spaces $V$ and $W$. We quickly recall the fact that at any point $R \in \SO(3)$ we have 
\begin{equation}
	T_R \SO(3) = R^* \Skew_3(\R) = \{R M \mid M \in \Skew_3(\R)\},
\end{equation}
where $\Skew_n(\R)$ denotes the set of skew-symmetric real matrices of size $n \times n$. Hence, we have in particular that for any $R \in \SO(3)$ and $\xi \in \R^4$
\begin{equation}
\begin{aligned}
	F_c(R, \xi) \in \mathcal{L}(\R^4, \R^3) \text{ and } F_{\theta}(R, \xi) \in \mathcal{L}(\R^4, R^* \Skew_3(\R))
\end{aligned}
\end{equation}
and therefore we can express both $F_c(R, \xi)$ and $F_{\theta}(R, \xi)$ as real matrices of size $3 \times 4$ once we have chosen a basis for the corresponding tangent spaces.


 In analogy to \cite{Alouges2017}, it is important to note here that the control system $F$ is independent of $c$ due to the translational invariance of the Stokes equations. However, the translational invariance is not the only symmetry that \spr satisfies. The goal of the following section is to examine the structure of the control system $F$ in consequence of the symmetries it must satisfy being driven by the Stokes equations.





























