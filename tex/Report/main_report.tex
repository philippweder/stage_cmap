\documentclass[10pt,a4paper]{article}
\usepackage[utf8]{inputenc}
\usepackage[english]{babel}
\usepackage{amsmath}
\usepackage{amsthm}
\usepackage{amsfonts}
\usepackage{amssymb}
\usepackage{graphicx}
\usepackage[font=small]{caption}
\usepackage{wrapfig}
\usepackage{subfig}
\usepackage[a4paper,width=140mm,top=25mm,bottom=25mm]{geometry}
\author{Philipp Weder}
\date{}





% packages for layout
\usepackage{fancyhdr}
\pagestyle{fancy}
\fancyhf{}
\fancyhead[L]{\textit{\nouppercase{\leftmark}}}
\fancyhead[R]{\thepage}


%section titles
\usepackage{titlesec}


\titleformat{\section}
  {\centering\Large\scshape}{\thesection. }{1em}{}
  
\titleformat{\subsection}
  {\centering\large\scshape}{\thesubsection. }{1em}{}
  
\titleformat{\subsubsection}
  {\centering\scshape}{\thesubsubsection. }{1em}{}


%roman enumeration
\renewcommand\labelenumi{(\roman{enumi})}
\renewcommand\theenumi\labelenumi
\numberwithin{equation}{section}

% font
%\usepackage{pxfonts}

% bibliography
\usepackage[style=ieee, sorting = nty, backend = biber]{biblatex}
\bibliography{report.bib}
\usepackage{csquotes}

% environments
% theorem
\theoremstyle{plain}
\newtheorem{theorem}{Theorem}

% corollary
\theoremstyle{plain}
\newtheorem{corollary}[theorem]{Corollary}

% lemma
\theoremstyle{plain}
\newtheorem{lemma}[theorem]{Lemma}

% remark
\theoremstyle{remark}
\newtheorem*{remark}{Remark}
% definition
\theoremstyle{definition}
\newtheorem{definition}[theorem]{Definition}
% example
\theoremstyle{definition}
\newtheorem{example}{Example}
% proposition
\theoremstyle{plain}
\newtheorem{proposition}[theorem]{Proposition}


\theoremstyle{plain}
\newtheorem{condition}[theorem]{Condition}


% additional packages
\usepackage{appendix}
\usepackage{amsmath}
\usepackage{amsfonts}
\usepackage{amssymb}
\usepackage{amsthm}
\usepackage{booktabs}
\usepackage{hyperref}

\newcommand{\N}{\mathbb{N}}
\newcommand{\M}{\mathcal{M}}
\newcommand{\R}{\mathbb{R}}
\newcommand{\h}{\mathcal{H}}
\newcommand{\K}{\mathcal{K}}
\DeclareMathOperator{\Skew}{Skew}
\DeclareMathOperator{\id}{id}
\newcommand{\so}{\mathfrak{so}}
\newcommand{\REF}{\mathrm{ref}}
\newcommand{\spr}{\textsc{SPr4}}
\DeclareMathOperator{\dist}{dist}
\DeclareMathOperator{\SO}{SO}
\DeclareMathOperator{\sgn}{sgn}
\DeclareMathOperator{\Aut}{Aut}
\DeclareMathOperator{\diag}{diag}
\newcommand{\chroexp}{\overset{\longrightarrow}{\exp}}
\DeclareMathOperator{\re}{Re}
\DeclareMathOperator{\Span}{span}
\newcommand{\dd}[1]{\mathrm{d}#1}
\DeclareMathOperator{\ad}{ad}
\newcommand{\T}{\mathcal{T}}
\newcommand{\strokes}{\dot{H}^{1}_{\sharp}(J, \R^4)}


\begin{document}
\thispagestyle{plain}
\begin{center}
\begin{Large}
\textbf{Final Report MAP592: Parking 4-sphere swimmer}\\
\end{Large}
\vspace{1em}
Submitted by\\
\textsc{Philipp Weder}\\
ETH Zürich\\
under the supervision of\\
\end{center}
\vspace{1em}
\begin{minipage}[t]{0.5\textwidth}
\begin{center}
\textsc{François Alouges}\\
CMAP, École Polytechnique,\\
route de Saclay,\\
91128 Palaiseau Cedex,\\
France
\end{center}
\end{minipage}
\begin{minipage}[t]{0.5\textwidth}
\begin{center}
\textsc{Aline Lefebvre-Lepot}\\
CMAP, École Polytechnique,\\
route de Saclay,\\
91128 Palaiseau Cedex,\\
France
\end{center}
\end{minipage}


\begin{abstract}
This article is about the parking 4-sphere swimmer (\spr). This is a low-Reynolds number swimmer composed of four balls of equal radii. The four balls can move along the four axes passing through the four vertices of a tetrahedron and its midpoint. The balls do not rotate around their axes such that the shape of the swimmer is characterized by the length of the four arms, measured from the midpoint to the center of each ball. Yet, the swimmer may rotate freely around its center of mass. The governing dynamical system is presented and its geometric structure is displayed. Then it is shown that, in the first order range of small strokes, optimal periodic strokes for planar displacements with an additional rotation about the axis orthogonal to the plane of movement are ellipses embedded in $4d$ space, i.e. closed curves of the form $t \in [0, 2 \pi] \mapsto (\cos t) a + (\sin t) b$ for suitable vectors $a, b \in \R^4$. A simple analytic expression for the vectors $a$ and $b$ is derived. Eventually, a conjecture about the general case is made.
\end{abstract}

\newpage

\tableofcontents

\newpage

\section{Introduction}

In his novel paper \cite{Purcell1977}, Purcell treats the issue of swimming on a microscopic level and the principal problems linked to it for the first time. He especially illustrates, why any micro-organism trying to swim using a reciprocal movement like the one of a scallop, i.e. swimming by opening and closing a valve, cannot move. This observation, also known as the \emph{scallop theorem}\footnote{For a proof as well as an elementary introduction to the topic we refer to the encyclopedia article \cite{DeSimone2011}.} entails the problem of finding the simplest swimming mechanism at microscopic scales; that is, the capacity to advance using a periodic change of  shape - a swimming \emph{stroke} - in  the absence of external forces. A variety of such mechanisms has already been proposed and analyzed, see e.g. \cite{Alouges2013, Najafi2004, Purcell1977}.


The principal mathematical challenge of this problem stems from the low value of the Reynolds number $\re = \rho u L/\mu$ which gives an estimate of the relative importance of inertial to viscous forces for an object of characteristic length scale $L$ moving at speed $u$ through a Newtonian fluid of density $\rho$ and dynamic viscosity $\mu$. In the low Reynolds number regime, i.e. $\re \ll 1$, the inertial forces become irrelevant and consequently, micro-swimmers can only utilize the viscous resistance of the surrounding fluid to move. In mathematical terms, the micro-swimmers are governed by the steady Stokes equations, which are linear and symmetric under time reversal. In the case of the scallop, this means that whatever forward motion is caused by closing its valves, it will exactly be compensated by the movement produced by reopening them, regardless of the speed of these two processes.

Let us formulate the basic problem of swimming: given a periodic record of shape changes of a swimmer, predict the corresponding history of positions and orientations in space. A closely related question is the one of \emph{controllability}; that is, whether it is possible to achieve any prescribed position and orientation in space starting from an arbitrary initial position and orientation using an appropriate sequence of shape deformations. In fact, the peculiarity of swimming at low values of $\re$ stems from the fact that reciprocal shape changes cannot contribute to the net displacement as inertial forces are negligible. This especially becomes an issue when we only have few control variables at our disposal. Indeed, the scallop theorem actually shows that swimmers with only one control variable are not controllable.

Once the controllability of a swimmer is assured, the natural follow-up question is to ask which swimming strokes achieve a prescribed net displacement with the lowest energy consumption. In other words, mathematically speaking, we face an \emph{optimal control problem}. From the point of view of biology, this is relevant in the light of \emph{natural selection} among micro-organisms, whereas from the engineering standpoint, this is crucial since, as it is pointed out in \cite{Avron2004}, if a micro-swimmer should have an effect on macroscopic scales, it has to swim faster than bacteria and therefore consumes $10^4$ more energy than a bacterium. Hence, it is desirable to know the most effective swimming strokes for a micro-swimmer.

In \cite{Alouges2013}, a whole class of controllable micro-swimmer is presented. The said paper also puts forward a numerical method to address the problem of optimal swimming. However, their explicit dynamics as well as the structure of optimal swimming strokes remain largely unknown. In this paper, we will analyze further the swimmer \spr from \cite{Alouges2013} and shed a light on the latter aspects. The analysis will take place very much in the spirit of the treatment of the swimmer \textsc{SPr3} in \cite{Alouges2017}, which originally had been presented in \cite{Alouges2013} as well. In fact, the swimmer \spr is a natural generalization of the swimmer \textsc{SPr3}, capable of moving in the entire $3d$ space instead of just a plane. Although the principal techniques used in this paper are in close analogy to the ones in \cite{Alouges2017}, the more complex geometry of both the position and the shape space cause the analysis to be more involved.

Aim of this paper is to \emph{analytically} address the optimal control problem for \spr in the range of \emph{small} strokes.

The rest of the paper is organized as follows: in section \ref{sec:modeling}, we give both a geometric and a kinematic description of parking 4-sphere swimmer (\spr). Next, we introduce the control system treated in this paper. In section \ref{sec: symmetries}, we study the geometric structure of the control system taking advantage of the symmetries it has to satisfy due to the underlying Stokes  equations. In section \ref{sec: linearization}, we unravel the properties of the control system in the range of small strokes. Eventually, section \ref{sec: optimization} addresses the characterization of energy minimizing strokes for a special class of prescribed net displacements.






\section[Modeling]{Modeling of the swimmer as a control problem}
\label{sec:modeling}
We restrict ourselves to considering the swimmer \textsc{Spr4} proposed in \cite{Alouges2013}. Let $(S_1, S_2, S_3, S_4)$ be a regular reference tetrahedron centered at $c \in \R^3$ such that $\dist(c, S_i) = 1$ for all $i \in \N_4$. Then the swimmer consists of four balls $(B_i)_{i \in \N_4}$ of $\R^3$ centered at $b_i \in \R^3$, all of radius $a > 0$, such that the ball $B_i$ can move along the ray starting at $c$ and passing through $S_i$, see figure \ref{fig:reference tetrahedron and spr4}.

\begin{figure}[h]
    \centering
    \begin{minipage}{0.45\textwidth}
        \centering
        \includegraphics[width=0.9\textwidth]{/Users/philipp/Documents/GitHub/stage_cmap/images/tetrahedron.png}
    \end{minipage}
    \begin{minipage}{0.45\textwidth}
        \centering
        \includegraphics[width=0.9\textwidth]{/Users/philipp/Documents/GitHub/stage_cmap/images/spr4.png} % second figure itself
    \end{minipage}
    \caption{The reference tetrahedron and the parking 4-sphere swimmer (\textsc{SPr4}).}
    \label{fig:reference tetrahedron and spr4}
\end{figure}

 This reflects the situation where the balls are linked together by thin jacks that are able to elongate and retract. However, the viscous resistance of these jacks is neglected and therefore the fluid is assumed to permeate the entire open set $\R^3 \setminus \bigcup_{i = 1}^{4} \overline{B}_i$. The balls do not rotate around their arms which implies that the shape of the swimmer is completely determined by the four lengths $\xi_1, \xi_2, \xi_3, \xi_4$ of its arms, measured from $c$ to the center of each ball $b_i$. However, there are no restrictions for the rotation of the swimmer around the center $c$, i.e. for fixed arm lengths, the swimmer is considered to be a rigid body in a Stokesian fluid.
Hence, the geometrical configuration of the swimmer can be described by two sets of variables:
\begin{enumerate}
	\item The vector of \emph{shape variables} $\xi := (\xi_1, \xi_2, \xi_3, \xi_4) \in \M := (\sqrt{\tfrac{3}{2}}a, +\infty)^4 \subseteq \R_+^4$, from which one obtains the relative distances $(b_{ij})_{i,j \in \N_4}$ between the balls,  where the lower bound in the open intervals is chosen such that the balls cannot overlap.
	\item The vector of \emph{position variables} $p = (c, R) \in \mathcal{P} :=  \R^3 \times \SO(3)$, which encodes the global position and orientation in space of the swimmer.
\end{enumerate}
To be more precise, we consider the reference tetrahedron convexly spanned by the four unit vectors $z_1 := (2 \sqrt{2}/3,0,-1/3)$, $z_2 := (-\sqrt{2}/3,-\sqrt{2/3},-1/3)$, $z_3 := (-\sqrt{2}/3,\sqrt{2/3},-1/3)$ and $z_4 := (0,0,1)$. Position and orientation in $\R^3$ are then described by the coordinates of the center $c \in \R^3$ and the rotation $R \in SO(3)$ of the swimmer with respect to the reference orientation induced by the reference tetrahedron, i.e. if the arms are aligned with the reference tetrahedron, then this corresponds to the identity matrix $I \in \SO(3)$. Thus, we set $b_i := c + \xi_i R z_i$ for the center of the ball $B_i$.

The swimmer is completely described by the parameters $(\xi, p) \in \M \times \mathcal{P}$. Indeed, if we denote by $B_a$ the ball in $\R^3$ of radius $a$ centered at the origin, then for any $r \in \partial B_a$, the position of the current point on the $i$-th sphere of the swimmer in the state $(\xi, p)$ is given, for any $(\xi, p, r) \in \M \times \mathcal{P} \times \partial B_a$, by the function
\begin{equation}
	r_i(\xi, p, r) :=  c + R(\xi_i z_i + r).
\end{equation}
Note that the functions $(r_i)_{i \in \N_4}$ are analytic in $\M  \times \mathcal{P}$ and thus we can use them to calculate the instantaneous velocity on the $i$-th sphere $B_i$, which for any $(\xi, p, r) \in \M \times \mathcal{P} \times \partial B_a$ and every $i \in \N_4$ is given by
\begin{equation}
	u_i(\xi, p, r) = \dot{c} + \omega \times (\xi_i z_i + r) + R z_i \dot{\xi}_i,
\end{equation}
where $\omega$ is the axial vector associated with the skew matrix $\dot{R} R$.

In \cite{Alouges2013} it is shown that the system \textsc{SPr4}, i.e. both the shape $\xi$ and the position $p$, is controllable only using the rate of change $\dot{\xi}$ of the shape. To do so, we have to understand how $p$ responds to a variation in $\dot{\xi}$. To that end, the assumptions of \emph{self-propulsion} and \emph{negligible inertia of the swimmer} (which is equivalent to assuming a very low Reynolds number) are made. They imply that the total viscous force and torque exerted by the surrounding fluid on the swimmer must vanish. More precisely, for details see \cite{Alouges2013}, the system can be written as
\begin{equation}
\label{eq: control system}
	\dot{p} = F(R, \xi) \dot{\xi} := \left ( \begin{array}{c}
	F_c(R, \xi) \\
	F_\theta(R, \xi)
	\end{array}  \right ) \dot{\xi},
\end{equation}
where $\dot{c} = F_c(R, \xi) \dot{\xi}$ and $\dot{R} = F_\theta (R, \xi) \dot{\xi} $. 

In preparation for what follows, let us note that if we denote by $T_p \mathcal{P}$ the tangent space of the smooth manifold $\mathcal{P}$ at the point $p$, we have $F(R, \xi) \in \mathcal{L}(\R^4, T_{p}\mathcal{P})$ for any $R \in \SO(3)$ and $\xi \in \R^4$, where $\mathcal{L}(V, W)$ denotes the set ofs linear maps between two vector spaces $V$ and $W$. We quickly recall the fact that at any point $R \in \SO(3)$, see e.g. \cite{Hall2015} for details, we have 
\begin{equation}
	T_R \SO(3)  = \{R M \mid M \in \Skew_3(\R)\},
\end{equation}
where $\Skew_n(\R)$ denotes the set of skew-symmetric real matrices of size $n \times n$. Hence, we have in particular that for any $R \in \SO(3)$ and $\xi \in \R^4$
\begin{equation}
\begin{aligned}
	F_c(R, \xi) \in \mathcal{L}(\R^4, \R^3) \text{ and } F_{\theta}(R, \xi) \in \mathcal{L}(\R^4,T_R \SO(3))
\end{aligned}
\end{equation}
and therefore we can express both $F_c(R, \xi)$ and $F_{\theta}(R, \xi)$ as real matrices of size $3 \times 4$ once we have chosen a basis for the corresponding tangent spaces. Indeed, one verifies quickly that $\Skew_3(\R)$ is a three-dimensional vector space over $\R$.


 In analogy to \cite{Alouges2017}, it is important to note here that the control system $F$ is independent of $c$ due to the translational invariance of the Stokes equations. However, the translational invariance is not the only symmetry property that \textsc{SPr4} satisfies. The goal of the following section is to examine the structure of the control system $F$ in consequence of the symmetries it must fulfill being driven by the Stokes equations.































\section{Symmetries}
\label{sec: symmetries}
For any initial condition $p_0 = (c_0, R_0) \in \mathcal{P}$ and any control curve $\xi: I \subseteq \R \to \M$, with $I$ a neighborhood of zero, we denote by $\gamma(c_0, R_0, \xi): I \to \mathcal{P}$ the solution associated to the dynamical system
\begin{equation}
\label{eq:dynamical system}
\begin{aligned}
	&\dot{p} = F(R, \xi) \dot{\xi},& & p(0) := p_0,
\end{aligned}
\end{equation}
as well as by $\gamma_c(c_0, R_0, \xi)$ and $\gamma_\theta(c_0, R_0, \xi)$ its projections on $\R^3$ and $\SO(3)$, respectively, such that for any $t \in I$
\begin{equation}
	\dot{\gamma}(c_0, R_0, \xi)(t) = F(\gamma_\theta(c_0, R_0, \xi)(t), \xi(t))\dot{\xi}(t).
\end{equation}

\subsection{Rotational invariance}
Rotational invariance of the Stokes equations expresses the fact that the solution of the dynamical system (\ref{eq:dynamical system}) is invariant under rotations, i.e. changing the initial orientation results in the same rotation of the trajectory. More precisely, for any rotation $R \in \SO(3)$ we have for the spatial part of the solution
\begin{equation}
\label{eq:spatial rotational invariance}
	\gamma_c(c_0, R R_0, \xi)(t) = R \gamma_c (c_0, R_0, \xi)(t) + (I - R) c_0,
\end{equation}
and for the angular part of the solution
\begin{equation}
\label{eq: angular rotational invariance}
	\gamma_\theta(c_0, R R_0, \xi)(t) =  R \gamma_\theta(c_0, R_0, \xi)(t),
\end{equation}
at any point in time $t \in I$. Eventually, we can rigorously state the following symmetry property of the control system (\ref{eq:dynamical system}) with respect to rotations:

\begin{condition}[Rotational invariance]
\label{cond:rotational invariance}
Consider $\gamma(c_0, R_0, \xi)$ and $\gamma(c_0, R R_0, \xi)$ solutions to (\ref{eq:dynamical system}) for the corresponding initial conditions. Then, relations (\ref{eq:spatial rotational invariance}) and (\ref{eq: angular rotational invariance}) hold.
\end{condition}

\begin{remark}
To follow the reasoning of \cite{Alouges2017}, the symmetry relations satisfied by \textsc{SPr4} are stated as hypotheses on the solution $\gamma$. In so doing, the results work for any control system of the form (\ref{eq: control system}) and satisfying the hypotheses we state, e.g. rotational invariance, independently of these hypotheses being guaranteed by the invariance of the Stokes equations under a certain group of transformations. 
\end{remark}
So, let us prove the following symmetry property of the control system:
\begin{proposition}
\label{prop: rotational invariance}
Let $\xi_0 := \xi(0) \in \M$ denote the initial state of the control parameters and by $T_{\xi}\M$ the tangent space of $\M$ at $\xi$. If the control system (\ref{eq:dynamical system}) is invariant under rotations and for every $\xi \in \M$ it holds that $T_{\xi} \M \simeq \R^4$, then
\begin{equation}
	F_c(R, \xi) = R F_c(\xi) \text { and } F_\theta(R, \xi) = R F_{\theta} ( \xi),
\end{equation}
for every $(R, \xi) \in \SO(3) \times \M$, where $F_c(\xi) := F_{c}(I, \xi)$ and $F_{\theta}(\xi) := F_{\theta}(I, \xi)$. 
\end{proposition}

\begin{proof}
On the one hand, we have by definition of the dynamical system (\ref{eq: control system}) that
\begin{equation}
	\dot{\gamma}_c(c_0, R, \xi) = F_c(\gamma_{\theta}(c_0, R, \xi), \xi) \dot{\xi}.
\end{equation}
On the other hand, using equation (\ref{eq:spatial rotational invariance}) and once more the definition of the dynamical system (\ref{eq:dynamical system}), we obtain
\begin{equation}
	\dot{\gamma}_c (c_0, R, \xi) = R  \dot{\gamma}_c(c_0, I, \xi) = 
	R F_c(\gamma_{\theta}(c_0, I, \xi), \xi) \dot{\xi}.
\end{equation}
Therefore, $F_c(\gamma_{\theta}(c_0, R, \xi), \xi) \dot{\xi} = R F_{c}(\gamma_{\theta}(c_0, I, \xi), \xi) \dot{\xi}$ for every $R \in \SO(3)$. Since $T_{\xi_0} \M \simeq \R^4$, evaluation of the preceding expression at $t = 0$ yields $F_{c}(R, \xi_0) = R F_{c}(I, \xi_0)$, as desired. The proof for $F_{\theta}$ is completely analogous and thus is omitted.
\end{proof}

\subsection{Permutation of two arms}
In this section, we investigate the effect of a swap of two arms on the generic solution of the dynamical system (\ref{eq:dynamical system}).
To that end, let $P_{ij} \in \mathcal{L}(\R^4, \R^4)$ denote the map that interchanges the $i$-th and $j$-th coordinates, which corresponds to the swap of the arms $||i$ and $||j$, denoted by $(||i\leftrightsquigarrow ||j$), if applied to the shape space $\M$. In addition, let $S_{ij}$ denote the reflection of $\R^3$ sending arm $||i$ onto arm $||j$ in the reference orientation $I$. Geometrical inspection of the reference tetrahedron shows that $S_{ij}$ is always a reflection at a plane containing the remaining arms $||k$ and $||l$.

Before we formulate the symmetry conditions for the interchanging of two arms, we recall some results about how rotations behave under reflections. So far, we have only regarded the orientation of \textsc{SPr4} as a rotation matrix in $\SO(3)$. However, by Euler's rotation theorem to every such rotation matrix $R \in \SO(3)$ there exists a corresponding rotation vector $\omega \in \R^3$ which is collinear to the unique axis of rotation defined by $R$, i.e. $\omega$ is an eigenvector associated to the eigenvalue 1 of $R$. Its length is given by the angle of rotation around this axis. The rotation vector $\omega$ is then directly related to the rotation matrix $R$ via the map $\exp: \so(3) \to \SO(3)$, where $\so(3) := T_I \SO(3) = \Skew_3(\R)$ denotes the Lie algebra over $\SO(3)$, which we will illustrate in the following paragraphs.


It is clear that $\dim \Skew_3(\R) = 3$. In particular, if $R_1(\theta), R_2(\theta)$ and $R_3(\theta)$ denote the simple rotations around the $\hat{e}_1$-, $\hat{e}_2$- and $\hat{e}_3$-axis, where $\hat{e}_1, \hat{e}_2, \hat{e}_3$ denote the canonical basis vectors of $\R^3$, then the matrices
\begin{align}
\label{eq: L1}
	&L_1 = \frac{\dd}{\dd\theta}R_1(\theta)_{\mid \theta =0} = \left(\begin{array}{ccc}
	0 & 0 & 0 \\ 
	0 & 0 & -1 \\ 
	0 & 1 & 0
	\end{array}  \right ),\\
\label{eq: L2}
	&L_2 = \frac{\dd}{\dd\theta}R_2(\theta)_{\mid \theta =0} = \left (\begin{array}{ccc}
	0 & 0 & 1 \\ 
	0 & 0 & 0 \\ 
	-1 & 0 & 0
	\end{array}  \right ),\\
\label{eq: L3}
	&L_3 = \frac{\dd}{\dd\theta}R_3(\theta)_{\mid \theta =0} = \left (\begin{array}{ccc}
	0 & -1 & 0 \\ 
	1 & 0 & 0 \\ 
	0 & 0 & 0
	\end{array}  \right ),
\end{align}
form a basis of $\so(3)$, denoted by $\mathcal{L}$, consisting of the infinitesimal rotations around the corresponding axes. A trivial computation then shows that $R = \exp\left ( \sum_{k \in \N_3} \omega_k L_k \right )$. 

However, to clearly state the relations between reflections and orientations, let us first fix some notation. We denote by $\mathcal{E} := (e_1, e_2, e_3, e_4)$ the canonical basis of $\R^4$. Then we denote the matrix representing an arbitrary linear map $T: V \to W$ between two vector spaces $V$ and $W$ with respect to two bases $\mathcal{B}$ and $\mathcal{C}$ by $[T]_{\mathcal{B}}^{\mathcal{C}}$. Subsequently, we will make use of the adjoint map on $\so(3)$ associated to an orthogonal transformation $Q$ of $\R^3$. More precisely, if $Q$ is an orthogonal transformation of $\R^3$, we define the adjoint map $\ad_Q : \so(3) \to \so(3)$ by
\begin{equation}
\ad_Q(M) := Q M Q^T,
\end{equation}
which is clearly a linear map. Then we have the following result about orthogonal transformations and their adjoint maps, which we state in a rather general fashion as this will be beneficial at a later stage.

\begin{lemma}
\label{lem:mirror image of orientation}
Let $S, Q:\R^3 \to \R^3$ be a reflection at a plane through the origin and a rotation of $\R^3$, respectively. Then the representation matrices of their adjoint maps are given by
\begin{eqnarray}
[\ad_{S}]_{\mathcal{L}} = [-S]_{\mathcal{E}} & \text{and}  & [\ad_{Q}]_{\mathcal{L}} = [Q]_{\mathcal{E}}.
\end{eqnarray}
In particular, if $R \in \SO(3)$ characterizes the orientation of a rigid body, the orientation $\tilde{R}$ of its mirror image under a reflection $S$ is characterized by $\tilde{R} = SRS$.
\end{lemma}

\begin{proof}
Let us denote by $S_i$ the reflection of the $\hat{e}_i$ axis. Then one finds by a straightforward computation that the first statement is true for $S_i$ and $R_i(\theta)$ for all $\theta \in \R$ and $i \in \N_3$. Now the statement follows by decomposing any rotation $Q \in \SO(3)$ into its Euler angles and any reflection $S$ into $S = Q S_i Q^{T}$ for some rotation $Q$ and one of the elementary reflections $S_i$.

For the second part, let $S$ be an arbitrary reflection in $\R^3$ and let us denote by $\tilde{\omega}$ the rotation vector of the mirror image. Since the rotation vector of a rigid body is a pseudovector, we have $\tilde{\omega} = -S\omega$. In other words, we reflect the rotation vector and change its sense of rotation, for details see e.g. \cite{ChrisDoran2015}. Furthermore, let $\Omega \in \so(3)$ be the element $\sum_{k \in \N} \omega_k L_k$, i.e. $[\Omega]_{\mathcal{L}} = [\omega]_{\mathcal{E}}$, and define $\tilde{\Omega} \in \so(e)$ similarly corresponding to $\tilde{\omega}$. Then we have, $R = \exp(\Omega)$, $\tilde{R} = \exp(\tilde{\Omega})$, and finally
\begin{align}
   [\tilde{\Omega}]_{\mathcal{L}} = [\tilde{\omega}]_{\mathcal{E}} = -[S]_{\mathcal{E}} [\omega]_{\mathcal{E}} = [\ad_{S} \Omega] = [S \Omega S]_{\mathcal{L}}. 
\end{align}
Hence, we find $\exp(\tilde{\Omega}) = \exp( S \Omega S) = S \exp(\Omega) S$, as desired.
\end{proof}

With this lemma at hand, Stokes' equations allow us to state the following

\begin{condition}[Swap $(||i \leftrightsquigarrow ||j)$]
\label{cond:swap}
Consider $\gamma(c_0, I, P_{ij} \xi)$ and $\gamma(S_{ij}c_0, I, \xi)$ solutions to (\ref{eq:dynamical system}). Then, the following relations hold
\begin{equation}
	\gamma_c(c_0, I, P_{ij} \xi) = S_{ij} \gamma_c(S_{ij}c_0, I, \xi),
\end{equation}
and
\begin{equation}
	\gamma_{\theta}(c_0, I, P_{ij} \xi ) = \ad_{S_{ij}} \gamma_{\theta} (S_{ij} c_0, I, \xi).
\end{equation}
\end{condition}

\begin{figure}[h]
\centering
\includegraphics[width = 0.9\textwidth]{images/reflection.png}
\caption{The reflection $S_{12}$ applied to \textsc{SPr4} in the reference orientation corresponding to the swap $(||1 \leftrightsquigarrow ||2)$ }
\label{fig:reflection of swimmer}
\end{figure}

\begin{remark}
In physical terms, the previous condition stems from the invariance of Stokes equations with respect to the observation point, see figure \ref{fig:reflection of swimmer}. In fact, an observer watching the dynamics of $\gamma(S_{ij}c_0, I, \xi)$ of \textsc{SPr4} in a mirror in the reflection plane of $S_{ij}$ sees the dynamics $\gamma(c_0, I, P_{ij} \xi)$ of a micro-swimmer obtained from \textsc{SPr4} by swapping arms $||i$ and $||j$.
\end{remark}


To avoid chaos in our notation, we treat the spatial and angular parts now separately. For the spatial part, we find the following symmetry relation:

\begin{proposition}
\label{prop: spatial permutation invariance}
If the control system (\ref{eq: control system}) is invariant under the swap $(||i \leftrightsquigarrow ||j)$ and $T_{\xi} \M \simeq \R^4$ for all $\xi \in \M$, then for all $\xi \in \M$
\begin{align}
	 F_c(P_{ij} \xi) = S_{ij} F_c(\xi) P_{ij}.
\end{align}
\end{proposition}

\begin{proof}
Let $\gamma_c(c_0, R_0, P_{ij} \xi)$ be the spatial part of any solution to the control problem (\ref{eq:dynamical system}). The hypothesis of rotational invariance, i.e. (\ref{eq:spatial rotational invariance}), implies that
\begin{equation}
	\gamma_c(c_0, R_0, P_{ij} \xi) = R_0 \gamma_c(c_0, I, P_{ij} \xi) + (I - R_0)c_0.
\end{equation}
From condition \ref{cond:swap}, we then get
\begin{equation}
\label{eq:spatial permutation invariance int1}
	\gamma_c(c_0, R_0, P_{ij}\xi) = R_0 S_{ij} \gamma_c(S_{ij} c_0, I, \xi) + (I - R_0)c_0.
\end{equation}
As both $\gamma_c(c_0, R_0, P_{ij} \xi)$ and $\gamma_c(S_{ij} c_0, I, \xi)$ are spatial parts of solutions to the control system (\ref{eq:dynamical system}), we have on the one hand using Proposition \ref{prop: rotational invariance}
\begin{equation}
\label{eq: spatial permutation invariance int2}
	\dot{\gamma_c}(c_0, R_0, P_{ij} \xi) = \gamma_{\theta}(c_0, R_0, P_{ij} \xi) F_c(P_{ij} \xi) P_{ij} \dot{\xi},
\end{equation}
and on the other hand using (\ref{eq:spatial permutation invariance int1}) and once more (\ref{eq:dynamical system}) together with Proposition \ref{prop: rotational invariance}
\begin{equation}
\label{eq: spatial permutation invariance int3}
	\dot{\gamma_c}(c_0, R_0, P_{ij} \xi) = R_0 S_{ij} \dot{\gamma}_c(S_{ij} c_0, I, \xi) = R_0 S_{ij} \gamma_\theta(S_{ij} c_0, I, \xi) F_c(\xi) \dot{\xi}.
\end{equation}
Equating (\ref{eq: spatial permutation invariance int2}) and (\ref{eq: spatial permutation invariance int3}) at $t = 0$ yields $ F_c(P_{ij} \xi_0) = S_{ij} F_c(\xi_0) P_{ij}$, since by hypothesis $T_{\xi_0} \M \simeq \R^4$. As $\xi_0$ was arbitrary, we conclude.
\end{proof}

%For the angular part, we first have to choose a basis and fix some notation. Naturally, we choose the canonical basis $\mathcal{E} := (e_1, e_2, e_3, e_4)$ for $\R^4$. For $\so(3)$, we choose the basis $\mathcal{L} := (L_1, L_2, L_3)$, where the matrices $L_i$ are defined in (\ref{eq: L1}) - (\ref{eq: L3}). Then we denote the matrix representing an arbitrary linear map $T: V \to W$ between two vector spaces $V$ and $W$ with respect to two bases $\mathcal{B}$ and $\mathcal{B}'$ by $[T]_{\mathcal{B}}^{\mathcal{B}'}$. Subsequently, we will especially make use of the adjoint map on $\so(3)$ associated to an orthogonal transformation $Q$ of $\R^3$. More precisely, if $Q$ is an orthogonal transformation of $\R^3$, we define the adjoint map $\ad_Q : \so(3) \to \so(3)$ by
%\begin{equation}
%\ad_Q(M) := Q M Q^T,
%\end{equation}
%which is clearly a linear map.
%
%Thus, before we address the invariance property of the angular part of the control system under a swap of arms, let us prove the following
%
%\begin{lemma}
%\label{lem:representation matrix of adjoint of reflection}
%Let $S: \R^3 \to \R^3$ be an arbitrary reflection at a plane. Then the representation matrix of the adjoint map $\ad_S$ is given by $[\ad_S]_{\mathcal{L}} = -S$.
%\end{lemma}
%
%
%\begin{proof}
%We have already seen in the proof of Lemma \ref{lem:mirror image of orientation} that for the reflection $S_i$ of the $\hat{e}_i$-axis, we have $[\ad_{S_i}]_{\mathcal{L}} = -S_i$. Moreover, by direct calculation, we find for the elementary rotations $R_i(\theta)$ that $[\ad_{R_i(\theta)}]_{\mathcal{L}} = R_{i}(\theta)$. Now if $S$ is an arbitrary reflection, we find $Q \in SO(3)$ such that $S = Q S_i Q^T$ for one of the elementary reflections $S_i$. In particular, we find by Euler's rotation theorem three angles $\alpha, \beta, \gamma \in \R$ such that $Q = R_{3}(\gamma) R_{2}(\beta) R_{1}(\alpha)$ and therefore
%\begin{equation}
%\ad_Q = \ad_{R_3(\gamma)} \circ \ad_{R_{2}(\beta)} \circ \ad_{R_{1}(\alpha)},
%\end{equation}
%which implies that $[\ad_{Q}]_{\mathcal{L}} = Q$. Since we clearly have $\ad_{Q}^{-1} = \ad_{Q^{-1}} = \ad_{Q^T}$, we have $[\ad_{Q^T}]_{\mathcal{L}} = Q^T$. Hence, due to $\ad_S = \ad_Q \circ \ad_{S_i} \circ \ad_{Q^T}$, we find
%\begin{equation}
%[\ad_S]_{\mathcal{L}} = - Q S_i Q^T = -S,
%\end{equation}
%as desired.
%\end{proof}
For the angular part, we have now the following result, in the proof of which we exploit the first, more general statement of Lemma \ref{lem:mirror image of orientation}:
\begin{proposition}
\label{prop: angular permutation invariance}
If the control system (\ref{eq:dynamical system}) is invariant under the swap $(||i \leftrightsquigarrow ||j)$ and $T_{\xi}\M \simeq \R^4$ for all $\xi \in \M$, then for all $\xi \in \M$
\begin{equation}
[F_{\theta}(P_{ij} \xi)]_{\mathcal{E}}^{\mathcal{L}} = - [S_{ij}]_{\mathcal{E}}^{\mathcal{E}} [F_{\theta}(\xi)]_{\mathcal{E}}^{\mathcal{L}} [P_{ij}]_{\mathcal{E}}^{\mathcal{E}}.
\end{equation}
\end{proposition}

\begin{proof}
Let $\gamma_\theta(c_0, R_0, P_{ij}\xi)$ be the angular part of any solution to the control problem (\ref{eq:dynamical system}). By the rotational invariance hypothesis, i.e. (\ref{eq: angular rotational invariance}), we have
\begin{equation}
	\gamma_\theta(c_0, R_0, P_{ij}\xi) = R_0 \gamma_\theta(c_0, I,P_{ij} \xi).
\end{equation}
Then Condition \ref{cond:swap} implies that
\begin{equation}
\label{eq: angular permutation invariance int1}
	\gamma_\theta(c_0, R_0,P_{ij} \xi) = R_0 \ad_{S_{ij}}( \gamma_\theta(S_{ij}c_0, I, \xi)).
\end{equation}
Since both $\gamma_\theta(c_0, R_0, P_{ij} \xi)$ and $\gamma_\theta(S_{ij} c_0, I, \xi)$ are the angular parts of solutions to the control problem (\ref{eq:dynamical system}), we obtain with Proposition \ref{prop: rotational invariance} on the one hand
\begin{equation}
\label{eq: angular permutation invariance int2}
	\dot{\gamma_\theta}(c_0, R_0, P_{ij} \xi)= \gamma_\theta(c_0, R_0, P_{ij} \xi) F_{\theta}(P_{ij} \xi) P_{ij } \dot{\xi}
\end{equation}
and on the other hand using (\ref{eq: angular permutation invariance int1}) and once more Proposition \ref{prop: rotational invariance}
\begin{equation}
\label{eq: angular permutation invariance int3}
	\dot{\gamma_\theta}(c_0, R_0, P_{ij} \xi) =  R_0 \ad_{S_{ij}}( \dot{\gamma_\theta}(S_{ij} c_0, I, \xi)) = R_0 \ad_{S_{ij}}( \gamma_\theta(S_{ij} c_0, I, \xi) F_{\theta}(\xi) \dot{\xi}).
\end{equation}
Imposing equality of (\ref{eq: angular permutation invariance int2}) and (\ref{eq: angular permutation invariance int3}) at $t = 0$ yields
\begin{equation}
	F_{\theta}(P_{ij} \xi_0) P_{ij}  \dot{\xi}(0) = \ad_{S_{ij}} (F_{\theta}(\xi_0) \dot{\xi}(0)).
\end{equation}
Therefore, by Lemma \ref{lem:mirror image of orientation}, we have
\begin{equation}
	[F_{\theta}(P_{ij} \xi_0)]_{\mathcal{E}}^{\mathcal{L}}  [P_{ij}]_{\mathcal{E}}^{\mathcal{E}} [\dot{\xi}(0)]_{\mathcal{E}} = [\ad_{S_{ij}}( F_{\theta}(\xi_0) \dot{\xi}(0))]_{\mathcal{L}} = -[S_{ij}]_{\mathcal{E}}^{\mathcal{E}} [F_{\theta}(\xi_0)]_{\mathcal{E}}^{\mathcal{L}} [\dot{\xi}(0)]_{\mathcal{E}}.
\end{equation}
Recalling that $T_{\xi_0} \M \simeq \R^4$ as well as the arbitrariness of $\xi_0$ finish the proof.
\end{proof}

In the following sections, we will always understand $F_{\theta}(\xi)$ as a matrix of size $3 \times 4$ and thus, since no confusion may arise, we will abandon the slightly cumbersome notation and identify $[F_{\theta}(\xi)]_{\mathcal{E}}^{\mathcal{L}}$ with $F_{\theta}(\xi)$.







\section[The small strokes regime]{The small stroke regime}
\input{sec4_report.tex}

\section{Energy minimizing strokes}
\label{sec: optimization}
In the spirit of \cite{Alouges2013} and \cite{Alouges2017}, we follow the notion of swimming efficiency suggested by Lighthill \cite{Lighthill1952} and we adopt the following notion of optimality: energy minimizing strokes are the ones that minimize the kinematic energy dissipated while trying to reach a given net displacement $\delta p \in \R^3 \times \so(3) \simeq \R^6$. Mathematically speaking, the total energy dissipation due to a stroke $\xi \in H^1_{\sharp}(J, \R^4)$ can be evaluated through an adequate quadratic energy functional, c.f. \cite{Alouges2013},
\begin{equation}
 \mathcal{G}(\xi) := \int_{J} \mathfrak{g}(\xi(t))\dot{\xi}(t) \cdot \dot{\xi}(t) \dd t,
\end{equation}
where the energy density $\mathfrak{g} \in C^1(\R^4)$ is a function with values in the space of symmetric and positive definite matrices $M_{4 \times 4}(\R)$. In other words, $\mathfrak{g}$ defines a continuous Riemannian metric on $\M$. In the small stroke regime, we can approximate the energy density by $\mathfrak{g}(\xi) = \mathfrak{g}(0) + o(1)$, where $\mathfrak{g}(0) \in M_{4 \times 4}(\R)$ is symmetric and positive definite. More precisely,
\begin{equation}
\label{eq: linearized energy functional}
	\mathcal{G}(\xi) := \int_{J} Q_{\mathfrak{g}}(\dot{\xi}(t))\dd t,
\end{equation}
with $Q_{\mathfrak{g}}(\eta) := \mathfrak{g}(0)\eta \cdot \eta$. For the same symmetry reasons as discussed in section \ref{sec: symmetries}, we necessarily have for all $\eta \in \R^4$
\begin{align}
	Q_{\mathfrak{g}}(P_{ij} \eta) = Q_{\mathfrak{g}}(\eta),\; i,j \in \N_4,
\end{align}
where $P_{ij}$ denotes the permutation matrix swapping the $i$-th and $j$-th entries. By direct computation, on finds that the symmetric positive matrix $G$ representing the quadratic form $Q_{\mathfrak{g}}$ is of the form
\begin{equation}
G = \left ( \begin{array}{cccc}
\kappa & h & h & h \\ 
h & \kappa & h & h \\ 
h & h & \kappa & h \\ 
h & h & h & \kappa
\end{array} \right ),
\end{equation}
for two parameters $h$ and $\kappa > \max(h, -3h)$. In particular, we observe that $G \tau_k = (\kappa - h ) \tau_k$ for $k \in  \N_3$ and $G \tau_4 = (\kappa + 3h) \tau_4$. In the following, we denote by $\mathfrak{g}_1 := \mathfrak{g}_2 := \mathfrak{g}_3 := \kappa - h$ and $\mathfrak{g}_4 := \kappa + 3h$ the eigenvalues of $G$. Furthermore, the eigenvalues $(\mathfrak{g}_i)_{i \in \N_4}$ allow us to diagonalize  $G$ as
\begin{eqnarray}
G = U \Lambda_{\mathfrak{g}} U^T, & U := [\tau_1 | \tau_2 |\tau_3 |\tau_4], & \Lambda_{\mathfrak{g}} := \diag(\mathfrak{g}_i).
\end{eqnarray}

The goal of this section is the minimization of $\mathcal{G}$ in $H_{\sharp}^1(J, \R^4)$ subject to a prescribed net displacement $\delta p \in \R^6$, i.e. subject to the constraint (c.f. (\ref{eq: net displacement}))
\begin{equation}
\label{eq: constraint}
\begin{aligned}
	 \delta p = \mathfrak{h}_{c} \sum_{k \in \N_3}\left ( \int_{J} \det( \xi(t) | \dot{\xi}(t) | \tau_{k+1} | \tau_{k+2}) \dd t \right ) f_k\\
	+ \mathfrak{h}_{\theta}  \sum_{k \in \N_3}\left ( \int_{J} \det ( \xi(t) | \dot{\xi}(t) | \tau_{k} | \tau_{4}) \dd t\right ) f_{k + 3},
\end{aligned}
\end{equation}
with $\mathfrak{h}_c = - 2 \sqrt{6} \alpha$ and $\mathfrak{h}_{\theta} = - 2 \sqrt{6} \delta$. The existence of such solutions follows readily by the direct method of variational calculus.


\subsection{Bivectors in four dimensions}
\label{sec:bivectors}
Let us recall in this section the basic definitions around the notion of \emph{bivectors}, where we refer to \cite{Lounesto2006} for details. As the abstract definition of general $k$-vectors is not very useful for our purposes, we merely illustrate them in $\R^3$, which then generalizes easily to higher dimensions. In $\R^3$, a bivector is an oriented plane segment; that is, a small piece of surface having a magnitude given by the area of the surface element as well as a direction given by the attitude of the plane the surface element lies in as well as a sense of rotation. Together, they form the vector space $\bigwedge^2 \R^3$. We can represent a bivector $\omega \in \bigwedge^2 \R^3$ as a small parallelogram which suggests that we can think of it as some product of the two vectors along its sides. This is realized by the \emph{exterior product}, also called \emph{wedge product}, $u \wedge v$ of two vectors $u$ and $v$. The product $u \wedge v$ then represents the bivector obtained by sweeping $v$ along $u$. This operation yields a direct link between $\R^3$ and the vector space $\bigwedge^2 \R^3$ of bivectors, a basis of which is given by
\begin{equation}
 \{\hat{e}_1 \wedge \hat{e}_2, \hat{e}_1 \wedge \hat{e}_3, \hat{e}_2 \wedge \hat{e}_3\},
\end{equation}
if $\{\hat{e}_1, \hat{e}_2, \hat{e}_3\}$ is a basis of $\R^3$. In fact, the standard scalar product on $\R^3$ extends to a scalar product on $\bigwedge^2 \R^3$ by
\begin{equation}
( u_1 \wedge u_2 , v_1 \wedge v_2 ) = \det \left (\begin{array}{cc}
u_1 \cdot v_1 & u_1 \cdot v_2 \\ 
u_2 \cdot v_1 & u_2 \cdot v_2
\end{array}  \right).
\end{equation}
In particular, $( u \wedge v, u \wedge v ) = |u|^2 |v|^2 \sin^2\psi$, where $\psi$ is the angle between the vectors $u$ and $v$. Eventually, the norm of a bivector $\omega = \omega_{12} \hat{e}_1 \wedge \hat{e}_2 + \omega_{13} \hat{e}_1 \wedge \hat{e}_3 + \omega_{23} \hat{e}_2 \wedge \hat{e}_3$ is given by
\begin{equation}
|\omega| = \sqrt{\omega_{12}^2 + \omega_{13}^2 + \omega_{23}^2}.
\end{equation}
These definitions then extend naturally to all higher dimensions. In particular, we note that if $\{e_1, e_2, e_3, e_4\}$ again denotes the canonical basis of $\R^4$, a basis of the space $\bigwedge^2 \R^4$ is given by
\begin{equation}
\{e_{12}, e_{13}, e_{14}, e_{23}, e_{24}, e_{34}\},
\end{equation}
where we write $e_{ij} := e_i \wedge e_j$ to simplify notation.

To finish this section, we will point out some properties of the space $\bigwedge^2 \R^4$ and how it differs from $\bigwedge^2 \R^3$, which at a later point will illustrate why the optimal control curves for $\textsc{SPr3}$ and \textsc{SPr4} in general do not have the same structure. Nevertheless, we will be able treat a simplified situation for \textsc{SPr4} in the same manner as \textsc{SPr3} in \cite{Alouges2017}.

As a matter of fact, one peculiarity of the bivectors in $\R^3$ is that they are isomorphic to the space $\R^3$ itself. This is realized by the so-called \emph{Hodge dual operator} $\star$, c.f. \cite{Lounesto2006} p. 38, which is defined in a way such that for any two vectors $u,v \in \R^3$ one has
\begin{equation}
\label{eq: hodge star}
u \wedge v = \star(u \times v),
\end{equation}
where $\times$ denotes the usual cross product. This entails two things: First, every bivector in $\R^3$ is \emph{simple}; that is, it can be written as the wegde product of two vectors. Second, every bivector in $\R^3$ defines one unique plane in $\R^3$. It is due to this underlying geometrical fact that one can always reduce the Fourier coefficients of an optimal control curve to one pair of vectors, by which one attains the same net displacement at the same energy consumption in \cite{Alouges2017}, c.f. Proposition 15. Furthermore, this implies that the optimal control curves are situated in a certain plane defined by the vector of the net displacement.

Yet, in $\R^4$, the geometry of its bivectors proves to be more involved. As $\dim \bigwedge^2 \R^4 = 6$, it is clear that the bivectors in $\R^4$ are not isomorphic to $\R^4$ itself. In particular, in $\R^4$ not all bivectors are simple. Indeed, the bivector $e_1 \wedge e_2 + e_3 \wedge e_4 \in \bigwedge^2 \R^4$ cannot be written as an exterior product of just two vectors in $\R^4$. Nevertheless, any bivector in $\R^4$ can be written as the sum of two orthogonal simple bivectors \cite{Lounesto2006}. Moreover, we have the following criterion to determine whether a bivector is simple:

\newpage

\begin{lemma}
\label{lem:simple bivector}
A bivector $\omega \in \bigwedge^2 \R^4$ is simple if and only if $\omega \wedge \omega = 0$.
\end{lemma}

\begin{proof}
If $\omega \in \bigwedge^2 \R^4$ is a simple bivector, i.e. if there are vectors $u,v \in \R^4$ such that $\omega = u \wedge v$, then it is clear from the anticommutativity and associativity of the wedge product that
\begin{equation}
\omega \wedge \omega = (u \wedge v) \wedge (u \wedge v) = - u \wedge u \wedge v \wedge v = 0.
\end{equation}
The inverse requires a rather lengthy proof by induction, see the lecture notes on projective geometry by Nigel Hitchin, chapter 3, p.48 \cite{Hitchin2003}.
\end{proof}

In what follows, we will find that the net displacement actually identifies with a bivector of $\R^4$ and the solution to the optimization problem in \cite{Alouges2017} then suggests that we should be able to find a similar solution for our optimization problem, at least in the case where the net displacement is a simple bivector. The preceding lemma then serves us to identify certain subspaces of $\bigwedge^2 \R^4$ consisting only of simple bivectors.

\subsection{G-Orthogonalization}
We begin by rewriting the energy functional (\ref{eq: linearized energy functional}) and the constraint (\ref{eq: constraint}) in terms of the orthonormal basis of eigenvectors $(\tau_i)_{i \in \N_4}$ of the matrix $G$. The change of variable $\eta(t) := U^T \xi(t) \in H_{\sharp}^{1}(J, \R^4)$, allows us to write
\begin{equation}
\label{eq: G-orth energy functional}
\mathcal{G}_{U}(\eta) = \int_{J} \Lambda_{\mathfrak{g}} \dot{\eta}(t) \cdot \dot{\eta}(t) \dd t,
\end{equation}
with $\mathcal{G}_{U}(\eta) := \mathcal{G}(\xi) = \mathcal{G}(U \eta)$. For the constraint, we note that
\begin{equation}
\det(\xi |\dot{\xi} | \tau_i | \tau_j) =  \det U \det (\eta | \dot{\eta} | e_i | e_j) = \det(\dot{\eta} | \eta | e_i |e_j),
\end{equation}
since $\det U = -1$. Eventually, we can express the determinants more elegantly in terms of exterior products. In fact, by direct calculation one obtains $\det(\dot{\eta} | \eta | e_k |e_4) = (\dot{\eta} \wedge \eta, e_{k + 1} \wedge e_{k+2})$ and $\det(\dot{\eta} |\eta | e_{k + 1} |e_{k + 2}) = (\dot{\eta} \wedge \eta, e_k \wedge e_4)$, for $k \in \N_3$ taken mod 3. Then, the isomorphism sending the standard basis $\{f_i\}_{i \in \N_6}$ of $\R^6$ onto the ordered basis 
\begin{equation}
\label{eq: basis of bivectors}
(e_{14}, e_{24}, e_{34}, e_{23}, e_{31}, e_{12})
\end{equation}
of $\bigwedge^2 \R^4$, where we write $e_{ij} := e_i \wedge e_j$, allows us to rewrite (\ref{eq: constraint}) as
\begin{equation}
\label{eq: G-orth constraint}
\Lambda_{\mathfrak{h}}^{-1} \delta p = \int_{J} \dot{\eta}(t) \wedge\eta(t) \dd t,
\end{equation}
with $\Lambda_{\mathfrak{h}} := \diag(\mathfrak{h}_{c}, \mathfrak{h}_{c}, \mathfrak{h}_{c}, \mathfrak{h}_{\theta}, \mathfrak{h}_{\theta}, \mathfrak{h}_{\theta})$.

\subsection{Fourier transformation of the minimization problem}
We denote by $\ell^2(\R^4)$ the space of sequences $\mathbf{u} := (u_n)_{n \in \N}$ in $\R^4$ such that the norm
\begin{equation}
	||\mathbf{u}||_{\ell^2(\R^4)} := \sqrt{ \sum_{n \in \N} |u_n|^2 }
\end{equation}
is finite. Consequently, we denote by $\dot{\ell}^2(\R^4)$ the Hilbert space of sequences $\mathbf{u} = (u_n)_{n \in \N} \in \ell^2(\R^4)$ such that $(n u_n)_{n \in \N} \in \ell^2(\R^4)$. As the elements in $H_{\sharp}^{1}(J, \R^4)$ are $2\pi$-periodic, we can express $\eta$ in terms of its Fourier series as
\begin{equation}
\eta(t) := \frac{1}{2} a_0 + \sum_{n \in \N} \cos(nt) a_n + \sin(n t) b_n,
\end{equation}
with $(a_n, b_n)_{n \in \N} \in \dot{\ell}^2(\R^4) \times \dot{\ell}^2(\R^4)$. Substitution of the Fourier series of $\dot{\eta}$ into the energy functional (\ref{eq: G-orth energy functional}) yields due to Parseval's equality
\begin{align}
\mathcal{G}_{U} (\eta) := \int_{J} \Lambda_{\mathfrak{g}} \dot{\eta}(t) \cdot \dot{\eta} dt &= \pi \sum_{n  \in \N} n^2(\Lambda_{\mathfrak{g}} a_n \cdot a_n + \Lambda_{\mathfrak{g}} b_n \cdot b_n) \\  &=
\frac{1}{2} ||\mathbf{u}||_{\ell^2(\R^4)}^2 + \frac{1}{2} ||\mathbf{v}||_{\ell^2(\R^4)}^2,
\end{align}
where we have set
\begin{align}
\label{eq:relation Fourier coeffs of eta}
	\mathbf{u} := (u_n)_{n \in \N} := \sqrt{2 \pi \Lambda_{\mathfrak{g}}}(n a_n)_{n \in \N} \text{ and } \mathbf{v} := (v_n)_{n \in \N} := \sqrt{2 \pi \Lambda_{\mathfrak{g}}} (n b_n)_{n \in \N}.
\end{align}
Clearly, we have $(\mathbf{u}, \mathbf{v}) \in \ell^2(\R^4) \times \ell^2(\R^4)$. As a result of the $L_{\sharp}^2(J, \R^4)$-orthogonality of the Fourier trigonometric system, we can express the constraint (\ref{eq: G-orth constraint}) in terms of Fourier coefficients as $2 \pi \sum_{n \in \N} \tfrac{1}{n} (nb_n) \wedge (n a_n) = \Lambda_{\mathfrak{h}}^{-1} \delta p$. Returning to our old notation for a moment, we note that for any $n \in \N$, we have
\begin{equation}
2 \pi n^2 \det(b_n | a_n | e_i | e_j) = \frac{\sqrt{\mathfrak{g}_i \mathfrak{g}_j}}{\sqrt{\det \Lambda_{\mathfrak{g}}}} \det(v_n | u_n |e_i | e_j),
\end{equation}
and hence by setting $\tilde{\Lambda}_{\mathfrak{g}} := \diag(\mathfrak{g}_c, \mathfrak{g}_c, \mathfrak{g}_c, \sqrt{\mathfrak{g}_c \mathfrak{g}_{\theta}}, \sqrt{\mathfrak{g}_c \mathfrak{g}_{\theta}}, \sqrt{\mathfrak{g}_c  \mathfrak{g}_{\theta}})$, with $\mathfrak{g}_1 :=\mathfrak{g}_2 := \mathfrak{g}_3 := \mathfrak{g}_c$ and $\mathfrak{g}_4 := \mathfrak{g}_\theta$, we eventually find
\begin{equation}
\sqrt{\det \Lambda_{\mathfrak{g}}} (\Lambda_{\mathfrak{h}} \tilde{\Lambda}_{\mathfrak{g}})^{-1} \delta p = \sum_{n \in \N} \frac{v_n  \wedge u_n}{n}
\end{equation}
We thus have proved the following

\begin{proposition}
\label{prop: l2-minimization}
The $H_{\sharp}^1(J, \R^4)$ minimization of the functional $\mathcal{G}_U$ given by (\ref{eq: G-orth energy functional}) under the constraint (\ref{eq: G-orth constraint}) is equivalent to the minimization of the functional
\begin{equation}
\label{eq:l2-energy}
	\mathcal{F}(\mathbf{u}, \mathbf{v}) := \frac{1}{2} ||\mathbf{u} ||^2_{\ell^2(\R^4)} + \frac{1}{2} ||\mathbf{v}||^2_{\ell^2(\R^4)},
\end{equation}
defined in the product Hilbert space $\ell^2(\R^4) \times \ell^2(\R^4)$ and subject to the constraint
\begin{equation}
\label{eq:l2-constraint}
\sum_{n \in \N} \frac{1}{n} v_n \wedge u_n = \omega \text{ with } \omega := \sqrt{\det \Lambda_{\mathfrak{g}}}(\Lambda_{\mathfrak{h}} \tilde{\Lambda}_{\mathfrak{g}})^{-1} \delta p,
\end{equation}
where $\delta p \in \R^3 \times \so(3)$ is a prescribed net displacement of position and orientation.
\end{proposition}

We observe that we are in a very similar situation as in \cite{Alouges2017} with the fundamental difference however that this time the constraint is a bivector. Nevertheless, it is natural to try to generalize the approach in \cite{Alouges2017}, which in fact is true at least in the case of $\omega$ being simple.

\subsection[The simple case]{The simple case}
With the remarks from section \ref{sec:bivectors}, we are able to solve the constrained minimization problem of Proposition \ref{prop: l2-minimization} in a similar manner to \cite{Alouges2017} whenever the net displacement is a simple bivector. In fact, we retrieve essentially the same result, i.e. that the optimal control curves are ellipses in a certain plane defined by the net displacement. Let us prove

\begin{proposition}
\label{prop:simple reduction}
If $\omega$ is a simple bivector, then for any $(\mathbf{u}, \mathbf{v}) \in \ell^2(\R^4) \times \ell^2(\R^4)$ such that the constraint (\ref{eq:l2-constraint}) holds, there exist two vectors $u,v \in \R^4$ such that for the sequences $\mathbf{u}_{\star} := \mathbf{e}_1 u$ and $\mathbf{v}_\star := \mathbf{e}_1 v \in \ell^2(\R^4)$ one has 
\begin{equation}
\mathcal{F}(\mathbf{u_{\star}}, \mathbf{v_{\star}}) = \mathcal{F}(\mathbf{u}, \mathbf{v}) \text{ and } v \wedge u = \omega.
\end{equation}
\end{proposition}

\begin{proof}
If $\omega = 0$, then the proof is trivial. Thus, let us denote by $\hat{\omega}$ the unit bivector associated to $\omega$. For a couple $(\mathbf{u}, \mathbf{v}) \in \ell^2(\R^4) \times \ell^2(\R^4)$, we then choose $u,v \in \R^4$ such that the following relations hold:
\begin{eqnarray}
\label{eq:reduction int1}
	|u| = ||\mathbf{u} ||_{\ell^2(\R^4)}, &  |v| = ||\mathbf{v} ||_{\ell^2(\R^4)} , & \frac{u \wedge v}{|u \wedge v|} = \hat{\omega}.
\end{eqnarray}
The latter is possible since $\hat{\omega}$ is a simple bivector by hypothesis. Hence, there exist $x, y \in \R^4$ such that $\omega = x \wedge y$. Then we have for all $x', y' \in \Span\{x,y\}$ such that $x' \wedge y' \neq 0$ that $x' \wedge y'/|x' \wedge y'| = \hat{\omega}$. Furthermore, we have $u \wedge v = ||\mathbf{u}||_{\ell^2(\R^4)} ||\mathbf{v}||_{\ell^2(\R^4)} (\sin\psi)\hat{\omega}$, where $\psi$ is the angle between $u$ and $v$. Therefore, the equality $u \wedge v = \omega$ can be satisfied by choosing the angle $\psi \in (0, \pi)$ such that
\begin{equation}
 \sin \psi = \frac{|\omega|}{||\mathbf{u}||_{\ell^2(\R^4)} ||\mathbf{v} ||_{\ell^2(\R^4)}}
 \end{equation}
This is possible under the condition that the right hand side  of the previous equation is not greater than one. In fact, we have using the Cauchy-Schwarz inequality
\begin{equation}
|\omega| \leq \sum_{n \in \N} \frac{1}{n} |v_n \wedge u_n| \leq \sum_{n \in \N} |v_n| |u_n| \leq ||\mathbf{u} ||_{\ell^2(\R^4)} ||\mathbf{v} ||_{\ell^2(\R^4)}.
\end{equation}
Finally, from (\ref{eq:reduction int1}) we obtain
\begin{equation}
\mathcal{F}(\mathbf{u_\star}, \mathbf{v_{\star}}) = \frac{1}{2} |u|^2 + \frac{1}{2} |v|^2 = \frac{1}{2} ||\mathbf{u}||_{\ell^2(\R^4)}^2 + \frac{1}{2} ||\textbf{v}||_{\ell^2(\R^4)}^2,
\end{equation}
which concludes the proof.
\end{proof}

We immediately have

\begin{corollary}
\label{cor:simple reduction}
If $\omega$ is a simple bivector, the minimization problem for $\mathcal{F}$ in $\ell^2(\R^4) \times \ell^2(\R^4)$, under the constraint (\ref{eq:l2-constraint}), is equivalent to the minimization in $\R^4 \times \R^4$ of the function
\begin{equation}
\label{eq:finite dim energy}
 	f(u,v) := \frac{1}{2}|u|^2 + \frac{1}{2} |v|^2
 \end{equation} 
 under the constraint
 \begin{equation}
 \label{eq:finite dim constraint}
 v \wedge u = \omega.
 \end{equation}
\end{corollary}

\begin{proof}
It suffices to observe that if $\mathcal{V}_{\omega}$ denotes the subset of $\ell^2(\R^4) \times \ell^2(\R^4)$ satisfying the constraint (\ref{eq:l2-constraint}) and by $V_{\omega}$ the subset of $(u, v)  \in \R^4 \times \R^4$ such that $u \wedge v = \omega$, then Proposition \ref{prop:simple reduction} yields
\begin{equation}
\min_{(\mathbf{u}, \mathbf{v}) \in \mathcal{V}_{\omega}}\mathcal{F}(\mathbf{u}, \mathbf{v}) = \min_{(u, v) \in V_\omega} \mathcal{F}(\mathbf{e}_1 u, \mathbf{e}_1 v) = \min_{(u,v) \in V_{\omega}} f(u,v).
\end{equation}
\end{proof}

Let us now prove the following


\begin{proposition}
\label{prop:finite dim minimization}
Any couple of vectors $(u_{\star}, v_{\star}) \in \R^4 \times \R^4$ minimizing the function $f$ given in (\ref{eq:finite dim energy}) and subject to the constraint (\ref{eq:finite dim constraint}) with $\omega = x \wedge y$ a simple bivector, is characterized by the following conditions:
\begin{eqnarray}
\label{eq:finite dim minimization conditions}
|u_{\star}|^2 = |v_{\star}|^2 = |\omega|, & 
u_{\star} \cdot v_{\star} = 0.
\end{eqnarray}
Therefore, any two vectors $\sigma, \mu \in \Span\{x, y\}$ such that $|\sigma|^2 = |\mu|^2 =|\omega|$ and $\sigma \cdot \mu = 0$, the couple $(\sigma, \mu) \in \R^4 \times \R^4$ is a (global) constrained minimizer for $f$.
\end{proposition}

\begin{remark}
To construct such a couple, it suffices to scale $x$ to get $u$ and then find $v$ by Gram-Schmidt orthogonalization.
\end{remark}

\begin{proof}
Note that to find the minimizers of the problem (\ref{eq:finite dim energy}) - (\ref{eq:finite dim constraint}), the constraint $u \wedge v = \omega$ implies the existence of a $\psi \in (0, \pi)$ such that $|u||v| \sin \psi= |\omega|$. Hence, the constrained minimization for $f$ is equivalent to the unconstrained minimization of the function $\hat{f}: \R^4 \times (0, \pi) \to \R$ defined by
\begin{equation}
(u, \psi) \mapsto \frac{1}{2} |u|^2 + \frac{1}{2} \frac{|\omega|^2}{|u|^2 \sin^2\psi},
\end{equation}
whose stationary satisfy $\psi_{\star} = \frac{\pi}{2}$ and $|u_{\star}|^2 = |\omega|$. This shows the necessity of the conditions stated in (\ref{eq:finite dim minimization conditions}). To show sufficiency of the condition, one observes that for any such points one has $\hat{f}(u_{\star}, \psi_{\star}) = |\omega|$. Indeed, for any $(u, \psi) \in D_{ijk, \psi} \in \R^4 \times (0, \pi)$ we have
\begin{equation}
\hat{f}(u, \psi) \geq \frac{1}{2} \frac{|u|^4 + |\omega|^2}{|u|^2} = |\omega| + \frac{1}{2}\frac{(|\omega| - |u|^2)^2}{|u|^2} \geq |\omega| = \hat{f}(u_{\star}, \psi_{\star}).
\end{equation}
A straightforward calculation shows then that for such $\sigma$ and $\mu$, one has $\sigma \wedge \mu \mid \mid \hat{\omega}$ since they are in the plane spanned by the vectors $x$ and $y$. By construction, one has $|\sigma \wedge \mu| = |\sigma| |\mu| = |\omega|$.
\end{proof}

Pasting everything worked out above together leads to the final result of this section. We have


\begin{theorem}
\label{thm:optimal control curves in the simple case}
Let $\delta p \in \R^3 \times \so(3) \simeq \bigwedge^2 \R^4$ be a prescribed net displacement. Moreover, assume that $\delta p \simeq x \wedge y$ identifies with a simple bivector. Then, any minimizer $\xi \in H_{\sharp}^1(J, \R^4)$ of the energy functional (\ref{eq: linearized energy functional}) subject to the constraint (\ref{eq: constraint}) is of the form
\begin{equation}
\xi(t) := (\cos t) a + (\sin t) b,
\end{equation}
i.e. an ellipse of $\R^4$ centered at the origin and contained in the plane spanned by the vectors $a$ and $b$. The vectors $a,b \in \R^4$ are obtained as follows:
\begin{enumerate}
\item We compute the vector $\omega$ via the relation
\begin{equation}
\omega := \diag \left (\frac{\sqrt{\mathfrak{g}_c \mathfrak{g}_{\theta}}}{\mathfrak{h}_c}, \frac{\sqrt{\mathfrak{g}_c \mathfrak{g}_{\theta}}}{\mathfrak{h}_c}, \frac{\sqrt{\mathfrak{g}_c \mathfrak{g}_{\theta}}}{\mathfrak{h}_c}, \frac{\mathfrak{g}_c}{\mathfrak{g}_\theta}, \frac{\mathfrak{g}_c}{\mathfrak{g}_\theta}, \frac{\mathfrak{g}_c}{\mathfrak{g}_\theta} \right ) \delta p \simeq \tilde{x} \wedge \tilde{y}.
\end{equation}
Then we consider two vectors $u,v \in \Span\{\tilde{x}, \tilde{y}\}$ such that
\begin{equation}
\label{eq:global minimizer condition}
|u|^2 = |v|^2 = |\omega| \text{ and } u \cdot v = 0.
\end{equation}

\item We set $\hat{\omega} := \omega/|\omega|$ and we calculate the vectors $a$ and $b$ via the relations
\begin{equation}
\label{eq:global minimizer form}
\begin{aligned}
a := \frac{U \Lambda_{\mathfrak{g}}^{-1/2}}{\sqrt{2 \pi}} u,&& b := \frac{U \Lambda_{\mathfrak{g}}^{-1/2}}{\sqrt{2 \pi}} v.
\end{aligned}
\end{equation}
\end{enumerate}
We then have $ v \wedge u = \omega$ and the minimum value of $\mathcal{G}$ is equal to $|\omega|$.

In addition, the vectors $a$ and $b$ are $\mathfrak{g}$-orthogonal, i.e. with respect to the inner product defined for every $x,y \in \R^4$ by $(x, y)_{\mathfrak{g}} := 2 \pi \Lambda_{\mathfrak{g}} x \cdot y$, and have the same $\mathfrak{g}$-norm $|a|_{\mathfrak{g}}^2 = |b|_{\mathfrak{g}}^2 = |\omega|$. 
\end{theorem}

%\begin{remark}
%Suppose that $\delta p \mid \mid e_{14}$, i.e. a net displacement along the $\hat{e}_1$-axis. Then we have both $\hat{\omega} \in D^{*}_{124}$ as well as $\hat{\omega} \in D_{134}^{*}$. In the first case, we have $\star \hat{\omega} = e_2$ and thus we can choose $u := \sqrt{|\omega|} e_4$ to satisfy (\ref{eq:global minimizer condition}). Then we have $v = -\sqrt{|\omega|} e_1$ and hence we get from (\ref{eq:global minimizer form}) that
%\begin{eqnarray}
%	a := \sqrt{\frac{|\omega|}{2 \pi \mathfrak{g}_4}} \tau_{4}&, & b := -\sqrt{\frac{|\omega|}{2 \pi \mathfrak{g}_{1}}} \tau_1.
%\end{eqnarray}
%In the case $\hat{\omega} \in D^{*}_{134}$ we have $\star\hat{\omega} = e_3$ and therefore we find the same vectors $a$ and $b$ from the relations (\ref{eq:global minimizer condition}) and (\ref{eq:global minimizer form}). Indeed, the resulting vectors are independent of the choice of one of the two possible spaces $D^*_{ijk}$ for any direction $\hat{\omega} = e_{ij}$. In other words, an energy minimizing net displacement along the $\hat{e}_1$-axis with respect to the standard euclidean space $(\R^4, (\cdot, \cdot)_2)$ is realized by an elliptic stroke contained in the plane spanned by the vectors $\tau_1$ and $\tau_4$. On the other hand, with respect to the inner-product space $(\R^4, (\cdot, \cdot)_{\mathfrak{g}})$, the energy optimizing strokes are circles of radius $\sqrt{|\omega|}$. More generally, if $\hat{\omega} = e_{ij}$, then the energy optimizing strokes describe ellipses with respect to the standard inner-product or $(\cdot, \cdot)_2$ circles of radius $\sqrt{|\omega|}$ with respect to the inner-product $(\cdot, \cdot)_{\mathfrak{g}}$ in the plane spanned by the vectors $\tau_i$ and $\tau_j$.
%\end{remark}

\begin{proof}
From Proposition \ref{prop:finite dim minimization}, Corollary \ref{cor:simple reduction} and then Proposition \ref{prop: l2-minimization}, we get that any $u, v \in \Span\{\tilde{x}, \tilde{y}\}$ satisfying the relations
\begin{eqnarray}
 u \cdot v = 0, & |u|^2 = |v|^2 =  |\omega|, & \omega := \sqrt{\det \Lambda_{\mathfrak{g}}} (\Lambda_{\mathfrak{h}} \tilde{\Lambda}_{\mathfrak{g}})^{-1} \delta p,
\end{eqnarray}
is associated to a (global) constrained minimizer for $\mathcal{G}_{U}$, via the curve $\eta(t) := (\cos t) \tilde{a} + (\sin t) \tilde{b}$, where the Fourier coefficients $\tilde{a}, \tilde{b} \in \R^4$ are related to $\omega$ (c.f. \ref{eq:relation Fourier coeffs of eta}) by $(\sqrt{2 \pi \Lambda_{\mathfrak{g}}}) \tilde{a} = u$ and $(\sqrt{2 \pi \Lambda_{\mathfrak{g}}}) \tilde{b} = v$. The minimum value of the energy is then $\mathcal{G}_{U}(\eta) = |\omega|$.

Finally, in the $\mathfrak{g}$-orthogonal reference frame, the inner product is defined by $(x, y)_{\mathfrak{g}} := 2 \pi \Lambda_{\mathfrak{g}}x \cdot y$ for $x,y \in \R^4$. Let us denote by $|\cdot|_{\mathfrak{g}}$ the associated norm. Then we have the following relations:
\begin{align}
|\tilde{a}|_{\mathfrak{g}}^2 = |\tilde{b}|_{\mathfrak{g}}^{2} = |\omega| \;  \text{ and } \; (\tilde{a}, \tilde{b})_{\mathfrak{g}} = 0.
\end{align}
Applying the orthogonal map $U$ to $\tilde{a}$ and $\tilde{b}$ finishes the proof.
\end{proof}

\subsection{Some examples of simple net displacements}
In light of Theorem \ref{thm:optimal control curves in the simple case} presented above, one might ask whether there are concrete cases in which the net displacement happens to be a simple bivector. It turns out that there is convenient correspondence for engineering purposes between certain subspaces of $\bigwedge^2 \R^4$ consisting only of simple bivectors and certain net displacements. Indeed, note that the condition in Lemma \ref{lem:simple bivector} is in particular satisfied if all coefficients corresponding to a certain index are zero, e.g. $\omega_{i4} = 0$ for $i \in \N_3$. This yields four subspaces $D^*_{ijk}$ of $\bigwedge^2 \R^4$ consisting only of simple bivectors. Then, by inspection of the basis of $\bigwedge^2 \R^4$, we have the following correspondences:
\begin{align*}
D^{*}_{123} &\longleftrightarrow\text{ rotations around all three axes } \hat{e}_1, \hat{e}_2, \hat{e}_3\\
D^{*}_{124} &\longleftrightarrow \text{ translation in the $\hat{e}_1\hat{e}_2$-plane, rotation around the $\hat{e}_3$-axis }\\
D^{*}_{134} &\longleftrightarrow  \text{ translation in the $\hat{e}_1\hat{e}_3$-plane, rotation around the $\hat{e}_2$-axis }\\
D^{*}_{234} &\longleftrightarrow  \text{ translation in the $\hat{e}_2\hat{e}_3$-plane, rotation around the $\hat{e}_1$-axis }
\end{align*}
By comparison, the non-simple bivector $e_{12} + e_{34}$ corresponds to the net displacement $e_3 + L_3$, i.e. a screw motion. This kind of movement requires a solution to the general problem.


\subsection{The general case}

Let us now address the case of a general net displacement, i.e. $\delta p \in \R^6 \simeq \bigwedge^2 \R^4$ which identifies to a non-simple bivector. The observations from section \ref{sec:bivectors} suggest that the optimal curve in the general case consist of two ellipses in certain planes reflecting the fact that $\delta p$ is the sum of two orthogonal simple bivectors and any simple bivector represents a plane in $\R^4$. We will indeed be able to prove this result. However, we have to return to variational calculus to do so.


\subsubsection{The optimization problem in the variational setting}
More precisely, we will establish the structure of the optimal control curves using the Euler-Lagrange equation associated with the optimization problem. To that end, let us quickly recast the optimization problem in its original form:

\begin{align}
\label{eq:original_optimization_problem}
\begin{cases}
 \text{Find } \min_{\xi \in \dot{H}^1_{\sharp}} \int_{J} G \dot{\xi}(t) \cdot \dot{\xi}(t) \dd t\\
 \text{under the constraints }\\
 \int_{J} M_i \xi(t) \cdot \dot{\xi}(t) \dd t = \delta p_i, i \in \N_6.
 \end{cases}
\end{align}

So, we are in the setting of a variational problem with six isoperimetric constraints. For $i \in \N_6$, denote by $K_i: \strokes \times \strokes \times J \to \R$ the map
\begin{align}
	K_i(\xi, \eta, t) := M_i \xi(t) \cdot \eta(t).
\end{align}
Furthermore, denote by $\mathcal{K}_i : \strokes \to \R$ the functional
\begin{align}
	\mathcal{K}_i(\xi) := \int_{J} K_i(\xi, \dot{\xi}, t) \dd t.
\end{align}
Then, the six isoperimetric constraints read $\mathcal{K}_i(\xi) = \delta p_i$ for $i \in \N_6$. Now, let us denote by $\delta \mathcal{G}$ and $\delta \mathcal{K}_i$ the first variations of $\mathcal{G}$ and the $\mathcal{K}_i$, respectively. Then a slight adaptation of Proposition 2.1.3. in \cite{Kielhoefer2018} shows that $\xi \in \strokes$ is a minimizer of (\ref{eq:original_optimization_problem}) and if $\xi$ is not critical for the constraints, i.e. $\delta \mathcal{K}_1(\xi), \dotsc, \delta \mathcal{K}_6(\xi)$ are linearly independent, then $\xi$ satisfies the Euler-Lagrange equation:

\begin{align}
\label{eq:euler_lagrange}
\frac{\dd}{\dd t} \frac{\partial}{\partial \dot{\xi}} \left ( G \dot{\xi}(t) \cdot \dot{\xi}(t) + \sum_{i \in \N_6} \mu_i K_i(\xi, \dot{\xi}, t)\right ) = \frac{\partial}{\partial \xi} \left ( G \dot{\xi}(t) \cdot \dot{\xi}(t) + \sum_{i \in \N_6} \mu_i K_i(\xi, \dot{\xi}, t)\right ),
\end{align}
for some $\mu \in \R^6$. To make use use of equation (\ref{eq:euler_lagrange}), let us prove the following

\begin{proposition}
\label{prop:linearly_independent_constraints}
Let $\delta p \in \R^6 \simeq \bigwedge^2 \R^4$ be non-simple and $\xi \in \strokes$ a minimizer of (\ref{eq:original_optimization_problem}). Then the functionals $\delta \mathcal{K}_1(\xi), \dotsc, \delta \mathcal{K}_6(\xi)$ are linearly independent.
\end{proposition}

\begin{proof}
Let $\lambda_1, \dotsc, \lambda_6 \in \R$ be such taht $\sum_{i \in \N_6} \lambda_i \delta \mathcal{K}_i(\xi) $ is the zero functional in $(\strokes)^*$. Note that by the periodicity of $\xi$ and integration by parts, we have for $h \in \strokes$ that
\begin{align}
	\delta \mathcal{K}_i(\xi) h =  2 \int_J M_i \dot{\xi}(t) \cdot h(t) \dd t.
\end{align}
Setting $\Omega(\lambda) := \sum_{i \in \N_6} \lambda_i M_i$ for $\lambda \in \R^6$, we have that the functional $h \in \strokes \mapsto \int_J \Omega(\lambda) \dot{\xi}(t) \cdot h(t) \dd t$ is the zero functional. Let us take for the time being $h = \Omega(\lambda) \dot{\xi}$ but note that $h$ is not necessarily in $\strokes$. Then we have
\begin{align}
	0 = \int_J \Omega(\lambda) \dot{\xi}(t) \cdot h(t) \dd t = ||h||_{L^2}^{2},
\end{align}
i.e. $h \equiv 0$. This can only happen in the two cases $\xi(t) \in \ker \Omega(\lambda)$ for all $t \in J$ or $\lambda_1 = \dotsm = \lambda_6 = 0$, in the latter of which we are done. So let us suppose that we are in the former case. Note that the matrix $\Omega(\lambda)$ is skew symmetric. Hence, we find an orthogonal transformation $S$ such that $\Omega(\lambda) = S \Sigma(\lambda) S^T$ with

\begin{align}
\Sigma(\lambda) = \diag \left (\left( \begin{array}{cc}
0 & \nu_+(\lambda) \\ 
- \nu_+ (\lambda) & 0
\end{array} \right ) , \left ( \begin{array}{cc}
0 & \nu_{-}(\lambda) \\ 
- \nu_{-}(\lambda) & 0
\end{array} \right ) \right ).
\end{align}
Denoting by $P$ and $Q$ the projections $\R^6 \to \R^3$ on the first and the last three coordinates, respectively, the scalars $\nu_{pm}$ are given by
\begin{align}
	\nu_{\pm} = 2 \sqrt{3} \sqrt{A \pm \sqrt{A^2 - K}},
\end{align}
with $A := \alpha^2 |P \lambda|^2 + \delta^2 |Q \lambda|^2$ and $K := 4 \alpha^2 \delta^2 |P\lambda \cdot Q \lambda|^2$. Hence, only $\nu_-$ can vanish, which implies that $\ker \Omega(\lambda)$ is at most of dimension two. However, this is excluded by the Lemma below, as $\delta p$ is assumed to be non-simple. Thus, we have $||h||_{L^2} > 0$. Now, approximation of $\dot{\xi}$ by smooth functions shows that $h \in \strokes \mapsto \int_J \Omega(\lambda) \dot{\xi}(t) \cdot h(t) \dd t$ cannot be the zero functional. Therefore, we must have $\lambda_1 = \dotsm = \lambda_6 = 0$, which finishes the proof.
\end{proof}

To complete the proof of Proposition \ref{prop:linearly_independent_constraints}, let us prove the following
\begin{lemma}
	Let $\xi \in \strokes$ be a control curve. Suppose that $\xi(t) \in D$ for all $t \in J$, where $D \subset \R^4$ is a plane through the origin. Then, the net displacement due to $\xi$ is a simple bivector.
\end{lemma}

\begin{proof}
Note that if $\xi$ only takes values in the plane $D$, then the rescaled Fourier coefficients in relation (\ref{eq:l2-constraint}) must also lie in a plane $D'$ (not necessarily the same). Hence, the net displacement due to $\xi$ lies in fact in $\bigwedge^2 D'$. Thus, it must be simple as $D'$ is of dimension two.
\end{proof}

\subsubsection{Structure of the solutions to the Euler-Lagrange equation}

By direct computation, one finds that that after integration and utilizing the fact that $\xi$ has average, that the Euler-Lagrange equation (\ref{eq:euler_lagrange}) reads
\begin{equation}
\label{eq:euler_lagrange_integrated}
G \dot{\xi} - \Omega(\mu) \xi = 0,
\end{equation}
where $\Omega(\mu)$ is the same skew symmetric matrix as above. To reveal the structure of the solutions to (\ref{eq:euler_lagrange_integrated}), we need to apply two basis transformations: First, setting $\eta := G^{1/2} \xi$ yields
\begin{equation}
\dot{\eta} - \tilde{\Omega}(\mu) \eta = 0,
\end{equation}
where $\tilde{\Omega}(\mu) := \sum_{i \in \N_6} \mu_i G^{-1/2} M_i G^{-1/2}$, which is still a skew symmetric matrix. Hence, we find an orthogonal transformation $Q$ such that $\tilde{\Omega} = Q \tilde{\Sigma}(\mu) Q^T$ with
\begin{align}
	\tilde{\Sigma}(\mu) \diag \left (\left( \begin{array}{cc}
0 & \sigma_1(\mu) \\ 
- \sigma_1(\mu) & 0
\end{array} \right ) , \left ( \begin{array}{cc}
0 & \sigma_2(\mu) \\
- \sigma_2(\mu) & 0
\end{array} \right ) \right ).
\end{align}

So setting $\phi := Q \eta$, we have the equation $\dot{\phi} = \tilde{\Sigma}(\mu) \phi$, the solution of which is given by $\phi(t) = \exp \left ( \tilde{\Sigma}(\mu) t \right )\phi_0$ with $\phi_0 := QG^{1/2}\xi(0)$. By defining the vectors
\begin{align}
	\phi_1 := (\phi_{0,1}, \phi_{0,2}, 0, 0)^T, & & \phi_1' := (\phi_{0,2}, - \phi_{0,1}, 0, 0)^T\\
	\phi_2 := (0,0,\phi_{0,3}, \phi_{0, 4})^T, & & \phi_2' := (0,0,- \phi_{0,4}, \phi_{0,3})^T,
\end{align}
we can recast the curve $\phi$ as
\begin{eqnarray}
\phi(t) = \sum_{i \in \N_2} [\cos(\sigma_i(\mu)t) \phi_i + \sin(\sigma_i(\mu)t) \phi_i'], & & t \in J.
\end{eqnarray}

Resubstituting the basis transformations, we find that $\xi$ must be of the form
\begin{eqnarray}
\label{eq:general_optimal_curve}
\xi(t) = \sum_{i \in \N_2} [\cos(\sigma_i(\mu)t) a_i + \sin(\sigma_i(\mu)t) a_i'], & & t \in J.
\end{eqnarray}
Note that the vectors $\phi_1, \phi_1', \phi_2$ and $\phi_2'$ are pairwise orthogonal. So, $\phi$ is in fact a rotation in two completely orthogonal planes of $\R^4$. In particular, this implies that the vectors $a_1, a_1', a_2$ and $a_2'$ are orthogonal in the reference system of $G$.

\subsubsection{Existence of integer eigenvalues}
It is clear that the eigenvalues $\sigma_1(\mu)$ and $\sigma_2(\mu)$ must be integers for the curve $\xi$ in (\ref{eq:general_optimal_curve}) to be periodic. Hence, we must be able to choose $\mu \in \R^6$ such that $\sigma_1(\mu), \sigma_2(\mu) \in \N$. This question is addressed in this subsection.

First, note that we cannot have $sigma_1(\mu) = \sigma_2(\mu)$ since then the relation (\ref{eq:l2-constraint}) would imply that $\delta p$ is simple. Next, one finds that the eigenvalues are given by
\begin{align}
	\sigma_{1,2}(\mu) = \frac{2 \sqrt{3}}{g_c \sqrt{g_\theta}} \sqrt{A \mp \sqrt{A^2 - K}},
\end{align}
where this time $A := \alpha^2 g_c |P \mu|^2 + \delta^2 g_{\delta} |Q \mu|^2 > 0$ and $K := 4 \alpha^2 \delta^2 g_c g_\theta |P\mu \cdot Q \mu|^2 > 0$. Clearly, we have $\sigma_2(\mu) \geq \sigma_1(\mu)$. So let us find the values for $\mu \in \R^6$ such that
\begin{align}
\label{eq:integer_condition}
	\sigma_1(\mu) = k \in \N \text{ and } \sigma_2(\mu) = lk, l \in \N.
\end{align}
This does not cover all possible pairs of integers. However, we will see in the next subsection that this is sufficient.

Imposing the above equations on the eigenvalues leads to the definition of the quadratic form associated to the positive definite matrix
\begin{align}
	\Gamma := \frac{48}{g_c^2 g_\theta} \left (\begin{array}{cc}
	\alpha^2 g_c I_3 & 0 \\ 
	0 & \delta^2 g_\theta I_3
	\end{array}  \right ).
\end{align}

Indeed, one can show that the solution sets to (\ref{eq:integer_condition}) are given by $E_+^{k,l} \cup E_{-}^{k,l}$, where
\begin{align}
E_+^{k,l} := \{\mu \in \R^6 \mid \Gamma \mu \cdot \mu = \frac{(1 + l^2)k^2}{l^2}\} \text{ and } E_{-}^{k,l} := \{\mu \in \R^6 \mid \Gamma \mu \cdot \mu = (1 + l^2)k^2\},
\end{align}
i.e. the solutions are the union of two specific ellipsoids in $\R^6$. Note in particular, that this calculation covers the case $\sigma_1(\mu) = 1$ and $\sigma_2(\mu) = 2$. So, we have assured the existence of periodic solutions to the Euler-Lagrange equation and we will show in the following subsection that the latter choice of values for the eigenvalues is indeed optimal.

\subsubsection{The optimal control curves}
Up to now, the geometric structure of the optimal control curves is clear from (\ref{eq:general_optimal_curve}). So, let us now settle their explicit construction from a given non-simple net displacement. To that end, let us transform the Fourier modes $a_1, a_1', a_2$ and $a_2'$ to the $G$-orthogonal by setting $\tilde{a}_1 := U^Ta_1$ and similarly for the others. Then, substituting the curve into the energy functional $G$ yields
\begin{align}
\label{eq:optimal_energy}
	\mathcal{G}(\xi) = \sigma_1(\mu)^2[\Lambda_{\mathfrak{g}}\tilde{a}_1 \cdot \tilde{a}_1 + \Lambda_\mathfrak{g} \tilde{a}_1' \cdot \tilde{a}_1'] + \sigma_2(\mu)^2[\Lambda_{\mathfrak{g}} \tilde{a}_2 \cdot \tilde{a}_2 + \Lambda_{\mathfrak{g}}\tilde{a}_2' \cdot \tilde{a}_2']\\
	=\frac{\sigma_1(\mu)}{2 \pi} [|u_{\sigma_1}|^2 + |v_{\sigma_1}|^2] + \frac{\sigma_2(\mu)}{2 \pi}[|u_{\sigma_2}|^2 + |v_{\sigma_2}|^2],
\end{align}
with 
\begin{align}
	u_{\sigma_i} := \sqrt{2 \pi \sigma_i(\mu) \Lambda_\mathfrak{g}}\tilde{a}_i &\text{ and }& v_{\sigma_i} := \sqrt{2 \pi \sigma_i(\mu) \Lambda_{\mathfrak{g}}} \tilde{a}_i',
\end{align}
for $i \in \N_2$. In particular, it follows from relation (\ref{eq:l2-constraint}) that
\begin{equation}
	\sqrt{\det \Lambda_{\mathfrak{g}}}(\Lambda_{\mathfrak{h}} \Lambda_{\mathfrak{g}})^{-1} \delta p = v_{\sigma_1} \wedge u_{\sigma_1} + v_{\sigma_2} \wedge u_{\sigma_2}.
\end{equation}
We observe that the latter relation in the sense of (\ref{eq:l2-constraint}) is independent of the indices $\sigma_1$ and $\sigma_2$.  So, we can choose $\sigma_1(\mu) = 1$ and $ \sigma_2(\mu) = 2$ up to permutation such that $|u_{\sigma_1}|^2 + |v_{\sigma_1}|^2 \geq |u_{\sigma_2}|^2 + |v_{\sigma_2}|^2$.

Moreover\footnote{Recall that we have shown in the section \ref{sec:bivectors} that the reverse is also true.}, note that the vectors $u_1, v_1, u_2,$ and $v_2$ are pairwise orthogonal due to the $G$-orthogonality of $a_1, a_1', a_2,$ and $a_2'$. Eventually, due to the bilinearity of the exterior product, we can always suppose that $|u_1| = |v_1|$ and $|u_2| = |v_2|$, which further minimizes the energy. Finally, we can summarize the argument above in the following

\begin{theorem}
Let $\delta p \in \R^6 \simeq \bigwedge^2 \R^4$ a net displacement identifying to a non-simple bivector, then the energy minimizing curve which attains $\delta p$ is given by
\begin{align}
\xi(t) := cos(t) a_1 + sin(t) a_1' + cos(2t) a_2 + sin(2t) a_2', t \in J,
\end{align}
where the vectors $a_1, a_1',a_2, a_2' \in \R^4$ are determined as follows:
\begin{enumerate}
\item First we decompose the still simple bivector $\omega := \sqrt{\det \Lambda_{\mathfrak{g}}}(\Lambda_{\mathfrak{h}} \tilde{\Lambda}_{\mathfrak{g}})^{-1} \delta p$ into the sum of two orthogonal simple bivectors, i.e.
\begin{align}
	\omega = v_1 \wedge u_1 + v_2 \wedge u_2,
\end{align}
with  all four vectors $u_1, u_2, v_1, v_2$ pairwise orthogonal and  $|u_i| = |v_i|$, for $i \in \N_2$.

\item If necessary, we permute the indices such that $|u_2|^2 + |v_2|^2 \leq |u_1|^2 + |v_1|^2$.

\item We set
	\begin{align}
		a_1 := \frac{U \Lambda_{\mathfrak{g}}^{-1/2}}{\sqrt{2 \pi}} u_1 &, & a_1' := \frac{U \Lambda_{\mathfrak{g}}^{-1/2}}{\sqrt{2 \pi}}v_1\\
	a_2 := \frac{U \Lambda_{\mathfrak{g}}^{-1/2}}{\sqrt{4 \pi}} u_2& , & a_2' := \frac{U \Lambda_{\mathfrak{g}}^{-1/2}}{\sqrt{4 \pi}} v_2.	
	\end{align}
\end{enumerate}
Then the four vectors $a_1, a_1', a_2, a_2'$ are $\mathfrak{g}$-orthogonal and the minimum value of the energy functional is $\frac{1}{2 \pi} [|u_1|^2 + |v_2|^2] + \frac{1}{ \pi} [|u_2|^2 + |v_2|^2]$.
\end{theorem} 






\section{The long arms approximation}
Despite having characterized the dynamics of \textsc{SPr4} as well as the optimal control curves for any prescribed net displacement, we are still missing the physical parameters $\mathfrak{a}, \alpha, \beta, \delta, \lambda, \kappa$ and $h$. Again inspired by \cite{Alouges2017}, we consider the regime of very long arms compared to the radius $a$ of the spheres $B_i$. To that end, let us review the fluid model. Recall that we neglected the arms such that our fluid domain was given by $\Omega := \bigcup_{i \in \N_4} \overline{B}_i$. The geometry of $\Omega$ is completely determined by the common radius $a$ of the four spheres and the matrix $b = (b_i)_{i \in \N_4}$ having as columns the centers of the spheres. Whenever necessary, we will emphasize this dependence by writing $\Omega := \Omega_b(a)$.

Since we are considering the swimmer \textsc{SPr4} to be microscopically small, we can assume the Reynolds number to be low such that the governing equations are given by Stokes' equations
\begin{align}
\label{eq:stokes}
\begin{cases}
- \mu \Delta u + \nabla p &= 0 \text{ in } \Omega,\\
\mathrm{div} u  &= 0  \text{ in } \Omega,
\end{cases}
\end{align}
subject to the traction boundary condition $- \boldsymbol\sigma n = f$ on $\partial \Omega$, and to the far-field condition $u(x) = o(|x|^{-1})$ as $|x| \to \infty$. The variables $u$ and $p$ denote the velocity field and the pressure of the fluid, respectively, $\mu$ is the viscosity, $n$ the outer unit normal to $\partial \Omega$ directed towards the interior of the balls, $\boldsymbol \sigma := \mu \nabla^{\mathrm{sym}}u - p I$ the Cauchy stress tensor, and $f$ the traction applied to the swimmer at each point of the boundary. The swimmer has a rigid structure. Hence, no-slip boundary conditions are imposed on the spheres, where we recall that the instantaneous velocity on the $i$-th sphere in the state $(\xi, p, \sigma) \in \M \times \mathcal{P} \times \partial B_a $ is given by
\begin{equation}
	u_i(\xi, p, \sigma) = \dot{c} + \omega \times (\xi_i z_i + \sigma) + R z_i \dot{\xi}_i,
\end{equation}
where $\omega$ is the axial vector associated with the skew matrix $\dot{R} R^T$. Furthermore, the unique velocity field $u$ solution of (\ref{eq:stokes}) can be expressed, for almost every $x \in \partial \Omega$, by a single-layer potential
\begin{align}
\label{eq:single_layer}
	u(x) = \int_{\partial \Omega} G(x - y) f(y) \dd y, & & G(x) := \frac{1}{8 \pi \mu} \left ( \frac{I}{|x|} + \frac{x \otimes x}{|x|^3} \right ),
\end{align}
where the stokeslet $G$ is the fundamental solution to Stokes' equations.

\subsection{The dynamics of \textsc{SPr4} in the long arms regime}

Due to the simple nature of $\Omega_b(a)$, we can recast the integral representation (\ref{eq:single_layer}) for any $i \in \N_4$ and any $\sigma \in -b_i + \partial B_a$ in the form
\begin{align}
	u(\sigma + b_i) = \int_{\partial B_a} G(\sigma - y)f_i(y) \dd y + \sum_{j \neq i \in \N_4} \int_{\partial B_a} G(b_{ij} + \sigma - y) f_j(y) \dd y,
\end{align}
where $b_{ij} := b_i - b_j, f_j(y) := f(b_j + y)$. In the very long arms regime, i.e., when $\min_{i < j} |b_{ij}| \gg a$, we have at the leading order that $G(b_{ij} + \sigma - y) \sim G(b_{ij})$, whenever $i \neq j \in \N_4$. Consequently, we can write the velocity field $u_{\mid \partial B_i}$ on the $i$-th sphere as
\begin{align}
\label{eq:velocity_field_laa}
	u_i(\sigma) := \frac{1}{|\partial B_a|} \int_{\partial B_a} G(\sigma - y) f_i(y) \dd y + \frac{1}{|\partial B_a|} \sum_{j \neq i \in \N_4} G(b_{ij}) \int_{\partial B_a} f_j(y) \dd y, & & \sigma \in \partial B_a.
\end{align}
A second benefit of the simple geometry of $\Omega$ is that it supports uniform tractions on the spheres, i.e. that if $f_i$ is constant on $\partial B_i$, so is $u_i$. Hence, we deduce from (\ref{eq:velocity_field_laa}) that for any $\sigma \in \partial B_a$
\begin{align}
\label{eq:velocity_field_uniform}
	u_i := u_i(\sigma) = \frac{1}{6 \pi \mu a} f_i + \sum_{j \neq i \N_4} G(b_{ij}) f_j.
\end{align}
The arms of the swimmer are assumed to have the same initial length $\xi \in \R^+$. So, as stated by (\ref{eq:velocity_field_uniform}), when the $i$-th arm of the swimmer is $\xi_0 + \xi_i$ with $\xi_0 \gg a$, the velocity field $u_i$ associated to the constant tractions $(f_i)_{i \in \N_4}$ is uniform on the boundary of the $i$-th sphere. Thus, without loss of generality, we can interpret the velocity field $u_i$ as applied to the center $b_i$ of the $i$-th ball. Furthermore, due to the negligible inertia, the total viscous force and torque exerted by the surrounding fluid on the swimmer must vanish, i.e. the dynamics must satisfy the balance equations
\begin{eqnarray}
\label{eq:balance_equations}
	\sum_{i \in \N_4} f_i = 0, & & \sum_{i \in \N_4} (b_i - c) \times f_i = 0.
\end{eqnarray}
For any $k \in \N_3$, the map $f \mapsto \omega_k(b_i, f) := ((b_i - c) \times f) \cdot \hat{e}_k$ defines a linear form on $\R^3$, where $(\hat{e}_1, \hat{e}_2, \hat{e}_3)$ denotes the standard basis of $\R^3$. Hence, since in particular $b_i - c = (\xi_0 + \xi_i) R z_i$, we find vectors $\omega_{ki}(\xi, R) \in \R^3$ such that
\begin{eqnarray}
	\omega_{ki}(\xi_i, R) \cdot f_i = \omega_k(b_i, f_i), & & k \in \N_3, i \in \N_4.
\end{eqnarray}
Therefore, the balance equation for the total torque is equivalent to
\begin{eqnarray}
\forall k \in \N_3: \sum_{i \in \N_4} \omega_{ki}(\xi_i, R) \cdot f_i = 0.
\end{eqnarray}

\begin{remark}
More explicitly, we have for any $k \in \N_3$ and $i \in \N_4$ that
\begin{equation}
	\omega_{ki}(\xi_i, R) = (\xi_0 + \xi_i) \sum_{j,l \in \N_3}(R z_i \cdot \hat{e}_j)\omega_{k}(\hat{e}_{j}, \hat{e}_{k})\hat{e}_l.
\end{equation}
In particular, we find for $i \in \N_4$
\begin{equation}
\begin{array}{lr}
	\omega_{1i}(\xi_i, R) = (\xi_0 + \xi_i) \left (
	0 ,
	-R z_i \cdot \hat{e}_3,
	R z_i \cdot \hat{e}_2
	\right )^T\\
	\omega_{2i}(\xi_i, R) = (\xi_0 + \xi_i) \left (
	R z_i \cdot \hat{e}_3 ,
	0,
	- R z_i \cdot \hat{e}_1
	\right )^T\\
	\omega_{3i}(\xi_i, R) = (\xi_0 + \xi_i) \left (
	- R z_i \cdot \hat{e}_2 ,
	R z_i \cdot \hat{e}_1,
	0
	\right )^T
\end{array}
\end{equation}
\end{remark}

Now, note that we have $G(b_{ij}) = G(b_{ji})$. So, let us define the matrices\footnote{We use modular arithmetic, i.e. $i + 1 = 1$ if $i = 4$.}
\begin{eqnarray}
 K_1 := G(b_{13}), &  K_2 := G(b_{24}), &  L_i := G(b_{i, i + 1}), i \in \N_4.
\end{eqnarray}
Furthermore, we define the vector of tractions $\boldsymbol f := (f_1, f_2, f_3, f_4)^T$ and the vector of velocities $\mathbf{u} := (u_1, u_2, u_3, u_4)^T$ in $\R^{12}$ as well as the matrices $\mathcal{I} := \diag(I, I, I, I) \in M_{12 \times 12}(\R)$
\begin{align}
\mathcal{L} := \left (\begin{array}{cccc}
0 & L_1 & K_1 & L_4 \\ 
L_1 & 0 & L_2 & K_2 \\ 
K_1 & L_2 & 0 & L_3 \\ 
L_4 & K_2 & L_3 & 0
\end{array}  \right ),
\end{align}
the \emph{mutual interaction matrix} where every block $L_i$ or $K_i$ describes the coupling of the two spheres corresponding to the indices. Besides, $\nu$ is the drag force exerted on a spherical object of radius $a$ immersed in a viscous fluid at very small Reynolds numbers. Finally, setting $\nu := 6 \pi \mu a$ allows us to write (\ref{eq:velocity_field_uniform}) in the purely algebraic form
\begin{equation}
	\mathbf{u} = (\frac{1}{\nu} \mathcal{I} + \mathcal{L}) \boldsymbol f.
\end{equation}
In particular, we then find the following result:

\begin{proposition}
\label{prop:control_system_laa}
In the limit of very long arms, and at the leading order, the swimming problem for \textsc{SPr4} at a point $p = (c, I) \in \mathcal{P}$ reduces to
\begin{eqnarray}
\label{eq:control_system_laa}
	\dot{p} = F(I, \xi) \dot{\xi}, & &  F(I, \xi) = - \frac{\mathcal{V}_{\nu, I}( \xi) \mathcal{X}_0}{\mathcal{V}_{\nu}(\xi, I) \mathcal{Y}(\xi)},
\end{eqnarray}
with $\mathcal{V}_{\nu} = \mathcal{W} \mathcal{L}_{\nu}$, where $\mathcal{W}$ is the torque matrix given by
\begin{align}
\mathcal{W}(\xi, R) := \left (\begin{array}{cccc}
I_{3 \times 3} & I_{3 \times 3} & I_{3 \times 3} & I_{3 \times 3} \\ 
\hline
\omega_{11}^T(\xi_1, R) & \omega_{12}^T(\xi_2, R) & \omega_{13}^T(\xi_3, R) & \omega_{14}^T(\xi_4, R) \\ 
\omega_{21}^T(\xi_1, R) & \omega_{22}^T(\xi_2, R) & \omega_{23}^T(\xi_3, R) & \omega_{24}^T(\xi_4, R) \\ 
\omega_{31}^T(\xi_1, R) & \omega_{32}^T(\xi_2, R) & \omega_{33}^T(\xi_3, R) & \omega_{34}^T(\xi_4, R)
\end{array}  \right )_{6 \times 12},
\end{align}
with the $\omega_{ki}(\xi_i, R)$ from  (\ref{eq:balance_equations}), and the shape matrices at the reference orientation $\mathcal{X}_0$ and $\mathcal{Y}$ are defined as
\begin{eqnarray}
 \mathcal{X}_0 := \left ( \begin{array}{cccc}
 z_1 & 0_{3 \times 1} & 0_{3 \times 1} & 0_{3 \times 1} \\ 
 0_{3 \times 1} & z_2 & 0_{3 \times 1} & 0_{3 \times 1} \\ 
 0_{3 \times 1} & 0_{3 \times 1} & z_3 & 0_{3 \times 1} \\ 
 0_{3 \times 1} & 0_{3 \times 1} & 0_{3 \times 1} & z_4
 \end{array} \right )_{12 \times 4}, &  \mathcal{Y}(\xi) := \left ( \begin{array}{cc}
 I_{3 \times 3} & (\xi_0 + \xi_1)[z_1]^T_\times \\ 
 I_{3 \times 3} & (\xi_0 + \xi_2)[z_2]^T_\times \\ 
 I_{3 \times 3} & (\xi_0 + \xi_3)[z_3]^T_\times \\ 
 I_{3 \times 3} & (\xi_0 + \xi_4)[z_4]^T_\times
 \end{array} \right )_{12 \times 6},
\end{eqnarray}
where $[z_i]_{\times}$ denotes the $3 \times 3$ matrix which satisfies $[z_i]_{\times} \times \omega $ for any $\omega \in \R^3$.
\end{proposition}

\begin{proof}
First, we recast (\ref{eq:velocity}) in the form
\begin{equation}
\label{eq:velocities_recast1}
\begin{array}{lr}
	u_1 &= \dot{c} + \dot{\xi}_1 z_1 + (\xi_0 + \xi_1) [z_1]^T_{\times} \omega\\
	u_2 &= \dot{c} + \dot{\xi}_2 z_2 + (\xi_0 + \xi_2) [z_2]^T_{\times} \omega\\
	u_3 &= \dot{c} + \dot{\xi}_3 z_3 + (\xi_0 + \xi_3) [z_3]^T_{\times} \omega\\
	u_4 &= \dot{c} + \dot{\xi}_4 z_4 + (\xi_0 + \xi_4) [z_4]^T_{\times} \omega
\end{array}
\end{equation}
where $\omega \in T_{I}\SO(3) \simeq \R^3$ is the axial vector associated to $\dot{I}$. Note that in the notation of section \ref{sec: symmetries}, we have exactly $\omega = [\dot{I}]_{\mathcal{L}}$, so we may indeed rewrite (\ref{eq:velocities_recast1}) in terms of $\mathcal{X}_0$ and $\mathcal{Y}$ as
\begin{equation}
\label{eq:velocities_recast2}
	\mathbf{u} = \mathcal{X}_0 \dot{\xi} + \mathcal{Y}(\xi)\dot{p}.
\end{equation}
In the limit of very long arms, i.e. if $\xi_0 + \min_{i \in \N_4}$ is sufficiently large, we have $\lim_{n \to \infty}(- \nu \mathcal{L})^{n} = 0$. Now, it follows from the Neumann series theorem that
\begin{equation}
	(\frac{1}{\nu} \mathcal{I} + \mathcal{L})^{-1} = \nu(\mathcal{I} + \nu \mathcal{L})^{-1} = \nu \sum_{n = 0}^{\infty}(- \nu \mathcal{L})^{-1} = \nu \mathcal{I} - \nu^2 \mathcal{L} + \mathcal{O}(||\mathcal{L}||^2).
\end{equation}
Hence, we can express the force at the leading order as a linear operator on the velocity vector:
\begin{equation}
	\boldsymbol f(\mathbf{u}) = \nu \mathcal{L}_\nu \mathbf{u},
\end{equation}
where $\mathcal{L}_{\nu} := \mathcal{I} - \nu \mathcal{L}$. So, in combination with (\ref{eq:velocities_recast2}) we can relate the rate of change of the shape and position variables at $R = I$ to the force by
\begin{equation}
\label{eq:force_to_shape_and_position}
\boldsymbol f = \nu \mathcal{L}_{\nu}[\mathcal{X}_0 \dot{\xi} + \mathcal{Y}(\xi) \dot{p}].
\end{equation}
Now, let us exploit the balance equations (\ref{eq:balance_equations}). It follows immediately from the definition of the torque matrix $\mathcal{W}$ and the vectors $\omega_{ki}(\xi_i, R)$ that we must have at $R = I$
\begin{equation}
\boldsymbol 0 = \frac{1}{\nu} \mathcal{W}(\xi, I) \boldsymbol f = \mathcal{W}(\xi, I) \mathcal{L}_{\nu} \boldsymbol u = \mathcal{W} (\xi, I) \mathcal{L}_{\nu} \mathcal{X}_0 \dot{\xi} + \mathcal{W}(\xi, I) \mathcal{L}_{\nu} \mathcal{Y}(\xi) \dot{p}.
\end{equation}
Eventually, setting $\mathcal{V}_{\nu} := \mathcal{W}  \mathcal{L}_{\nu}$, we find $\mathcal{V}_{\nu}(\xi, I) \mathcal{Y}(\xi) \dot{p} = - \mathcal{V}_{\nu}(\xi, I) \mathcal{X}_0 \dot{\xi}$. The matrix $\mathcal{V}_{\nu}(\xi, I) \mathcal{Y}(\xi)$, being a composition of matrices of full rank, is invertible, which finishes the proof.
\end{proof}

\subsection{Optimal swimming in the long arms regime}
Let us return to the question of optimal swimming. In section \ref{sec: optimization}, we have adopted the notion of optimal swimming efficiency proposed by Lighthill \cite{Lighthill1952}, i.e. that energy minimizing strokes are those minimizing the kinetic energy dissipated during one stroke which realizes a prescribed net displacement $\delta p \in \R^3 \times \so(3)$. One can evaluate the kinetic energy dissipated due to a stroke $\xi: J \to \M$ by integrating the instantaneous power dissipated at $t \in J$ given by $\mathcal{P}(\boldsymbol u) = \frac{1}{\nu} \boldsymbol f(\boldsymbol u) \cdot \boldsymbol u$. Here, we observe that the instantaneous power must be independent of the orientation of the swimmer due to the rotational invariance of the Stokes equations. Thus, by combining equations (\ref{eq:control_system_laa}) from Proposition \ref{prop:control_system_laa} and (\ref{eq:force_to_shape_and_position}), we can write the quadratic form $\mathcal{P}$ as
\begin{equation}
\mathcal{P}(\boldsymbol u) = \mathfrak{g}(\xi, I) \dot{\xi} \cdot \dot{\xi},
\end{equation}
with
\begin{equation}
\label{eq:energy_matrix_laa}
\mathfrak{g}(\xi, I) := \mathcal{X}_0^T \left  [ \mathcal{I} - \mathcal{Y}(\xi)\frac{\mathcal{V}_{\nu}}{\mathcal{V}_{\nu} \mathcal{Y}(\xi)} \right ]^T \mathcal{L}_{\nu} \left  [ \mathcal{I} - \mathcal{Y}(\xi)\frac{\mathcal{V}_{\nu}}{\mathcal{V}_{\nu} \mathcal{Y}(\xi)} \right ] \mathcal{X}_0.
\end{equation}
Staying in the regime of small deviations from $\xi_0$ by $\xi$, we can write for the instantaneous power at $t \in J$ as $\mathcal{P}(\boldsymbol u(t)) = G \dot{\xi}(t) \cdot \dot{\xi}(t)$ with $G := \mathfrak{g}(0, I)$, such that the total kinetic energy dissipated due to a stroke $\xi: J \to \M$ is given by
\begin{equation}
	\mathcal{G}(\xi) := \int_{J} G \dot{\xi}(t) \cdot \dot{\xi}(t) \dd t.
\end{equation}
Thus, we have naturally returned to the setting in section \ref{sec: optimization} with the bonus of being able to quantify the missing parameters. Indeed, one can easily check that the matrix in (\ref{eq:energy_matrix_laa}) is symmetric, positive definite and of the special form
\begin{equation}
G = \left ( \begin{array}{cccc}
\kappa & h & h & h \\ 
h & \kappa & h & h \\ 
h & h & \kappa & h \\ 
h & h & h & \kappa
\end{array} \right ),
\end{equation}
with the two parameters $\kappa$ and $h$ now given in terms of the radius of the spheres $a$ and the initial length of the arms $\xi_0$ by
\begin{eqnarray}
\kappa(a, \xi_0) &= \frac{3}{256} \left ( 64 + \frac{25 \sqrt{6} a}{\xi_0} + \frac{12 a}{9 a - 2 \sqrt{6} \xi_0} \right ),\\
h(a, \xi_0)  &= \frac{1}{768} \left (60 + \frac{69 \sqrt{6} a}{\xi_0} - \frac{16 \xi_0}{3 \sqrt{6} a - 4 \xi_0}  \right ).
\end{eqnarray}
Recall that the zeroth order term $F_0$ of the control system in the limit of small strokes is characterized by a parameter $\mathfrak{a}$, cf. equation (\ref{eq:position_zeroth_order_term}), which we can now compute directly from Proposition \ref{prop:control_system_laa}:
\begin{equation}
\mathfrak{a}(a, \xi_0) = \frac{-5}{16} + \frac{\xi_0}{16 \xi_0 - 12 \sqrt{6} a}.
\end{equation}
In addition, we find for the eigenvalues of $G$
\begin{align}
	\mathfrak{g}_c(a, \xi_0) &= \frac{1}{192} \left  (129 + \frac{39 \sqrt{6} a}{\xi_0} + \frac{4}{3 \sqrt{6} a/\xi_0 - 4} + \frac{27 a}{9a - 2 \sqrt{6} \xi_0} \right )\\
	\mathfrak{g}_{\theta}(a, \xi_0) &= 1 + \frac{9}{8} \sqrt{\frac{3}{2}} \frac{a}{\xi_0}.
\end{align}


To determine the remaining parameters, we follow the same approach as for \textsc{SPr3} \cite{Alouges2020}: For any $i, j \in \N_4$ the $(ij)$-entry of the matrices $A_k$ and $B_k$ (cf. (\ref{eq: dynamics first approx}) is given by $\dot{\phi}_{e_i}(0)e_j \cdot f_k$ with $\phi_{e_i}(t) := F(t e_i)$. In view of the latter relation, let us recall the following identity: For smooth matrix valued Functions $\mathcal{A}, \mathcal{B}: \R \to \R^{N \times N}$ with $\mathcal{A}(0)$ invertible, it holds that
\begin{equation}
\partial_t[\mathcal{A}(t)^{-1} \mathcal{B}(t)]_{\mid t = 0} = \frac{\mathcal{I}}{\mathcal{A}(0)}\left ( \dot{\mathcal{B}}(0) - \dot{\mathcal{A}}(0) \frac{\mathcal{B}(0)}{\mathcal{A}(0)}\right ).
\end{equation}
Therefore, setting $\mathcal{A}(t) := - \mathcal{V}_{\nu}(t e_i)\mathcal{Y}(t e_i)$ and $\mathcal{B}(t) := \mathcal{V}_{\nu}(t e_i) \mathcal{X}_0$ yields
\begin{equation}
\dot{\phi}_{e_i}(0)e_j = \frac{\mathcal{I}}{\mathcal{V}_\nu (0, I) \mathcal{Y}(0)} \left ( \partial_t [\mathcal{V}_{\nu}(t e_i, I) \mathcal{Y}(t e_i)]_{\mid t = 0} \frac{\mathcal{V}_{\nu}(0, I)}{\mathcal{V}_{\nu}(0, I) \mathcal{Y}(0)} - \partial_t[\mathcal{V}_{\nu}(t e_i, I)]_{\mid t = 0}\right ) \mathcal{X}_0 e_j.
\end{equation}
One then finds using a symbolic computation program the following expressions of the remaining parameters in terms of $a$ and $\xi_0$:
\begin{align}
\alpha(a, \xi_0) &= -\frac{\sqrt{3} a}{8(3 \sqrt{6} a - 4 \xi_0)^2}\\
\beta(a, \xi_0) &= - \frac{3 a( 81 \sqrt{2} a - 16 \sqrt{3} \xi_0}{128 \sqrt{2} (3 \sqrt{6} a - 4 \xi_0)^2 \xi_0}\\
\lambda(a, \xi_0) &= \frac{9 a (27 a - 8 \sqrt{6} \xi_0)}{128 \xi_0 (3 \sqrt{6} a - 4 \xi_0)^2}\\
\delta(a, \xi_0) &= \frac{81 a^2 - 240 \sqrt{6} a \xi_0 + 256 \xi_0^2}{64 \xi_0^2(-27 \sqrt{6} a^2 - 252 a \xi_0 + 64 \sqrt{6} \xi_0^2)}.
\end{align}







\section{Conclusion and outlook}
%In summary, in sections \ref{sec: symmetries} and \ref{sec: linearization}, we have found closed expressions for the dynamics of the swimmer \textsc{SPr4} in the small stroke regime up to certain scalar parameters as well as explicit formulas to calculate the net displacement produced by a given control curve. 

Next, in section \ref{sec: optimization}, we introduced bivectors to rewrite the initial optimization problem. In doing so, we could reveal the underlying geometric structure of the optimization problem due to the properties of the space of bivectors of $\R^4$. More precisely, we have the case of the prescribed net displacement being a simple bivector and the general case, in which the prescribed net displacement is the sum of at most two orthogonal simple bivectors. In more concrete terms, the simple case corresponds to planar displacements with an additional rotation around an axis orthogonal to the plane of movement. An important situation where we face the general case are screw motions, e.g. $\delta p \mid \mid e_{12} + e_{34}$.

In the simple case, the optimal control curves could be characterized in a very similar way as for the parking 3-sphere swimmer \textsc{SPr3} in \cite{Alouges2017}. Indeed, the energy minimizing control curves turn out to be elliptical strokes in a plane of $\R^4$ defined by the prescribed net displacement, see Theorem \ref{thm:optimal control curves in the simple case}.

However, the search for the structure of the energy minimizing swimming strokes in the general case remains unsuccessful for the time being. Nonetheless, we claim that in the general case, the optimal control curves are contained in two completely orthogonal planes of $\R^4$. We have two reasons to believe so: On the one hand, one can write any bivector of $\R^4$ can be written as the sum of two orthogonal simple bivectors, c.f. \cite{Lounesto2006}. Thus, we expect there to be a way of generalizing the proof of Proposition \ref{prop:simple reduction}, which appears to be non-trivial so far. On the other hand, the Euler-Lagrange equation associated to the minimization problem (c.f. \cite{DeSimone2011}) reads
\begin{equation}
G \ddot{\xi} - \Omega(\mu) \dot{\xi} = 0,
\end{equation}
with $\mu \in \R^6$ and $\Omega(\mu) = \sum_{k = 1}^{6} \mu_k M_k$. The matrix $\Omega(\mu)$ being skew symmetric, its exponential is a rotation of $\R^4$ and therefore the solution is contained one single plane or in two completely orthogonal planes of $\R^4$.

\section{Acknowledgments}
At this point, I would like to thank my supervisors François Alouges and Aline Lefebvre-Lepot for their numerous comments, suggestions and corrections during my internship and for making this internship possible in the first place in spite of the difficult situation due to the COVID-19 pandemic. I would also like to offer a word of thanks to my co-intern Ward Haegeman for his keen eye for mistakes in my calculations and to Giovanni di Fratta for his help with the 3D rendering of the figures.

\newpage
\section{Appendix}
\subsection{Complete computations of the first order terms}
 
 In analogy to the calculation of the matrices $M_k$, we define $\mathcal{T}_{0, c}(\xi \otimes \eta) := \tfrac{1}{2}  [\mathcal{H}_{c,0}(\xi \otimes \eta) + \mathcal{H}_{c,0}(\eta \otimes \xi)]$ and similarly $\mathcal{T}_{0, \theta}(\xi \otimes \eta)$ for $\xi, \eta \in \R^4$ such that $\T_{0,c}(\xi \otimes \eta) = \T_{0,c}(\eta \otimes \xi)$ as well as $\T_{0,\theta}(\xi \otimes \eta) = \T_{0,\theta}(\eta \otimes \xi)$. Clearly, $\T_{c,0}$ and $\T_{\theta,0}$ satisfy the same symmetry relations as $\h_{c,0}$ and $\h_{\theta,0}$ and thus we find for any permutation matrix $P_{ij}$ and the corresponding reflection $S_{ij}$ that
\begin{align}
\label{eq:even_position_first_order_condition}
 S_{ij}\T_{c,0}(P_{ij} \xi \otimes P_{ij} \eta) &= \T_{c,0}(\xi \otimes \eta),\\
 \label{eq:even_angle_first_order_condition}
  -S_{ij}\T_{\theta,0}(P_{ij} \xi \otimes P_{ij} \eta) &= \T_{\theta,0}(\xi \otimes \eta).
\end{align}
We treat the spatial part first. Since the matrix $S_{ij}$ represents the reflection at the plane spanned by the remaining two arms $z_k$ and $z_l$, equation (\ref{eq:even_position_first_order_condition}) implies that $\T_{c,0}(e_i \otimes e_j) \in \Span\{z_k, z_l\}$. Next, we find, using again (\ref{eq:even_position_first_order_condition}) and the fact that the reflection $S_{kl}$ is an orthogonal transformation, that
\begin{equation}
\T_{c,0}(e_i \otimes e_j) \cdot z_k = S_{kl}\T_{c,0}(e_i \otimes e_j) \cdot S_{kl} z_k = \T_{c,0}(e_i \otimes e_j) \cdot z_l,
\end{equation}
from which we deduce that $\T_{c,0}(e_i \otimes e_j) = \beta_{ij} (z_k + z_l)$ for some scalar $\beta_{ij} \in \R$. However, the same holds for $\T_{c,0}(e_i \otimes e_k)$ and we find
\begin{align}
\beta_{ik} (z_j + z_l) = \T_{c,0}(e_i \otimes e_k) = S_{jk} \T_{c,0}(e_i \otimes e_j) = S_{jk} \beta_{ij} (z_k + z_l) = \beta_{ij} (z_j + z_l).
\end{align}
Since the vectors $z_1, z_2, z_3$ and $z_4$ are normalized and enclose pairwise the same angle, we can conclude that $\beta_{ik} = \beta_{ij}$ or more generally that $\T_{c,0}(e_i \otimes e_j) = \beta (z_k + z_l)$ for all $i \neq j \in \N_4$. Furthermore, for the term $\T_{c,0}(e_i \otimes e_i)$ we find in a similar fashion that $\T_{c,0}(e_i \otimes e_i) \in \Span\{z_i, z_j\}$, $\T_{c,0}(e_i \otimes e_i) \in \Span\{z_i, z_k\}$ and $\T_{c,0}(e_i \otimes e_i) \in \Span\{z_i, z_l\}$. By noting that the line of intersection of these three planes is $\{\lambda z_i | \lambda \in\R\}$, we obtain $\T_{c,0}(e_i \otimes e_i) = \lambda_i z_i$ for some $\lambda_i \in \R$. Again by using the orthogonality of the reflections $Sij$, we find that
\begin{equation}
\lambda_j = \T_{c,0}(e_j \otimes e_j) \cdot z_j = S_{ij} \T_{c,0}(e_j \otimes e_j) \cdot S_{ij} z_j = \T_{c,0}(e_i \otimes e_i) \cdot z_i = \lambda_i.
\end{equation}
Hence, we have $\T_{c,0}(e_i\otimes e_i) = \lambda z_i$ for all $i\in \N_4$ and some $\lambda \in \R$.

For the rotational part, we observe that equation (\ref{eq:even_angle_first_order_condition}) implies that on the one hand we have $\T_{\theta,0}(e_i \otimes e_j) = - S_{ij} \T_{\theta,0}(e_i \otimes e_j)$ and on the other hand that $\T_{\theta,0}(e_i \otimes e_j) = -S_{kl} \T_{\theta,0}(e_i \otimes e_j)$. However, the first equation implies that $\T_{\theta,0}(e_i \otimes e_j)$ is proportional to $z_k \times z_l$, while the second implies that $\T_{\theta,0}(e_i \otimes e_j)$ is proportional to $z_i \times z_j$. Yet, from this we conclude that necessarily $\T_{\theta,0}(e_i \otimes e_j) = 0$ since $z_k \times z_l \in \Span\{z_i, z_j\}$ and vice-versa. The argument for $\T_{\theta,0}(e_i \otimes e_i)  = 0$ is very similar and thus omitted.

In conclusion, the matrices $N_k :=( \T_{c,0}(e_i \otimes e_j)\cdot  \hat{e}_k)_{i,j \in \N_4}  = \tfrac{1}{2}(A_k + A_k^T), k \in \N_3$ and $N_{k + 3} := ( \T_{\theta,0}(e_i \otimes e_j)\cdot  \hat{e}_k)_{i,j \in \N_4}  = \tfrac{1}{2}(B_k + B_k^T), k \in \N_3$ are given by


\begin{align}
 N_1 = \frac{\sqrt{2}}{3}\left(
\begin{array}{cccc}
 2 \lambda & -\beta & -\beta & -2 \beta \\
 -\beta & -\lambda & 2 \beta & \beta \\
 -\beta & 2 \beta & -\lambda & \beta \\
 -2 \beta & \beta & \beta & 0 \\
\end{array}
\right),
\end{align}

\begin{eqnarray}
N_2 =\sqrt{\frac{2}{3}} \left(
\begin{array}{cccc}
 0 & \beta & -\beta & 0 \\
 \beta & -\lambda & 0 & \beta \\
 -\beta & 0 & \lambda & -\beta \\
 0 & \beta & -\beta & 0 \\
\end{array}
\right), & &
 N_3 = \frac{1}{3}\left(
\begin{array}{cccc}
 -\lambda & 2 \beta & 2 \beta & -2 \beta \\
 2 \beta & -\lambda & 2 \beta & -2 \beta \\
 2 \beta & 2 \beta & -\lambda & -2 \beta \\
 -2 \beta & -2 \beta & -2 \beta & 3 \lambda \\
\end{array}
\right)
\end{eqnarray}
\renewcommand{\arraystretch}{1}
\begin{align}
N_4 = N_5 = N_6 = 0.
\end{align}


\subsection{Decomposition of bivectors}
\label{subsec:decomposition}
For the implementation of an algorithm that constructs optimal control curves to a given net displacement, the decomposition of bivectors is crucial; that is, given a bivector $\omega \in \bigwedge^2 \R^4$, we want to find vectors $u_1, v_1, u_2, v_2 \in \R^4$ such that $\omega = u_1 \wedge v_1 + u_2 \wedge v_2$. Moreover, if we are in the case of a simple bivector, we even want to find $u, v \in \R^4$ such that $\omega = u \wedge v$. Fortunately, Lemma \ref{lem:simple bivector} gives us already an efficient way to check whether a bivector is simple or not. In fact, the decomposition will be largely based on its proof, which first decomposes $\omega$ into the sum of two simple bivectors, see \cite{Hitchin2003}.

So, let us first decompose any $\omega \in \bigwedge^2 \R^4$ into the sum of two simple bivectors. To that end, let $\omega = a_{14}e_{14} +  a_{24}e_{24} +  a_{34}e_{34} +  a_{23}e_{23} +  a_{31}e_{31} +  a_{12}e_{12}$ be a bivector expressed in the basis related to the problem, cf. (\ref{eq: basis of bivectors}). Then we can rewrite
\begin{equation}
    \omega = u \wedge e_4 + (a_{23} e_2 - a_{31}e_1) \wedge e_3 + a_{12} e_{12},
\end{equation}
where $u := ( a_{14}e_{1} +  a_{24}e_{2} +  a_{34}e_{3})$. If $a_{31} = 0$, we immediately find the decomposition
\begin{equation}
    \omega = u \wedge e_4 + (a_{12} e_1 - a_{23} e_3) \wedge e_2.
\end{equation}
Otherwise, set $\alpha := - \tfrac{a_{12}}{a_{31}} a_{23}$. Then, we have
\begin{equation}
    a_{12} e_{12} = (\alpha e_2 + a_{12} e_1) \wedge e_2 = - \frac{a_{12}}{a_{31}} (a_{23} e_2 - a_{31} e_1) \wedge e_2.
\end{equation}
Hence, we may rewrite $\omega$ as
\begin{equation}
    \omega = u \wedge e_4 + (a_{23} e_2 - a_{31} e_1) \wedge \left(e_3 - \frac{a_{12}}{a_{31}}e_2 \right), 
\end{equation}
which yields the desired decomposition.

To construct a decomposition in the simple case, we first have to know how to decompose any bivector $\omega \in \bigwedge^2 V$, where $V$ is a vector space of dimension 3, into a simple bivector. Let $(u_1, u_2, u_3)$ be an ordered basis of $V$. Then, we can write $\omega \in \bigwedge^2 V$ as $\omega = \lambda_{12} u_1 \wedge u_2 + \lambda_{13} u_1 \wedge u_3 + \lambda_{23} u_2 \wedge u_3$. If $\lambda_{13} = 0$, we immediately have $\omega = (\lambda_{12} u_1 - \lambda_{13} u_3) \wedge u_2$. Otherwise, we have $\lambda_{23} u_2 \wedge u_3 = \frac{\lambda_{23}}{\lambda_{13}} u_2 \wedge (\lambda_{12} u_2 + \lambda_{13} u_3)$ and thus
\begin{equation}
    \omega =\left (u_1 + \frac{\lambda_{23}}{\lambda_{13}} u_2 \right) \wedge \left(\lambda_{12} u_2 + \lambda_{13} u_3 \right).
\end{equation}
So, if we already know that $\omega \wedge \omega = 0$ for $\omega \in \bigwedge^2 \R^4$, we decompose it according to the first paragraph into $\omega = u \wedge e_4 + v_1 \wedge v_2$. Note that the condition $\omega \wedge \omega = 0$ implies that $u \wedge v_1 \wedge v_2 = 0$ and thus the latter three vectors must be linearly dependent. If $v_1$ and $v_2$ are linearly dependent, then $v_1 \wedge v_2 = 0$ and we are done. Otherwise, write $u = \lambda_1 v_1 + \lambda_2 v_2$. Then, we have
\begin{equation}
    \omega = \lambda_1 (v_1 \wedge e_4) + \lambda_2(v_2 \wedge e_4) + v_1 \wedge v_2.
\end{equation}
Note that due to the definition of $v_1$ and $v_2$ in the first paragraph, the vectors $v_1, v_2$, and $e_4$ are linearly independent and thus we can apply the decomposition algorithm for three-dimensional vector spaces.


\printbibliography[title = Bibliography]



\end{document}