\documentclass[10pt,a4paper]{article}
\usepackage[utf8]{inputenc}
\usepackage[french]{babel}
\usepackage{amsmath}
\usepackage{amsthm}
\usepackage{amsfonts}
\usepackage{amssymb}
\usepackage{graphicx}
\usepackage[a4paper,top=25mm,bottom=25mm]{geometry}
\author{Philipp Weder}
\title{Soutenance de stage - dispotition détaillée}





% packages for layout
\usepackage{fancyhdr}
\pagestyle{fancy}
\fancyhf{}
\fancyhead[L]{\textit{\nouppercase{\leftmark}}}
\fancyhead[R]{\thepage}

\renewcommand{\headrulewidth}{0.5pt}


%roman enumeration
\renewcommand\labelenumi{(\roman{enumi})}
\renewcommand\theenumi\labelenumi

% font
%\usepackage{pxfonts}

% bibliography
\usepackage[style=ieee, sorting = nty, backend = biber]{biblatex}
\bibliography{/Users/philipp/Documents/GitHub/stage_cmap/tex/Report/report.bib}
\usepackage{csquotes}

% environments
% theorem
\theoremstyle{plain}
\newtheorem{theorem}{Theorem}[section]
% corollary
\theoremstyle{plain}
\newtheorem{corollary}{Corollary}[theorem]
% lemma
\theoremstyle{plain}
\newtheorem{lemma}[theorem]{Lemma}
% remark
\theoremstyle{definition}
\newtheorem*{remark}{Remark}
% definition
\theoremstyle{definition}
\newtheorem{definition}{Definition}[section]
% example
\theoremstyle{definition}
\newtheorem{example}{Example}[section]
% proposition
\theoremstyle{plain}
\newtheorem{proposition}{Proposition}[section]

% additional packages
\usepackage{appendix}
\usepackage{amsmath}
\usepackage{amsfonts}
\usepackage{amssymb}
\usepackage{amsthm}
\usepackage{booktabs}

\newcommand{\N}{\mathbb{N}}
\newcommand{\M}{\mathcal{M}}
\newcommand{\R}{\mathbb{R}}
\newcommand{\h}{\mathcal{H}}
\newcommand{\K}{\mathcal{K}}
\DeclareMathOperator{\Skew}{Skew}
\DeclareMathOperator{\id}{id}
\newcommand{\so}{\mathfrak{so}}
\newcommand{\REF}{\mathrm{ref}}
\newcommand{\spr}{\textsc{SPr4}}
\DeclareMathOperator{\dist}{dist}
\DeclareMathOperator{\SO}{SO}
\DeclareMathOperator{\sgn}{sgn}
\DeclareMathOperator{\Aut}{Aut}
\DeclareMathOperator{\diag}{diag}
\newcommand{\chroexp}{\overset{\longrightarrow}{\exp}}
\DeclareMathOperator{\re}{Re}
\DeclareMathOperator{\Span}{span}
\newcommand{\dd}[1]{\mathrm{d}#1}
\DeclareMathOperator{\ad}{ad}
\newcommand{\T}{\mathcal{T}}

\begin{document}
\renewcommand\labelitemi{--}
\linespread{1.05}
\maketitle
\section{Mot de bienvenue}

\section{Introduction}


\section{Structure du projet}
La structure du reste de la présentation est comme ce qui suit:
\begin{enumerate}
\item Tout d'abord, je vais vous présenter le modèle mathématique qui a été introduit dans \cite{Alouges2013} et je vais vous montrer les propriétés de symétrie satisfaites par le système de contrôle en question.

\item Ensuite, je vais introduire l'hypothèse des courbes de contrôle petites et je vais présenter les implications qui en résultent pour notre système de contrôle. 

\item Puis, je parlerai du problème d'optimisation associé dont on verra la solution dans un cas particulier.

\item Finalement, j'aborderai les perspectives sur le sujet et je présenterai la conjecture sur le cas général, qu'on a faite à la fin du projet.
\end{enumerate}


\section{Modélisation et symétries}
\subsection{Notation et modèle}
\begin{itemize}

\item Pour le modèle du nageur vu avant, on se donne un tétraèdre de référence avec sommets $(S_1, S_2, S_3, S_4)$ centré à $c \in \R^4$ tel que $\dist(c, S_i) = 1$.

\item Alors le nageur \textsc{SPr4} consiste de quatre boules $B_i$ centrées à $b_i$ de rayon $a > 0$ telles que la boule $B_i$ peut bouger le long de la demi-droite d'origine $c$ passant par $S_i$.

\item Donc, on est dans la situation où les quatre boules sont reliées à $c$ par des bras très fin, qui peuvent s'allonger et se rétracter. Or, on néglige la résistance visqueuse des bras. De plus, il n'y a aucune restriction pour l'orientation du nageur dans le fluide, c'est-à-dire à longueurs de bras fixés, le nageur est juste un corps rigide dans un fluide Stokesien.

\item La configuration géométrique est complètement décrite par deux ensembles de variables:
\begin{enumerate}
\item \emph{Les variables de forme:} $\xi := (\xi_1, \xi_2, \xi_3, \xi_4) \in \M := (\sqrt{3/2}a, + \infty)^4$, où les $\xi_i$ sont les longueurs des bras.

\item \emph{Les variables de position:} $p = (c, R) \in \mathcal{P} := \R^3 \times \SO(3)$.
\end{enumerate}
De plus, on pose $z_i := \overline{S_i c}$. En effet, ceci est une description complète du nageur, car si $r \in B_a$, la boule de rayon $a$ centrée à l'origine, le point actuel sur la boule $B_i$ est donné par
\begin{equation}
	r_i(\xi, p, r) :=  c + R(\xi_i z_i + r).
\end{equation}
Les fonctions $r_i$ sont analytique, par conséquent on peut en déduire la vitesse instantanée sur la boule $B_i$:
\begin{equation}
	u_i(\xi, p, r) = \dot{c} + \omega \times (\xi_i z_i + r) + R z_i \dot{\xi}_i,
\end{equation}
où $\omega$ est le vecteur axial associé à la matrice anti-symétrique $\dot{R}R$.

\item Dans \cite{Alouges2013}, il a été montré comment $p$ change si on varie $\xi$. Plus précisément, on a le système dynamique suivant:
\begin{equation}
\label{eq: control system}
	\dot{p} = F(R, \xi) \dot{\xi} := \left ( \begin{array}{c}
	F_c(R, \xi) \\
	F_\theta(R, \xi)
	\end{array}  \right ) \dot{\xi},
\end{equation}
tel que $\dot{c} = F_{c}(R, \xi) \dot{\xi}$ et $\dot{R} = R_{\theta}(R, \xi) \dot{\xi}$.

\item On remarque qu'on a
\begin{equation}
\begin{aligned}
	F_c(R, \xi) \in \mathcal{L}(\R^4, \R^3) \text{ et } F_{\theta}(R, \xi) \in \mathcal{L}(\R^4,T_R \SO(3)).
\end{aligned}
\end{equation}
De plus, on sait que pour $R \in \SO(3)$ fixé, on a
\begin{align}
T_{R}SO(3) = \{R M \mid M \in \Skew_3(\R)\}.
\end{align}
En particulier, on a $\dim T_R \SO(3) = 3$. Ainsi, dès qu'on a choisi une base pour les espaces tangents correspondants, on peut exprimer $F_{c}(R, \xi)$ et $F_{\theta}(R, \xi)$ comme des matrices de taille $4 \times 3$ pour $R$ et $\xi$ fixes.

\end{itemize}

\subsection{Symétries}
Passons à l'investigation du système de contrôle \ref{eq: control system}. Pour ceci, choisissons une condition initiale $p_0 = (c_0, R_0) \in \mathcal{P}$ et une courbe de contrôle $\xi: J \subset \R \to \M$, avec $J$ un voisinage de zéro. Puis, notons $\gamma(c_0, R_0, \xi): I \to \mathcal{P}$ la solution associée au système dynamique
\begin{equation}
\label{eq:dynamical system}
\begin{aligned}
	&\dot{p} = F(R, \xi) \dot{\xi},& & p(0) := p_0.
\end{aligned}
\end{equation}
On notera par $\gamma_c(c_0, R_0, \xi)$ et $\gamma_{\theta}(c_0, R_0, \xi)$ les projections à $\R^3$ et $\SO(3)$, respectivement. En particulier, on a par définition
\begin{equation}
	\dot{\gamma}(c_0, R_0, \xi)(t) = F(\gamma_\theta(c_0, R_0, \xi)(t), \xi(t))\dot{\xi}(t), \forall t \in J.
\end{equation}

\subsubsection{Invariance rotationelle}
\begin{itemize}

\item Les équations de Stokes sont invariant sous rotations, c'est-à-dire si on tourne tout le domaine géométrique, la solution se transforme de la même façon.

\item Ceci implique pour la solution de notre système dynamique que
\begin{equation}
\label{eq:spatial rotational invariance}
	\gamma_c(c_0, R R_0, \xi)(t) = R \gamma_c (c_0, R_0, \xi)(t) + (I - R) c_0, \forall t \in J
\end{equation}
et
\begin{equation}
\label{eq: angular rotational invariance}
	\gamma_\theta(c_0, R R_0, \xi)(t) =  R \gamma_\theta(c_0, R_0, \xi)(t), \forall t \in J
\end{equation}
Ensuite, on trouve avec un calcul la propriété suivante du champs de vecteur $F$:
\begin{proposition}
\label{prop: rotational invariance}
Soit $\xi_0 := \xi(0) \in \M$ le point initiale de la courbe de contrôle et notons $T_{\xi}\M$ l'espace tangent à $\xi$. Si le système de contrôle \ref{eq: control system} est invariant sous rotations et si pour tout $\xi \in \M$ on a $T_{\xi}\M \simeq \R^4$, alors
\begin{equation}
	F_c(R, \xi) = R F_c(\xi) \text { and } F_\theta(R, \xi) = R F_{\theta} (R, \xi), \forall (R, \xi) \in \SO(3) \times \M,
\end{equation}
où $F_c(\xi) := F_{c}(I, \xi)$ et $F_{\theta}(\xi) := F_{\theta}(I, \xi)$
\end{proposition}

\item Ceci signifie juste qu'on peut toujours factoriser l'orientation de $F$.

\end{itemize}

\subsubsection{Permutation des bras}
\begin{itemize}
\item Notons $P_{ij} \in Mat_{4 \times 4}(\R)$ la matrice de permutation qui échange les indices $i$ et $j$ d'un vecteur. Cette transformation appliquée à l'espace de contrôles $\M$ signifie la permutation des bras $i||$ et $j||$.

\item Soit $S_{ij}$ la réflexion qui envoi $i ||$ sur $j||$ dans $\mathcal{P}$. Remarquons que la réflexion $S_{ij}$ se passe au plans qui passe par les bras $k||$ et $l||$.

\item Les équations sont invariantes sous changement de point de vue, ce qui implique pour notre solution que pour la position initiale $p_0 := (c, I)$ on a
\begin{equation}
\label{eq:spatial_perm_inv}
	\gamma_c(c_0, I, P_{ij} \xi) = S_{ij} \gamma_c(S_{ij}c_0, I, \xi)
\end{equation}
et
\begin{equation}
\label{eq:angular_perm_inv}
	\gamma_{\theta}(c_0, I, P_{ij} \xi ) = S_{ij} \gamma_{\theta} (S_{ij} c_0, I, \xi) S_{ij}.
\end{equation}

\item À l'aide d'un calcul un peu plus technique, on trouve le résultat suivant:

\begin{proposition}
\label{prop: spatial permutation invariance}
Si le système de contrôle (\ref{eq: control system}) satisfait les équations (\ref{eq:spatial_perm_inv}) et (\ref{eq:angular_perm_inv}) et $T_{\xi} \M \simeq \R^4$ pour tout $\xi \in \M$, alors
\begin{align}
	 F_c(P_{ij} \xi) = S_{ij} F_c(\xi) P_{ij} \text{ et } F_{\theta}(P_{ij} \xi) = - S_{ij} F_{\theta}(\xi) P_{ij}. \forall \xi \in \M.
\end{align}
\end{proposition}

\item Attention: Il faut toujours choisir la base canonique $\mathcal{E} = (e_1, e_2,e_3, e_4$ pour $\R^4$ et la base $\mathcal{L} = (L_1, L_2, L_3)$ avec
\begin{align}
\label{eq: L1}
	&L_1 = \frac{\dd}{\dd\theta}R_1(\theta)_{\mid \theta =0} = \left(\begin{array}{ccc}
	0 & 0 & 0 \\ 
	0 & 0 & -1 \\ 
	0 & 1 & 0
	\end{array}  \right ),\\
\label{eq: L2}
	&L_2 = \frac{\dd}{\dd\theta}R_2(\theta)_{\mid \theta =0} = \left (\begin{array}{ccc}
	0 & 0 & 1 \\ 
	0 & 0 & 0 \\ 
	-1 & 0 & 0
	\end{array}  \right ),\\
\label{eq: L3}
	&L_3 = \frac{\dd}{\dd\theta}R_3(\theta)_{\mid \theta =0} = \left (\begin{array}{ccc}
	0 & -1 & 0 \\ 
	1 & 0 & 0 \\ 
	0 & 0 & 0
	\end{array}  \right ),
\end{align}
pour justifier la notation dans l'équation à droite dans la proposition.
\end{itemize}

\section{Régime des petites courbes de contrôle}
\begin{itemize}
\item Maintenant, on a envie d'exploiter les propriétés de symétrie de $F$ de la partie précédente. Pour faire ceci, attaquons $F$ du côté analytique. 

\item On part de la factorisation
\begin{equation}
\label{eq: reminder control system}
	F_{c}(R, \zeta) = R F_{c}(\zeta) \text{ et } F_{\theta}(R, \xi) = R F_{\theta}(\zeta), \forall R \in \SO(3),
\end{equation}
où $F_{c}(\zeta) := F_c(I, \zeta)$ et $F_{\theta}(\zeta) := F_{\theta}(I, \zeta)$. Puis, supposons que $\zeta = \xi_0 + \xi$, où $\xi_0$ a toutes les composoantes égales. Finalement, posons $F_{c, \xi_0}(\xi) := F_{c}(\xi_0 + \xi)$ et $F_{\theta, \xi_0}(\xi) := F_{\theta}(\xi_0 + \xi)$.

\item Il a été démontré dans \cite{Alouges2013} que $F$ et donc $F_{c, \xi_0}$ et $F_{\theta, \xi_0}$ sont analytiques. Par conséquents, nous avons le droit de faire le développement limité suivant:
\begin{align}
\label{eq: spatial control expansion}
	F_{c, \xi_0}(\xi)\eta &= F_{c, 0} \eta  + \h_{c,0}(\xi \otimes \eta) + \mathcal{O}(|\xi|)\eta\\
\label{eq: angular control expansion}
	F_{\theta, \xi_0}(\xi) \eta &= F_{\theta, 0} \eta + \h_{c, 0}(\xi \otimes \eta) + \mathcal{O}(|\xi|)\eta,
\end{align}
où $F_{c,0} := F_{c}(\xi_0) \in M_{3 \times 4}(\R)$, $\h_{c,0}\in \mathcal{L}(\R^4 \otimes \R^4, \R^3)$ représent la dérivée d'ordre 1 de $F_{c, \xi_0}$ à $\xi = 0$. $F_{\theta, \xi}$ est défini de façon analogue.

\item Pour les termes d'ordre zéro on peut montrer que
\begin{align}
\label{eq:zeroth_order_sym}
	F_{c,0} &= S_{ij} F_{c,0} P_{ij}\\
	F_{\theta, 0} &= -S_{ij} F_{\theta, 0} P_{ij}
\end{align}
ainsi que pour les termes d'ordre un que
\begin{align}
\label{eq:first_order_sym}
\h_{c,0}(P_{ij} \xi\otimes \eta) &= S_{ij} \h_{c,0}(\xi \otimes P_{ij} \eta), &\forall \xi, \eta \in \R^4\\
\h_{\theta,0}(P_{ij} \xi \otimes \eta) &= -S_{ij} \h_{\theta,0}(\xi \otimes P_{ij} \eta), &\forall \xi, \eta \in \R^4
\end{align}
\end{itemize}

\subsection{Termes d'ordre zéro}
\begin{itemize}
\item Un calcul élémentaire, qui n'utilise que (\ref{eq:zeroth_order_sym}) et les propriétés des $S_{ij}$ montre que
\begin{equation}
	F_{c,0} = \mathfrak{a} (z_1 |z_2| z_3|z_4),
\end{equation}
avec $\mathfrak{a} \in \R$ ou bien
\begin{equation}
	F_{c,0} = - 3 \sqrt{3} \mathfrak{a} [\tau_1| \tau_2 |\tau_3]^T,
\end{equation}
où $\tau_1 : = \tfrac{1}{\sqrt{6}}(-2,1,1,0)^T$, $\tau_2 := \tfrac{1}{\sqrt{2}}(0,1,-1,0)^T$, $\tau_{3}:= \tfrac{1}{2 \sqrt{3}} (1,1,1,-3)^T$ forment une base orthonormale ensemble avec $\tau_4:= \tfrac{1}{2}(1,1,1,1)^T$. Cette base orthonormale sera utile plus tard.
\end{itemize}

\subsection{Termes d'ordre un}
\begin{itemize}
\item Pour déterminer les termes d'ordre un, on a suivi l'approche dans \cite{Alouges2017}, où on a évalué les tenseurs sur la base canonique et où on a posé
\begin{align}
A_k &:= (\h_{c,0}(e_i \otimes e_j)\cdot \hat{e}_k)_{i,j \in \N_4}, k \in \N_3\\
B_k &:= (\h_{\theta,0}(e_i \otimes e_j)\cdot \hat{e}_k)_{i,j \in \N_4}, k \in \N_3.
\end{align}
Ainsi, on a pour tout $\xi, \eta \in \R^4$
\begin{align}
\label{eq: matrix representation of spatial first order term}
	\h_{c,0}(\xi \otimes \eta) = \sum_{k \in \N_3}(A_k \eta \cdot \xi)\hat{e}_k, \\
\label{eq: matrix representation of angular first order term}
	\h_{\theta, 0}(\xi \otimes \eta) = \sum_{k \in \N_3}(B_k \eta \cdot \xi) L_k.
\end{align}

\item À lieu de calculer directement les matrices $A_k$ et $B_k$, on a calculé leurs parties symétriques et anti-symétriques en utilisant des arguments de symétrie. En fait, seulement les parties anti-symétriques seront importantes pour la suite. Notons-les
\begin{align}
M_k &:= \frac{1}{2}[A_k - A_k^T], k \in \N_3\\
M_{k + 3} &:= \frac{1}{2}[B_k - B_k^T], k \in \N_3.
\end{align}

\item Finalement, il ne reste que 5 paramètres inconnus dans les matrices $A_k$ et $B_k$.
\end{itemize}

\subsection{Linéarisation}
\begin{itemize}
\item Maintenant, on se donne un espace pour les courbes, notamment $H^1_{\sharp}(J, \R^4)$, où $J := [0, 2\pi]$.

\item On notera $\langle f \rangle := (2 \pi)^{-1 }\int_{J} f(t) d t$ pour $f \in H^1_{\sharp}(J, \R^4)$.

\item Reprenons le système dynamique. Initialement, pour $\zeta \in \M$, le système dynamique s'écrivait
\begin{align}
\begin{cases}
	\dot{c} &= R F_c(\zeta) \dot{\zeta}\\
	\dot{R} &= R F_{\theta}(\zeta) \dot{\zeta}.
\end{cases}
\end{align}
Puis dans la partie précédente, on a posé $\zeta = \xi_0 + \xi$ et un développement limité autour de $\xi = 0$ nous a fournit le système simplifié
 \begin{align}
 \label{eq: dynamics first approx}
 \begin{cases}
 	\dot{c} &= R F_{c,0} \dot{\xi} + R \sum_{k \in \N_3}(A_k \dot{\xi} \cdot \xi)\hat{e_k}\\
 	\dot{R} &= R \sum_{k \in \N_3} (B_k \dot{\xi} \cdot \xi) L_k.
 \end{cases}
 \end{align}
 
\item Définissons les déplacements nets
\begin{align}
	&\delta c: H_{\sharp}^1(J,\R^4) \to \R^3\\
	&\xi \mapsto 2\pi \langle \dot{c} (\xi) \rangle \nonumber \\
	&\delta p: H_{\sharp}^1(J,\R^4) \to \so(3)\\
	&\xi \mapsto 2 \pi \langle \dot{R}(\xi) \rangle \nonumber
\end{align}
Il est impossible d'évaluer ses expressions exactement, mais un argument du calcul chronologique nous a permis de linéariser le système simplifié autour de $R_0 = I$. Ainsi, on est arrivé à établir le résultat suivant:
\begin{proposition}
\label{prop:net displacement}
Pour tout $\xi \in H_\sharp^1(J, \R^4)$, dans un voisinage de $0 \in H_{\sharp}^{1}(J, \R^4)$, on a les estimes suivants
\begin{equation}
\begin{aligned}
\delta c(\xi) &= 2 \pi \sum_{k \in \N_3} \langle A_k \dot{\xi} \cdot \xi \rangle \hat{e}_k + \mathcal{O}(||\xi||_{H^1_{\sharp}}^3),\\
\delta R(\xi) &= 2 \pi \sum_{k \in \N_3} \langle B_k \dot{\xi} \cdot \xi \rangle L_k + \mathcal{O}(||\xi||^4_{H_\sharp^1}).
\end{aligned}
\end{equation}
\end{proposition}

\item Si $A$ est une matrice symétrique, on trouve par intégration par parties que $\langle A \xi \cdot \dot{\xi} \rangle = - \langle A \xi \cdot \dot{\xi} \rangle = 0$ et donc
\begin{align}
\langle A_k \dot{\xi} \cdot \xi \rangle &= \langle M_k \dot{\xi} \cdot \xi \rangle, &\forall k \in \N_3\\
\langle B_k \dot{\xi} \cdot \xi \rangle &= \langle M_{k + 3} \dot{\xi} \cdot \xi \rangle, &\forall k \in \N_3
\end{align}

\item Par un calcul direct, on trouve
\begin{align}
  M_k \dot{\xi} \cdot \xi &= - 2 \sqrt{6} \,\alpha \det( \xi | \dot{\xi} | \tau_{k+1} | \tau_{k+2}), & & k \in \N_3 \\
  M_{3 + k} \dot{\xi} \cdot \xi &= - 2 \sqrt{6} \,\delta \det ( \xi | \dot{\xi} | \tau_{k} | \tau_{4}), & & k \in \N_3,
\end{align},
où $\{\tau_l\}_{l \in \N_4}$ est la base orthonormée vue précédemment et $k$ est pris mod 3.

\item Si on note $\{f_k\}_{k \in \N_6}$ la base canonique de $\R^6$, ceci fournit finalement
\begin{equation}
\label{eq: net displacement}
\frac{\delta p}{2 \pi}= - 2  \sqrt{6} \alpha \sum_{k \in \N_3} \det( \xi | \dot{\xi} | \tau_{k+1} | \tau_{k+2}) f_k  - 2  \sqrt{6} \delta \sum_{k \in \N_3} \det ( \xi | \dot{\xi} | \tau_{k} | \tau_{4}) f_{k + 3},
\end{equation}
où $k$ est de nouveau pris mod 3.

\item En résumé, jusque là on a établit la dynamique du nageur dans le régime de petites courbes de contrôle à cinq paramètres près et on sait exprimer le déplacement net à 3 paramètres près.
\end{itemize}

\section{Optimisation I}


\section{Bivecteurs en $\R^4$}


\section{Optimisation II}


\section{Cas simple}


\section{Conclusion et perspectives}


\printbibliography


\end{document}