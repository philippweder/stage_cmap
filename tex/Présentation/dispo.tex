\documentclass[10pt,a4paper]{article}
\usepackage[utf8]{inputenc}
\usepackage[french]{babel}
\usepackage{amsmath}
\usepackage{amsthm}
\usepackage{amsfonts}
\usepackage{amssymb}
\usepackage{graphicx}
\usepackage[a4paper,width=133mm,top=25mm,bottom=25mm]{geometry}
\author{Philipp Weder}
\title{Soutenance de stage - dispotition}





% packages for layout
\usepackage{fancyhdr}
\pagestyle{fancy}
\fancyhf{}
\fancyhead[L]{\textit{\nouppercase{\leftmark}}}
\fancyhead[R]{\thepage}

\renewcommand{\headrulewidth}{0.5pt}


%roman enumeration
\renewcommand\labelenumi{(\roman{enumi})}
\renewcommand\theenumi\labelenumi

% font
\usepackage{pxfonts}

% bibliography
\usepackage[style=authoryear]{biblatex}

% environments
% theorem
\theoremstyle{plain}
\newtheorem{theorem}{Theorem}[section]
% corollary
\theoremstyle{plain}
\newtheorem{corollary}{Corollary}[theorem]
% lemma
\theoremstyle{plain}
\newtheorem{lemma}[theorem]{Lemma}
% remark
\theoremstyle{definition}
\newtheorem*{remark}{Remark}
% definition
\theoremstyle{definition}
\newtheorem{definition}{Definition}[section]
% example
\theoremstyle{definition}
\newtheorem{example}{Example}[section]
% proposition
\theoremstyle{plain}
\newtheorem{proposition}{Proposition}[section]

% additional packages
\usepackage{appendix}
\usepackage{amsmath}
\usepackage{amsfonts}
\usepackage{amssymb}
\usepackage{amsthm}
\usepackage{booktabs}

\newcommand{\N}{\mathbb{N}}
\newcommand{\M}{\mathcal{M}}
\newcommand{\R}{\mathbb{R}}
\newcommand{\h}{\mathcal{H}}
\newcommand{\K}{\mathcal{K}}
\DeclareMathOperator{\Skew}{Skew}
\DeclareMathOperator{\id}{id}
\newcommand{\so}{\mathfrak{so}}
\newcommand{\REF}{\mathrm{ref}}
\newcommand{\spr}{\textsc{SPr4}}
\DeclareMathOperator{\dist}{dist}
\DeclareMathOperator{\SO}{SO}
\DeclareMathOperator{\sgn}{sgn}
\DeclareMathOperator{\Aut}{Aut}
\DeclareMathOperator{\diag}{diag}
\newcommand{\chroexp}{\overset{\longrightarrow}{\exp}}
\DeclareMathOperator{\re}{Re}
\DeclareMathOperator{\Span}{span}
\newcommand{\dd}[1]{\mathrm{d}#1}
\DeclareMathOperator{\ad}{ad}
\newcommand{\T}{\mathcal{T}}

\begin{document}
\renewcommand\labelitemi{--}
\linespread{1.05}
\maketitle
\section{Mot de bienvenue}
\begin{itemize}
\item Dire bonjour à tout le monde
\end{itemize}

\section{Introduction}
\begin{itemize}
\item Introduction au sujet, mentionner Purcell
\item Expliquer pourquoi le sujet est important
\item Expliquer les questions mathématiques
\item Montrer l'historique du sujet, montrer les différents nageurs, mentionner que la contrôlabilité a déjà été démontrée
\end{itemize}

\section{Structure du projet}
\begin{itemize}
\item Modélisation et symétries
\item Approximation par petites courbes de contrôle (périodiques) et linéarisation
\item Optimisation des courbes de contrôle
\item Conjecture et perspectivess
\end{itemize}

\section{Modélisation et symétries}
\emph{Cette partie n'est pas particulièrement intéressante car il n'y a rien de nouveau, i.e. économiser du temps!}
\begin{itemize}
\item Introduire le système de contrôle, la notation, les espaces etc.
\item Montrer les condition de symétrie du système
\item Montrer les résultats pour le système de contrôle
\end{itemize}

\section{Régime des petites courbes de contrôle}
\emph{Cette partie n'est pas particulièrement intéressante car il n'y a rien de nouveau, i.e. économiser du temps!}
\begin{itemize}
\item Introduire la notation et le développement limité
\item Montrer le résultat pour le développement limité
\item Montrer le résultat pour les termes d'ordre zéro
\item Introduire la notation pour les termes  de première ordre, matrices etc.
\item Evtl. expliquer les calculs pour trouver la partie anti-symétrique
\item Reprendre le système de contrôle, introduire la linéarisation
\item Montrer le résultat sur le déplacement net
\item Mentionner pourquoi la partie symétrique n'a pas d'influence, introduire les déterminantes
\end{itemize}

\section{Optimisation I}
\begin{itemize}
\item Introduire la notation et la linéarisation
\item Réécrire le déplacement net
\end{itemize}

\section{Bivecteurs en $\R^4$}
\begin{itemize}
\item Introduire la notion de bivecteur en illustrant en $\R^3$ $\to$ figure!
\item Montrer la différence entre $\bigwedge^2 \R^3$ et $\bigwedge^2 \R^4$, expliquer les implications
\item Mentionner le lemme
\end{itemize}

\section{Optimisation II}
\begin{itemize}
\item Montrer le changement de variable et la passage en Fourier
\item Présenter la Proposition 10 et ses implications: Le déplacement net est la somme infinie de bivecteurs et soi-même un bivecteur etc.
\end{itemize}

\section{Cas simple}
\begin{itemize}
\item Réduction à dimension finie
\item Montrer que le reste passe aussi
\item Présenter Thm. 14
\end{itemize}

\section{Le cas général}
\begin{itemize}
\item Montrer le point de difficulté $\to$ réduction à la dimension finie
\item Montrer les approches pour la résolution du problème
\item MPrésenter la conjecture et expliquer pourquoi nous sommes convaincus qu'elle est vraie
\end{itemize}

\section{Conclusion et perspectives}
\begin{itemize}
\item Résumé
\item Conjecture
\item Approximation par bras longs
\end{itemize}

\printbibliography


\end{document}