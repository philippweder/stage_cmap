In summary, in sections \ref{sec: symmetries} and \ref{sec: linearization}, we have found closed expressions for the dynamics of the swimmer \textsc{SPr4} in the small stroke regime up to certain scalar parameters as well as explicit formulas to calculate the net displacement produced by a given control curve. 

Next, in section \ref{sec: optimization}, we introduced bivectors to rewrite the initial optimization problem. In doing so, we could reveal the underlying geometric structure of the optimization problem due to the properties of the space of bivectors of $\R^4$. More precisely, we have the case of the prescribed net displacement being a simple bivector and the general case, in which the prescribed net displacement is the sum of at most two orthogonal simple bivectors. In more concrete terms, the simple case corresponds to planar displacements with an additional rotation around an axis orthogonal to the plane of movement. An important situation where we face the general case are screw motions, e.g. $\delta p \mid \mid e_{12} + e_{34}$.

In the simple case, the optimal control curves could be characterized in a very similar way as for the parking 3-sphere swimmer \textsc{SPr3} in \cite{Alouges2017}. Indeed, the energy minimizing control curves turn out to be elliptical strokes in a plane of $\R^4$ defined by the prescribed net displacement, see Theorem \ref{thm:optimal control curves in the simple case}.

However, the search for the structure of the energy minimizing swimming strokes in the general case remains unsuccessful for the time being. Nonetheless, we claim that in general, the optimal control curves are contained in two completely orthogonal planes of $\R^4$. We have two reasons to believe so: On the one hand, any bivector of $\R^4$ can be written as the sum of two orthogonal simple bivectors, c.f. \cite{Lounesto2006}. Thus, we expect there to be a way of generalizing the proof of Proposition \ref{prop:simple reduction}, which appears to be non-trivial so far. On the other hand, the Euler-Lagrange equation associated to the minimization problem (c.f. \cite{DeSimone2011}) reads
\begin{equation}
G \ddot{\xi} - \Omega(\mu) \dot{\xi} = 0,
\end{equation}
with $\mu \in \R^6$ and $\Omega(\mu) = \sum_{k = 1}^{6} \mu_k M_k$. The matrix $\Omega(\mu)$ being skew symmetric, its exponential is a rotation of $\R^4$ and therefore the solution is contained one single plane or in two completely orthogonal planes of $\R^4$.