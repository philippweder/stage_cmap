\documentclass[11pt,a4paper]{article}
\usepackage[utf8]{inputenc}
\usepackage[french]{babel}
\usepackage{amsmath}
\usepackage{amsthm}
\usepackage{amsfonts}
\usepackage{amssymb}
\usepackage{graphicx}
\usepackage[a4paper,width=133mm,top=25mm,bottom=25mm]{geometry}
\author{}
\title{\textbf{Rotations et réflexions}}
\date{}





% packages for layout
\usepackage{fancyhdr}
\pagestyle{fancy}
\fancyhf{}
\fancyhead[L]{\textit{\nouppercase{\leftmark}}}
\fancyhead[R]{\thepage}

\renewcommand{\headrulewidth}{0.5pt}


%roman enumeration
\renewcommand\labelenumi{(\roman{enumi})}
\renewcommand\theenumi\labelenumi

% font
%\usepackage{pxfonts}

% bibliography
\bibliography{stage}
\usepackage[style=numeric, backend = biber]{biblatex}
\usepackage{csquotes}

% environments
% theorem
\theoremstyle{plain}
\newtheorem{theorem}{Théorème}
% corollary
\theoremstyle{plain}
\newtheorem{corollary}{Corollaire}
% lemma
\theoremstyle{plain}
\newtheorem{lemma}{Lemme}
% remark
\theoremstyle{definition}
\newtheorem*{remark}{Remarque}
% definition
\theoremstyle{definition}
\newtheorem{definition}{Définition}
% example
\theoremstyle{definition}
\newtheorem{example}{Example}
% proposition
\theoremstyle{plain}
\newtheorem{proposition}{Proposition}

% additional packages
\usepackage{appendix}
\usepackage{amsmath}
\usepackage{amsfonts}
\usepackage{amssymb}
\usepackage{amsthm}
\usepackage{booktabs}

\newcommand{\N}{\mathbb{N}}
\newcommand{\M}{\mathcal{M}}
\newcommand{\R}{\mathbb{R}}
\DeclareMathOperator{\asym}{Asym}
\DeclareMathOperator{\id}{id}
\newcommand{\h}{\mathcal{H}}
\newcommand{\so}{\mathfrak{so}}

\begin{document}
\maketitle

Il y a plusieurs façons de représenter le groupe de rotations $SO(3)$ de $\R^3$. La plus courante est d'identifier
\begin{align*}
	SO(3) = \{O \in \mathcal{O}(3) \mid \det O = 1\}.
\end{align*}
Or, le théorème d'Euler dit que pour toute rotation dans $SO(3)$ il existe une axe de rotation $\mathbf{u} \in S^2$ de sorte que l'on puisse la représenter par le vecteur d'Euler $\mathbf{\omega} = \theta \mathbf{u}$, où $\theta$ est l'angle de rotation. En effet, on peut identifier $SO(3)$ à la boule de rayon $\pi$ autour de l'origine avec points antipodaux identifiés, ce qui montre en passage que $SO(3) \simeq \R P^3$, dont on pourra déduire le groupe fondamental par exemple.

Dans la suite, on veut trouver le lien entre le vecteur d'Euler $\omega$ et la matrice de rotation $R$ associée. On se rappelle que
\begin{align*}
	\so(3) = T_{I}SO(3) = \asym_3(\R).
\end{align*}
De plus, on notera $R_{x}(\theta), R_{y}(\theta)$ et $R_{z}(\theta)$ les rotations élémentaires autour des axes $x,y$ et $z$, respectivement. On voit facilement que $\dim \asym_3(\R) = 3$ et on trouve que les matrices
\begin{align*}
	&L_x = \frac{d}{d\theta}R_x(\theta)_{\mid \theta =0} = \left(\begin{array}{ccc}
	0 & 0 & 0 \\ 
	0 & 0 & -1 \\ 
	0 & 1 & 0
	\end{array}  \right )\\
	&L_y = \frac{d}{d\theta}R_y(\theta)_{\mid \theta =0} = \left (\begin{array}{ccc}
	0 & 0 & 1 \\ 
	0 & 0 & 0 \\ 
	-1 & 0 & 0
	\end{array}  \right )\\
	&L_z = \frac{d}{d\theta}R_z(\theta)_{\mid \theta =0} = \left (\begin{array}{ccc}
	0 & -1 & 0 \\ 
	1 & 0 & 0 \\ 
	0 & 0 & 0
	\end{array}  \right ),
\end{align*}
forment une base de $\so(3)$. On montrera dans la suite que $\exp: \so(3) \to SO(3)$ est bien-défini et surjectif. En effet, pour $A \in \so(3)$ on obtient avec les propriétés de l'exponentielle que
\begin{align*}
	(e^{A})^T e^{A} = e^{A^T} e^{A} = e^{-A} e^{A} = e^{I} = I.
\end{align*}
Donc, cette application est bien-définie et en particulier on a $R_z(\theta) = e^{\theta L_z}$ et similaire pour les autres rotations similaires. On remarque que pour tout $A \in \so(3)$ et $Q \in SO(3)$ on a $QAQ^T \in \so(3)$. Soit maintenant $R \in SO(3)$ quelconque. On trouve toujours un $Q \in SO(3)$ de sorte que
\begin{align*}
	R = Q R_{z}(\theta) Q^T = Q e^{\theta L_z} Q^{T} = e^{\theta QL_zQ^{T}} = e^{\theta \mathbf{u} \cdot \mathbf{L}},
\end{align*}
avec $\mathbf{L} = (L_x, L_y, L_z)^T$ où on permet un petit abus de notation et $\mathbf{u} \in S^2$ car $Q$ est une application orthogonale. Ceci montre bien que $\exp: \so(3) \to SO(3)$ est surjectif. Réciproquement, le calcul direct montre que pour un $\mathbf{u} \in S^2$ et $\theta \in \R$ la matrice $R = \exp(\theta \mathbf{u} \cdot \mathbf{L})$ est bien la matrice de rotation associée.

Soit maintenant $S$ la réflexion au plan $yz$, i.e.
\begin{align*}
	S = \left ( \begin{array}{ccc}
	-1 & 0 & 0 \\ 
	0 & 1 & 0 \\ 
	0 & 0 & 1
	\end{array} \right ),
\end{align*}
ce que l'on appellera la forme canonique. Soit $R \in SO(3)$ l'orientation d'un corps rigide dans $\R^3$ avec vecteur d'Euler $\omega$ associé. On notera $\tilde{R} \in SO(3)$ l'orientation de l'image miroir du corps rigide avec vecteur d'Euler $\tilde{\omega}$. On s'aperçoit que $\tilde{\omega} = -S \omega$, i.e. l'axe de rotation est reflétée et au même temps le sens de rotation est inversé pour les rotations parallèles au plan de réflexion. Un petit calcul montre que
\begin{align*}
	\tilde{\omega} \cdot \mathbf{L} = S(\omega \cdot \mathbf{L})S,
\end{align*} 
dont il suit que
\begin{align*}
	\tilde{R} = \exp(\tilde{\omega} \cdot \mathbf{L}) = \exp(S(\omega \cdot \mathbf{L}) S) = S R S.
\end{align*}
Si maintenant $S'$ est une réflexion quelconque, on trouve toujours un $Q \in SO(3)$ tel que $S' = QSQ^T$. Pour un $R' \in SO(3)$ on peut écrire $R' = QRQ^T$ pour un $R \in SO(3)$. En particulier, on obtient
\begin{align*}
	\tilde{R}' = Q \tilde{R} Q^T = Q S R S Q^T = S' R' S'^T.
\end{align*}

 
\end{document}