To summarize, in sections \ref{sec: symmetries} and \ref{sec: linearization} we revealed the structure of the dynamical system governing the micro-swimmer \textsc{SPr4} with the help of the geometrical properties of Stokes' equations in the regime of small strokes. More precisely, we identified the dynamics of \textsc{SPr4}, c.f. equations (\ref{eq: dynamics first approx}), and we described the net displacement after a $2\pi$-periodic strokes up to higher order terms in Proposition \ref{prop:net displacement}. 

Next, we addressed the problem of optimal swimming for \textsc{SPr4} in the small stroke regime in section \ref{sec: optimization}. The use of the exterior algebra allowed us on the one hand to rewrite the initial optimization problem, see equations (\ref{eq: linearized energy functional}) and (\ref{eq: constraint}), in terms of exterior products of the Fourier coefficients of the strokes in Proposition \ref{prop: l2-minimization}. On the other hand, in doing so, the considerably more complex geometry of $\R^4$ in comparison with $\R^3$ manifested itself in the fact that in $\R^4$ not all bivectors are simple, i.e. can be written as a single exterior product of two vectors of $\R^4$. In fact, in the general case, every bivector of $\R^4$ can be written as the sum of two orthogonal simple bivectors.

As a consequence, we failed at reducing the infinite dimensional minimization problem in $\ell^2(\R^4) \times \ell^2(\R^4)$ of Proposition \ref{prop: l2-minimization} to a finite dimensional one in full generality. Nevertheless, in the special case of certain subspaces consisting only of simple bivectors, we managed to retrieve a result similar to the one for the swimmer \textsc{SPr3} stated in Theorem \ref{thm:optimal control curves in the simple case}. In particular, this simplified case covers all planar net displacement with an additional rotation around the axis orthogonal to the plane of movement.

In the general case, which for example is needed to determine optimal swimming strokes for screw motions, e.g. $\delta p \mid \mid e_{12} + e_{34} \simeq L_3 + \hat{e}_3$, we make the following conjecture: In general, optimal swimming strokes are ellipses contained in one or at most two completely orthogonal planes of $\R^4$ such that the frequency of rotation in one ellipse is twice the one in the other. Indeed, we have the following two reasons to believe so: First, we expect the proof of Proposition \ref{prop:simple reduction} to generalize to the general case where $\delta p = x_1 \wedge y_1 + x_2 \wedge y_2$ is the sum of two orthogonal simple bivectors by carefully choosing the lengths of the vectors $x_1, x_2, y_1$ and $y_2$ as well as the angles between them. However, the latter appears to be non-trivial for the time being. Second, if one considers the Euler-Lagrange equation associated to the initial minimization problem, c.f. \cite{DeSimone2011}, one finds that it reads
\begin{equation}
\label{eq:euler-lagrange}
G \ddot{\xi} - \Omega(\mu) \dot{\xi} = 0,
\end{equation}
where $\mu \in \R^6$ and $\Omega(\mu) = \sum_{k \in \N_6} \mu_k M_k$, with $M_k$ from section \ref{sec: symmetries}. The matrix $\Omega(\mu) \in M_{4 \times 4}(\R)$ being skew symmetric, exponentiation yields a rotation of $\R^4$ and therefore the solution to (\ref{eq:euler-lagrange}) must lie in two completely orthogonal planes by the canonical form of rotations in euclidean space.
