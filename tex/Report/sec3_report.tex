\label{sec: symmetries}
For any initial condition $p_0 = (c_0, R_0) \in \mathcal{P}$ and any control curve $\xi: I \subseteq \R \to \M$, with $I$ a neighborhood of zero, we denote $\gamma(p_0, \xi): I \to \mathcal{P}$ the solution associated to the dynamical system
\begin{equation}
\label{eq:dynamical system}
\begin{aligned}
	&\dot{p} = F(R, \xi) \dot{\xi},& & p(0) := p_0,
\end{aligned}
\end{equation}
as well as $\gamma_c(c_0, R_0, \xi)$ and $\gamma_\theta(c_0, R_0, \xi)$ its projections on $\R^3$ and $\SO(3)$, respectively, such that for any $t \in I$
\begin{equation}
	\dot{\gamma}(c_0, R_0, \xi)(t) = F(\gamma_\theta(c_0, R_0, \xi)(t), \xi(t))\dot{\xi}(t),
\end{equation}
and similarly for the projections $\gamma_c(c_0, R_0, \xi)$ and $\gamma_\theta(c_0, R_0, \xi)$.

\subsection{Rotational invariance}
Rotational invariance of the Stokes equations expresses the fact that the solution of the dynamical system (\ref{eq:dynamical system}) is invariant under rotations, i.e. that for any rotation $R \in \SO(3)$ we have for the spatial part of the solution
\begin{equation}
\label{eq:spatial rotational invariance}
	\gamma_c(c_0, R R_0, \xi)(t) = R \gamma_c (c_0, R_0, \xi)(t) + (I - R) c_0
\end{equation}
and for the angular part of the solution
\begin{equation}
\label{eq: angular rotational invariance}
	\gamma_\theta(c_0, R R_0, \xi)(t) =  R \gamma_\theta(c_0, R_0, \xi)(t)
\end{equation}
at any point in time $t \in I$. Eventually, we can rigorously state the following symmetry property of the control system (\ref{eq:dynamical system}) with respect to rotations:

\begin{condition}[Rotational invariance]
\label{cond:rotational invariance}
If $\gamma(c_0, R_0, \xi)$ is a solution of the control system (\ref{eq:dynamical system}) then so is $\gamma(c_0, R R_0, \xi)$ and (\ref{eq:spatial rotational invariance}) and (\ref{eq: angular rotational invariance}) hold.
\end{condition}

\begin{remark}
To follow the reasoning of \cite{Alouges2017}, the symmetry relations satisfied by \spr are stated as hypotheses on the solution $\gamma$. In so doing, the results work for any control system of the form (\ref{eq: control system}) and satisfying the hypotheses we state, independently of these hypotheses being guaranteed by the invariance of the Stokes equations under a certain group of transformations. 
\end{remark}
We then have
\begin{proposition}
\label{prop: rotational invariance}
Let $\xi_0 := \xi(0) \in \M$ denote the initial state of the control parameters and by $T_{\xi}\M$ the tangent space of $\M$ at $\xi$. If the control system (\ref{eq:dynamical system}) is invariant under rotations and for every $\xi \in \M$ it holds that $T_{\xi} \M \simeq \R^4$, then
\begin{equation}
	F_c(R, \xi) = R F_c(\xi) \text { and } F_\theta(R, \xi) = R F_{\theta} (R, \xi),
\end{equation}
for every $(R, \xi) \in \SO(3) \times \M$, where $F_c(\xi) := F_{c}(I, \xi)$ and $F_{\theta}(\xi) := F_{\theta}(I, \xi)$. 
\end{proposition}

\begin{proof}
On one hand, we have by definition of the dynamical system (\ref{eq:dynamical system}) that
\begin{equation}
	\dot{\gamma}_c(c_0, R, \xi) = F_c(\gamma_{\theta}(c_0, R, \xi), \xi) \dot{\xi}.
\end{equation}
On the other hand, using equation (\ref{eq:spatial rotational invariance}) and once more the definition of the dynamical system (\ref{eq: control system}), we obtain
\begin{equation}
	\dot{\gamma}_c (c_0, R \xi) = R  \dot{\gamma}_c(c_0, I, \xi) = 
	R F_c(\gamma_{\theta}(c_0, I, \xi), \xi) \dot{\xi}.
\end{equation}
Therefore, $F_c(\gamma_{\theta}(c_0, R, \xi), \xi) \dot{\xi} = R F_{c}(\gamma_{\theta}(c_0, I, \xi), \xi) \dot{\xi}$ for every $R \in \SO(3)$. Since $T_{\xi_0} \M \simeq \R^4$, evaluation of the preceding expression at $t = 0$ yields $F_{c}(R, \xi_0) = R F_{c}(I, \xi_0)$, as desired. The proof for $F_{\theta}$ is completely analogous.
\end{proof}
\subsection{Permutation of two arms}
In this section, we investigate the effect of a swap of two arms on the generic solution of the dynamical system (\ref{eq:dynamical system}).
To that end, let $P_{ij} \in M_{4 \times 4}(\R)$ denote the permutation matrix that interchanges the $i$-th and $j$-th index, which corresponds to the swap of the arms $i$ and $j$, denoted by $(||i\leftrightsquigarrow ||j$), if applied to the shape space $\M$. In addition, let $S_{ij}$ denote the reflection of $\R^3$ sending arm $||i$ onto arm $||j$ in the reference orientation $I$. Geometrical inspection of the reference tetrahedron shows that $S_{ij}$ is always a reflection at a plane containing the remaining arms $||k$ and $||l$.

Before we formulate the symmetry conditions for the interchanging of two arms, we recall some results about how rotations behave under reflections. So far, we have only regarded the orientation of \spr as a rotation matrix in $\SO(3)$. However, by Euler's rotation theorem to every such rotation matrix $R \in \SO(3)$ there exists a corresponding rotation vector $\omega \in \R^3$ which is collinear to the unique axis of rotation defined by $R$, i.e. $\omega$ is an eigenvector associated to the eigenvalue 1 of $R$. It's length is then given by the angle of rotation around this axis. The rotation vector $\omega$ is then directly related to the rotation matrix $R$ via the map $\exp: \so(3) \to \SO(3)$, where $\so(3) = T_I \SO(3) = \Skew_3(\R)$ denotes the Lie algebra over $\SO(3)$, which we will illustrate in the following paragraphs.


It is clear that $\dim \Skew_3(\R) = 3$. In particular, if $R_1(\theta), R_2(\theta)$ and $R_3(\theta)$ denote the simple rotations around the $\hat{e}_1$-, $\hat{e}_2$- and $\hat{e}_3$-axis, where $\hat{e}_1, \hat{e}_2, \hat{e}_3$ denote the canonical basis vectors of $\R^3$, then the matrices
\begin{align}
\label{eq: L1}
	&L_1 = \frac{d}{d\theta}R_1(\theta)_{\mid \theta =0} = \left(\begin{array}{ccc}
	0 & 0 & 0 \\ 
	0 & 0 & -1 \\ 
	0 & 1 & 0
	\end{array}  \right ),\\
\label{eq: L2}
	&L_2 = \frac{d}{d\theta}R_2(\theta)_{\mid \theta =0} = \left (\begin{array}{ccc}
	0 & 0 & 1 \\ 
	0 & 0 & 0 \\ 
	-1 & 0 & 0
	\end{array}  \right ),\\
\label{eq: L3}
	&L_3 = \frac{d}{d\theta}R_3(\theta)_{\mid \theta =0} = \left (\begin{array}{ccc}
	0 & -1 & 0 \\ 
	1 & 0 & 0 \\ 
	0 & 0 & 0
	\end{array}  \right ),
\end{align}
form a basis of $\so(3)$ consisting of the infinitesimal rotations around the corresponding axes. If we now write $\textbf{L} := (L_1, L_2, L_3)^T$ and allow the slight abuse of notation
\begin{align}
	\omega \cdot \mathbf{L} = \omega_1 L_1 + \omega_2 L_2 + \omega_3 L_3,
\end{align}
we find by direct computation that $\exp(\omega \cdot \mathbf{L}) = R$. This relationship allows us to formulate the behavior of the orientation of \spr under reflection and thus under permutation of two arms as we shall see later. Indeed, we have

\begin{lemma}
For any orientation of a rigid body characterized by $R \in \SO(3)$, the orientation of its mirror image under a reflection $S$ is characterized by
\begin{align}
	\tilde{R} = S R S.
\end{align}
\end{lemma}

\begin{proof}
Let us first consider the simple case where $S$ is the reflection of the $\hat{e}_1$-axis. Let $\omega$ and $\tilde{\omega}$ be the rotation vectors corresponding to $R$ and $\tilde{R}$, respectively. They are related by $\tilde{\omega} = -S \omega$, where the gain of the minus sign stems from the fact that rotation vectors are in fact pseudovectors. In other words, we not only reflect the axis of rotation but we also reverse the sense of rotation around the axis. It follows then from direct computation that
\begin{equation}
\label{eq: transformation of rotation axis}
 \tilde{\omega} \cdot \mathbf{L} = (-S\omega) \cdot \mathbf{L} = S(\omega \cdot \mathbf{L})S
\end{equation}
and thus we have
\begin{equation}
\tilde{R} = \exp(\tilde{\omega} \cdot \mathbf{L}) = \exp(S(\omega \cdot \mathbf{L})S) = SRS,
\end{equation}
as $S^{-1} = S$.

If now $S'$ is an arbitrary reflection, we always find a rotation $Q \in \SO(3)$ such that $S' = Q S Q^T$. Moreover, for any rotation $R' \in \SO(3)$ we find another $R \in \SO(3)$ such that $R' = Q R Q^T$. In particular, we have
\begin{equation}
\tilde{R}' = Q \tilde{R} Q^T = Q SRS Q^T = S' R' S'^T,
\end{equation}
as desired.
\end{proof}

With this lemma at hand, we can now finally state the following

\begin{condition}[Swap $(||i \leftrightsquigarrow ||j)$]
\label{cond:swap}
Let the initial position be $p_0 := (c_0, I)$. If $\gamma(c_0, I, P_{ij} \xi)$ is a solution of the control system (\ref{eq: control system}), then so is $\gamma(S_{ij}c_0, I, \xi)$ and the following relations hold
\begin{equation}
	\gamma_c(c_0, I, P_{ij} \xi) = S_{ij} \gamma_c(S_{ij}c_0, I, \xi)
\end{equation}
and
\begin{equation}
	\gamma_{\theta}(c_0, I, P_{ij} \xi ) = S_{ij} \gamma_{\theta} (S_{ij} c_0, I, \xi) S_{ij}.
\end{equation}
\end{condition}
To avoid chaos in our notation, we treat the the spatial and angular parts now separately. For the spatial part, we find

\begin{proposition}
\label{prop: spatial permutation invariance}
If the control system (\ref{eq:dynamical system}) is invariant under the swap $(||i \leftrightsquigarrow ||j)$ and $T_{\xi} \M \simeq \R^4$ for all $\xi \in \M$, then for all $\xi \in \M$
\begin{align}
	 F_c(P_{ij} \xi) = S_{ij} F_c(\xi) P_{ij}.
\end{align}
\end{proposition}

\begin{proof}
Let $\gamma_c(c_0, R_0, P_{ij} \xi)$ be the spatial part of any solution of the control problem (\ref{eq: control system}). The hypothesis of rotational invariance, i.e. (\ref{eq:spatial rotational invariance}), implies that
\begin{equation}
	\gamma_c(c_0, R_0, P_{ij} \xi) = R_0 \gamma_c(c_0, I, P_{ij} \xi) + (I - R_0)c_0.
\end{equation}
From condition \ref{cond:swap}, we then get
\begin{equation}
\label{eq:spatial permutation invariance int1}
	\gamma_c(c_0, R_0, P_{ij}\xi) = R_0 S_{ij} \gamma_c(S_{ij} c_0, I, \xi) + (I - R_0)c_0.
\end{equation}
As both $\gamma_c(c_0, R_0, P_{ij} \xi)$ and $\gamma_c(S_{ij} c_0, I, \xi)$ are spatial parts of solutions of the control system (\ref{eq:dynamical system}), we have on one hand using Proposition \ref{prop: rotational invariance}
\begin{equation}
\label{eq: spatial permutation invariance int2}
	\dot{\gamma_c}(c_0, R_0, P_{ij} \xi) = \gamma_{\theta}(c_0, R_0, P_{ij} \xi) F_c(P_{ij} \xi) P_{ij} \dot{\xi},
\end{equation}
and on the other hand using (\ref{eq:spatial permutation invariance int1}) and once more (\ref{eq:dynamical system}) together with Proposition \ref{prop: rotational invariance}
\begin{equation}
\label{eq: spatial permutation invariance int3}
	\dot{\gamma_c}(c_0, R_0, P_{ij} \xi) = R_0 S_{ij} \dot{\gamma}_c(S_{ij} c_0, I, \xi) = R_0 S_{ij} \gamma_\theta(S_{ij} c_0, I, \xi) F_c(\xi) \dot{\xi}.
\end{equation}
Equating (\ref{eq: spatial permutation invariance int2}) and (\ref{eq: spatial permutation invariance int3}) at $t = 0$ yields $ F_c(P_{ij} \xi_0) = S_{ij} F_c(\xi_0) P_{ij}$, since by hypothesis $T_{\xi_0} \M \simeq \R^4$. As $\xi_0$ was arbitrary, we conclude.
\end{proof}

For the angular part, we first have to choose a basis and fix some notation. Naturally, we choose the canonical basis $\mathcal{E} := (e_1, e_2, e_3, e_4)$ for $\R^4$. For $\so(3)$, we choose the basis $\mathcal{L} := (L_1, L_2, L_3)$, where the matrices $L_i$ are defined in (\ref{eq: L1}) - (\ref{eq: L3}). Then we denote the the matrix representing an arbitrary linear map $T: V \to W$ between two vector spaces $V$ and $W$ with respect to two bases $\mathcal{B}$ and $\mathcal{B}'$ by $[T]_{\mathcal{B}}^{\mathcal{B}'}$. Let $S$ be an arbitrary reflection at a plane in $\R^3$ and define the adjoint isomorphism $T_S: \so(3) \to \so(3)$ by $M \mapsto S MS$. Then the calculations leading to (\ref{eq: transformation of rotation axis}) show that in fact we have $[T_S]_{\mathcal{L}}^{\mathcal{L}} = -S$. This leads to the following 

\begin{proposition}
\label{prop: angular permutation invariance}
If the control system (\ref{eq:dynamical system}) is invariant under the swap $(||i \leftrightsquigarrow ||j)$ and $T_{\xi}\M \simeq \R^4$ for all $\xi \in \M$, then for all $\xi \in \M$
\begin{equation}
[F_{\theta}(P_{ij} \xi)]_{\mathcal{E}}^{\mathcal{L}} = - S_{ij} [F_{\theta}(\xi)]_{\mathcal{E}}^{\mathcal{L}} P_{ij}.
\end{equation}
\end{proposition}

\begin{proof}
Let $\gamma_\theta(c_0, R_0, P_{ij}\xi)$ be the angular part of any solution of the control problem (\ref{eq:dynamical system}). By the rotational invariance hypothesis, i.e. (\ref{eq: angular rotational invariance}), we have
\begin{equation}
	\gamma_\theta(c_0, R_0, P_{ij}\xi) = R_0 \gamma(c_0, I, \xi).
\end{equation}
Then Condition \ref{cond:swap} implies that
\begin{equation}
\label{eq: angular permutation invariance int1}
	\gamma_\theta(c_0, R_0,P_{ij} \xi) = R_0 S_{ij} \gamma_\theta(S_{ij}c_0, I, \xi) S_{ij}.
\end{equation}
Since both $\gamma_\theta(c_0, R_0, P_{ij} \xi)$ and $\gamma_\theta(S_{ij} c_0, I, \xi)$ are the angular parts of solutions of the control problem (\ref{eq:dynamical system}), we obtain with Proposition \ref{prop: rotational invariance} on one hand
\begin{equation}
\label{eq: angular permutation invariance int2}
	\dot{\gamma_\theta}(c_0, R_0, P_{ij} \xi)= \gamma_\theta(c_0, R_0, P_{ij} \xi) F_{\theta}(P_{ij} \xi) P_{ij } \dot{\xi}
\end{equation}
and on the other hand using (\ref{eq: angular permutation invariance int1}) and once more Proposition \ref{prop: rotational invariance}
\begin{equation}
\label{eq: angular permutation invariance int3}
	\dot{\gamma_\theta}(c_0, R_0, P_{ij} \xi) =  R_0 S_{ij} \dot{\gamma_\theta}(S_{ij} c_0, I, \xi) S_{ij} = R_0 S_{ij} \gamma_\theta(S_{ij} c_0, I, \xi) F_{\theta}(\xi) \dot{\xi} S_{ij}.
\end{equation}
Imposing equality of (\ref{eq: angular permutation invariance int2}) and (\ref{eq: angular permutation invariance int3}) at $t = 0$ yields
\begin{equation}
	F_{\theta}(P_{ij} \xi_0) P_{ij}  \dot{\xi}(0) = S_{ij} F_{\theta}(\xi_0) \dot{\xi}(0) S_{ij}.
\end{equation}
By choice of the canonical basis for $\R^4$ we clearly have $[P_{ij}]_{\mathcal{E}}^{\mathcal{E}} = P_{ij}$. Therefore, by the reasoning concerning the linear map $T_{S_{ij}}$ above, we have
\begin{equation}
	[F_{\theta}(P_{ij} \xi_0)]_{\mathcal{E}}^{\mathcal{L}}  P_{ij} [\dot{\xi}(0)]_{\mathcal{E}} = [S_{ij} F_{\theta}(\xi_0) \dot{\xi}(0) S_{ij}]_{\mathcal{L}} = -S_{ij} [F_{\theta}(\xi_0)]_{\mathcal{E}}^{\mathcal{L}} [\dot{\xi}(0)]_{\mathcal{E}}.
\end{equation}
Recalling that $T_{\xi_0} \M \simeq \R^4$ as well as the arbitrariness of $\xi_0$ finishes the proof.
\end{proof}

In the following sections, we will always understand $F_{\theta}(\xi)$ as a matrix of size $3 \times 4$ and thus, since no confusion may arise, we will abandon the slightly cumbersome notation and identify $[F_{\theta}(\xi)]_{\mathcal{E}}^{\mathcal{L}}$ with $F_{\theta}(\xi)$.





